\documentclass[handout]{beamer}

\usetheme[progressbar=frametitle]{metropolis}
\metroset{block=fill}

\subtitle{NTIN071 Automata and Grammars}
\author{Jakub Bulín (KTIML MFF UK)}

\date{Spring 2025\\ 
    \vspace{1in} 
    \begin{flushleft}
        \it \footnotesize * Adapted from the Czech-lecture slides by Marta Vomlelová with gratitude. The translation, some modifications, and all errors are mine.
    \end{flushleft}
}

%% packages

\usepackage{amsmath}
\usepackage{amssymb}
\usepackage{amsthm}
\usepackage{cancel}
\usepackage{color}
\usepackage{colortbl}
\usepackage{forest}
\usepackage[utf8x]{inputenc}
\usepackage{multicol}
\usepackage{multirow}

%% colors
\definecolor{Gray}{gray}{0.9}

%% TikZ
\usepackage{tikz}
    \usetikzlibrary{
        automata,
        arrows,
        backgrounds,
        decorations.pathmorphing,
        fit,
        positioning,
        shapes,
        shapes.geometric,
        tikzmark
    } 
    \tikzset{>=stealth',shorten >=1pt,auto,node distance=2cm}
    \tikzset{initial text={}}
    \tikzset{elliptic state/.style={draw,ellipse}}

%% amsthm
\theoremstyle{plain}
    \newtheorem*{algorithm}{Algorithm}    
    \newtheorem*{observation}{Observation}
    \newtheorem*{proposition}{Proposition}

\theoremstyle{remark}
    \newtheorem*{exercise}{Exercise}
    \newtheorem*{remark}{Remark}

%% macros
\DeclareMathOperator{\RegE}{RegE}
\DeclareMathOperator{\RL}{RL}

% Just for Lecture 2
\newcommand{\x}{$\times$}
\newcommand{\nx}{\ }





\title{Lecture 13 -- Intro to Complexity theory}


\begin{document}


\frame{\titlepage}


\begin{frame}{Recap of Lecture 12}

    \begin{itemize}
        \item the Diagonal language $L_D$ is not recursively enumerable
        \item the Universal language $L_U$, the Universal TM: simulate any $M$ on any $w$
        \item recursive languages are closed under complement
        \item Post's theorem: $L$ recursive iff both $L,\overline{L}$ are RE
        \item $L_U$, $\overline{L_D}$ are recursively enumerable but not recursive
        \item reductions between decision problems
        \item the Halting problem is undecidable        
        \item (Rice's thm: nontriv. properties of programs are undecidable)
        \item Undecidable problems about context-free grammars
        \item Source of undecidability: Post's correspondence problem
    \end{itemize}

\end{frame}


\begin{frame}{Summary of Lecture 13}

    \begin{itemize}
        \item time complexity
        \item 
    \end{itemize}

\end{frame}


\section{\sc Chapter 5: Intro to Complexity}


\section*{Time complexity}


\begin{frame}{Asymptotic notation}
    
    \textbf{Big-O notation:} Let $f,g: \mathbb{N}\to \mathbb{R}^+$. We say that \alert{$f(n)\in O(g(n))$}, if there exist $C,n_0\in \mathbb{N}^+$ such that
    $$
    (\forall n\geq n_0)\ f(n)\leq C\cdot g(n)
    $$

    e.g. $\limsup_{n\to\infty}\frac{f(n)}{g(n)}<\infty$. In that case we say that $g(n)$ is an [asymptotic] \alert{upper bound} [up to a constant multiple] for $f(n)$. 
    
    \textbf{Note:} Often the imprecise term `upper bound' is used; sometimes you will encounter $f(n)=O(g(n))$.

    For example, $f(5n^3+2n^2+22n+6)\in O(n^3)$ with $n_0=10$, $C=6$.

    \medskip

    \textbf{Little-o notation:} \alert{$f(n)\in o(g(n))$}, if for all $c>0$ there exists $n_0\in \mathbb{N}^+$ so that
    $(\forall n\geq n_0)\ f(n)<c\cdot g(n)$, i.e. $\lim_{n\to\infty}\frac{f(n)}{g(n)}=0$. Then we say $f(n)$ is [asymptotically] \alert{dominated} by $g(n)$.
    
    Analogously for $\geq$ instead of $\leq$:  $\Omega,\omega$.

\end{frame}


\begin{frame}{Classes of time complexity}

    \begin{definition}
        Let $M$ be a Turing machine that halts on every input. The \alert{time complexity} of $M$ is the function $f:\mathbb{N}\to \mathbb{N}$, where $f(n)$ is the maximum number of computation steps for inputs of length $n$.
    \end{definition}
    
    \begin{definition}
        For $t:\mathbb{N}\to \mathbb{R}^+$, \alert{$\mathrm{TIME}(t(n))$} is the class of all languages decidable by a TM of time complexity in $O(t(n))$ (i.e., always halts and for $|w|=n$ correctly answers in at most $O(t(n))$ steps).
    \end{definition}

    \textbf{NB:} Here we mean the standard, single-tape, deterministic TM.
    
\end{frame}


\begin{frame}{Example}

    \begin{example}[$L=\{0^i1^i \mid i\geq 0\}$ is in $\mathrm{TIME}(n^2)$]
        \begin{enumerate}
            \item check if the input is $0^i1^j$, if a $0$ follows a $1$, reject (time $O(n)$)
            \item return to the beginning: hidden in the constant $O(2n)=O(n)$
            \item go through the $0$s, in time $O(n^2)$
            \begin{enumerate}
                \item rewrite the next $0$ to $X$
                \item find the first $1$, rewrite to $X$
                \item return to the beginning
            \end{enumerate}
            \item if no more $0$s, check that no more $1$s remain and accept (if $1$ found, reject)  (time $O(n)$)\vspace{-6pt}
        \end{enumerate}
    \end{example}
    
    \vdots
    Can we do it faster?

\end{frame}


\begin{frame}{Can we do it faster?}

    \textbf{Idea:} ``compare the binary representations of $i$ and $j$'', $\log n$ bits, for each bit need to traverse through the word


    \begin{example}[$L=\{0^i1^i \mid  i\geq 0\}$ is also in $\mathrm{TIME}(n\log n)$]
        \begin{enumerate}
            \item check if the input is $0^i1^j$ \alert{and even length} (time $O(n)$)          
            \item iterate while there are $0$s, in time $O(n\log n)$
            \begin{enumerate}
                \item rewrite every other $0$ to $X$, then every other $1$ to $X$
                \item check if the number of remaining $0$s+$1$s is even, if not, reject                
            \end{enumerate}
            \item if no more $0$s, check that no more $1$s and accept (time $O(n)$)
            \end{enumerate}    
    \end{example}

    \vdots
    Can we do it even faster?

\end{frame}


\begin{frame}{Time complexity and regular languages}
    
    Can we do it even faster? Not really.

    \begin{theorem}
        Every language decidable in time $o(n \log n)$ [on a single-tape, deterministic TM] is regular. %problem 7.49
    \end{theorem}

    [We omit the proof.]

\end{frame}

    
\begin{frame}{Multi-tape TM}
    
    \begin{example}[Multi-tape TM for $L=\{0^i1^i \mid  i\geq 0\}$]
        \begin{itemize}
            \item copy 0s to Tape 2
            \item at first 1, switch state; erase 1 from Tape 1 \& 0 from Tape 2
            \item accept if both tapes are erased
        \end{itemize}

    \end{example}
    
    \begin{lemma}
        Every multi-tape Turing Machine with time complexity $t(n)$ is equivalent to a [single-tape] Turing Machine with time complexity $O(t^2(n))$.
    \end{lemma}
    \textbf{Proof:} Simulation of $n$ steps of a $k$-tape TM can be done in $O(n^2)$ moves since one step takes $4n+2k$ moves (heads at most $2n$ fields apart, read, write, move head marks).\hfill\qedsymbol

\end{frame}


\begin{frame}{Nondeterministic time complexity}
    
    The \alert{time complexity} of a \textbf{nondeterministic} Turing machine that always halts is defined analogously: $f(n)$ is the maximum number of steps in \textbf{any branch} of the computation tree.
    
    \begin{definition}
        For $t:\mathbb{N}\to \mathbb{R}^+$, \alert{$\mathrm{NTIME}(t(n))$} is the class of all languages decidable by a nondetermistic TM of time complexity in $O(t(n))$.
    \end{definition}
    \vspace{-6pt}
    (An NTM \alert{decides} $L$ if halts on all inputs and recognizes $L$.)

    \smallskip

    \begin{theorem}        
    Any nondeterministic TM of time complexity $t(n)\geq n$, has a determistic equivalent of time complexity in $2^{O(t(n))}$.
    \end{theorem}

    \begin{corollary}
        If $t(n)\geq n$, then $\mathrm{NTIME}(t(n))\subseteq \mathrm{TIME}(2^{O(t(n))})$.
    \end{corollary}
    
\end{frame}


\begin{frame}{Proof}

    Recall the construction: BFS of the computation graph, keep a queue of configurations to process.
    \begin{itemize}
        \item At most $d$ possible transitions for any $(q,X)\in (Q\setminus F)\times \Gamma$. 
        \item So after $k$ steps at most $d^k$ configurations.
        \item Processing one configuration can be `hidden' in the constant.
        \item Therefore the simulation is in time:
        $$
        O(t(n)d^{t(n)})=2^{O(t(n))}
        $$
        \item We need to simulate multiple tapes, but: 
        $$
        (2^{O(t(n))})^2=2^{O(2t(n))}=2^{O(t(n))}
        $$

        \vspace{-12pt}
        \hfill\qedsymbol
    \end{itemize}

\end{frame}


\section*{P vs. NP}


\begin{frame}{The class $\mathrm{P}$}

    \begin{definition}
        Let \alert{$\mathrm{P}$} (also \alert{$\mathrm{PTIME}$}) be the class of all languages decidable in \alert{polynomial time} by a [single-tape, deterministic] Turing machine:

        \vspace{-18pt}
        $$
        \mathrm{P}=\bigcup_k \mathrm{TIME}(n^k)
        $$
        \vspace{-9pt}
    \end{definition}
    \vspace{-9pt}
    \begin{itemize}
        \item Path in a graph
        \item Primality of an integer (Agrawal–Kayal–Saxena 2002)
        \item Linear programming
        \item Horn-SAT
    \end{itemize}
    \vspace{-3pt}
    (The last two are $\mathrm{P}$-complete under $\mathrm{LOGSPACE}$ reductions.)

    \medskip

    \begin{theorem}[$CFL\subseteq P$]
        Every context free language belongs to $\mathrm{P}$.
    \end{theorem}  
    
    \vspace{-10pt}
    \textbf{Proof:}
        Take a ChNF grammar for $L$. Given input $\omega$, run the CYK algorithm (polynomial, in $O(n^3)$).\hfill\qedsymbol

\end{frame}


\end{document}
