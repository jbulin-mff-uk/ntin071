\begin{frame}{Časová složitost}
    \begin{definition}[časová složitost]
    Mějme Turingův stroj $M$, který zastaví na každém vstupu.
    \alert{Časová složitost} $M$ je funkce $f:\mathbb{N}\to \mathbb{N}$, kde $f(n)$ je maximální počet kroků výpočtu $M$ nad vstupy délky $n$.
    \end{definition}
    %\pause
    \begin{definition}[(Asymptotická) horní hranice $O(g(n))$]
    Mějme funkce $f,g: \mathbb{N}\to \mathbb{R}^+$. Říkáme, že \alert{$f(n)=O(g(n))$}, pokud existují $c,n_0\in \mathbb{N}^+$ taková, že:
    $$\forall n\geq n_0 \hbox{ platí  } f(n)\leq c\cdot g(n)\hbox{.}
    $$
    
    V takovém případě říkáme, že $g(n)$ je (asymptotická) \alert{horní hranice} pro $f(n)$. (Slovo asymptotická vyjadřující ignorování prvních $n_0$ i konstanty $c$ se zpravidla vynechává.)
    \end{definition}
    \begin{itemize}
        \item[] Reálná čísla jsou tam kvůli logaritmu.
    \end{itemize}
    \end{frame}
    
    
    
    \newcommand{\T}{TIME}
    \begin{frame}%{Třída časové složitosti}
    \begin{definition}[třída časové složitosti]
    Mějme funkci $t: \mathbb{N}\to \mathbb{R}^+$. Definujeme \alert{třídu časové složitosti $\T(t(n))$} jakožto množinu všech jazyků, které jsou rozhodnutelné turingovým strojem v~čase $O(t(n))$.
    
    \end{definition}
    
    \begin{lemma}
    Mějme funkci $t: \mathbb{N}\to \mathbb{R}^+$, $t(n)\geq n$. Každý vícepáskový Turingův stroj s časem $t(n)$ má jednopáskový ekvivalent $O(t^2(n))$.
    \end{lemma}
    \begin{definition}[doba běhu nedeterministického TM]
    Mějme {\bf nedeterministický} Turingův stroj $N$, který zastaví na každém vstupu.
     \alert{Doba běhu $N$} je funkce $f: \mathbb{N}\to \mathbb{N}$, kde $f(n)$ je maximální počet kroků který $N$ potřebuje v jakékoli větvi výpočtu nad jakýmkoli vstupem délky $n$.
    
    \end{definition}
    
    \begin{lemma}
    Mějme funkci $t: \mathbb{N}\to \mathbb{R}^+$, $t(n)\geq n$. Každý nedeterministický Turingův stroj s~časem $t(n)$ má deterministický ekvivalent $2^{O(t(n))}$.
    \end{lemma}
    
    \end{frame}
    
    
    
    
    \begin{frame}%{Polynomiální problémy $P$}
    \begin{definition}[třída $P$]
    Definujeme \alert{$P$ ($PTIME$)} \alert{třídu jazyků rozhodnutelných v polynomiálním čase} jednopáskovým deterministickým Turingovým strojem. Tedy:
    $$P=\bigcup_k TIME(n^k)\hbox{.}$$
    \end{definition}
    
    \begin{theorem}[$CFL\subseteq P$]
    Každý bezkontextový jazyk patří to $P$.
    \end{theorem}
    \begin{itemize}
        \item CYK algorithmu je polynomiální.
    \end{itemize}
    
    \end{frame}
    
    
    
    
    
    
    
    
    
    \begin{frame}{Verifikátory, třída $NP$}
    
    \begin{definition}[Verifikátor]
    \alert{Verifikátor} jazyka $L$ je algoritmus $V$, kde:
    $$L=
    \{ w\ | \ V \hbox{pro nějaké $c\in \Sigma^*$ přijímá }\langle w,c\rangle \}\hbox{.}
    $$
    
    Časová složitost verifikátoru se měří pouze vzhledem k délce $w$, \alert{polynomiální verifikátor} rozhoduje v čase polynomiálním vzhledem k $|w|$.
    
    Jazyk $L$ je \alert{polynomiálně verifikovatelný}, pokud má polynomiální verifikátor.
    
    \alert{Třída jazyků rozhodnutelných v polynomiálním čase} \alert{$NP$ }  je tvořena jazyky s~polynomiálním verifikátorem.
    \end{definition}
    
    
    \begin{definition}[$NP$]
    Mějme funkci $t: \mathbb{N}\to \mathbb{R}^+$. Definujeme třídu
    $$\hbox{\alert{$NTIME(t(n))$}}=
    \{ L\ | \ L \hbox{ jazyk rozhodnutelný nedeterminist. TM v čase $O(t(n))$.}\}
    $$
    
    \end{definition}
    
    \end{frame}
    
    \begin{frame}{ Třída $NP$}
    \begin{theorem}
    $$NP=\bigcup_k NTIME(n^k)\hbox{.}$$
    \end{theorem}
    \begin{itemize}
        \item Idea důkazu: převedeme verifikátor na NTM a opačně.
        \item NTM uhodne certifikát a simuluje verifikátor.
        \item Verifikátor bere přijímající větev NTM jakožto certifikát.
    \end{itemize}
    \begin{proof}[verif. $\Rightarrow \bigcup$ ]
    \begin{itemize}
        \item Předpokládáme $L\in NP$.
        \item Hledáme nedeterministický TM $M$.
        \item Vezmeme verifikátor $V$ z definice $NP$. Nechť rozhoduje $L$ v čase $n^k$.
        \item $M$ na vstupu $w$ délky $n$:
        
        \begin{itemize}
            \item Nedeterministicky uhodne řetězec $c$ délky nanejvýš $n^k$.
            \item Spustí $V$ na vstupu $\langle w, c\rangle$.
    \item Pokud $V$ přijme, $M$ také přijme, jinak nepřijímá.
        \end{itemize}
    \end{itemize}
    \end{proof}
    \end{frame}
    
    \begin{frame}
    \begin{proof}[ $\bigcup\Rightarrow $ verifikátor ]
    \begin{itemize}
        \item Předpokládáme $L=L(M)$ rozhodnutelný nějakým NTM $M$ v polynomiálním čase.
        \item Hledáme verifikátor $V$.
        \item $V$ na vstupu $\langle w, c\rangle$:
        
        \begin{itemize}
            \item Simuluje $M$ na vstupu $w$, v bodech větvení vybere větev podle $c$.
    \item Pokud tato větev NTM přijme, $V$ přijme, jinak nepřijímá.
        \end{itemize}
    \end{itemize}
    \end{proof}
    \end{frame}
    
    
    
    
    
    \begin{frame}{Převoditelnost v polynomiálním čase}
    \begin{definition}[polynomiálně vyčíslitelná funkce]
    Funkci $f: \Sigma^*\to \Sigma^*$ nazveme \alert{polynomiálně vyčíslitelnou}, pokud existuje Turingův stroj $M$, který pro každý vstup $w$ v polynomiálním čase zastaví s $f(w)$ na pásce.
    
    Jazyk $A$ je \alert{převoditelný v polynomiálním čase} na jazyk $B$, $A\leq_P B$, pokud existuje funkce $f: \Sigma^*\to \Sigma^*$ vyčíslitelná v polynomiálním čase a pro každé $w\in \Sigma^*$
    $$w\in A \Leftrightarrow f(w)\in B\hbox{.}
    $$
    Funkci $f$ pak nazýváme \alert{polynomiální redukci $A$ do $B$}.
    \end{definition}
    
    \begin{definition}[$NP$ úplnost]
    Jazyk $B$ je \alert{$NP$ úplný}, pokud je $NP$ a každý jazyk $A\in NP$ je na $B$ polynomiálně převeditelný.
    \end{definition}
    
    \end{frame}
    
    \begin{frame}{}
    \begin{theorem}
    Pokud $B$ je NP-úplný a $B\in P$, pak $P=NP$.
    \end{theorem}
    \begin{proof}
    Přímý důsledek definice polynomiální převoditelnosti a $NP$.
    \end{proof}
    \begin{theorem}
    Pokud $B$ je NP-úplný a $B\leq_P C$ pro nějaké $C\in NP$, pak $C$ je $NP$-úplný.
    \end{theorem}
    \begin{proof}
    Převod problému na $B$ dále převedeme na $C$, stačí polynomiální čas.
    \end{proof}
    
    
    \end{frame}
    
    
    
    
    
    %\begin{frame}{Redukce}
    %\begin{minipage}{0.57\textwidth}
    %\begin{definition}[Redukce]
    %\end{definition}
    %%\pause
    %\end{minipage}
    %\begin{minipage}{0.01\textwidth}
    %\end{minipage}
    %\begin{minipage}{0.41\textwidth}
    %\includegraphics[width=\textwidth]{reduction.PNG}
    %\end{minipage}
    %\begin{example}
    %\begin{minipage}{0.48\textwidth}
    %Redukce  TM pro $L_d$ na  TM pro $\overline{L_u}$:
    %\includegraphics[width=\textwidth]{LU.PNG}
    %\end{minipage}
    %\begin{minipage}{0.01\textwidth}
    %\end{minipage}
    %\begin{minipage}{0.41\textwidth}
    %\begin{itemize}
        %\item $P_1=$ Nepřijímá TM reprezentovaný $w$ vstupní slovo $w$?
        %\item $P_2=$ Nepřijímá TM reprezentovaný $M$ vstupní slovo $w$?
        %\end{itemize}
    %\end{minipage}
    %\end{example}
    %\end{frame}
    
    
    \begin{frame}{Cook-Levin-ova věta}
    \begin{definition}[SAT< 3SAT]
    Formuli $\phi$ nazveme \alert{3-cnf formule}, pokud je formule výrokové logiky v CNF, kde v každé klauzuli jsou právě tři literály.
    
    Formule \alert{je splnitelná}, pokud existuje takové ohodnocení výrokových proměnných, že je hodnota formule TRUE.
    
    Problém \alert{3SAT} je pro každou 3-cnf formuli rozhodnout, zda je splnitelná, tj.
    $$3SAT=\{\phi \ |\  \phi \hbox{ je splnitelná 3-cnf formule}\}\hbox{.}
    $$
    
    Problém \alert{SAT} je pro každou booleovskou formuli v rozhodnout, zda je splnitelná, tj.
    $$SAT=\{\phi \ |\  \phi \hbox{ je splnitelná formule}\}\hbox{.}
    $$
    \end{definition}
    
    \begin{theorem}[Cook-Levin-ova věta]
    $SAT$ je $NP$-úplný.
    \end{theorem}
    \begin{itemize}
        \item idea důkazu úplnosti: převedeme výpočet Turingova stroje na SAT.
    \end{itemize}
    \end{frame}
    
    
    \begin{frame}{}
    \begin{proofs}
    \begin{itemize}
        \item SAT is NP
        
        \begin{itemize}
            \item Nedeterministický TM uhodne správné ohodnocení a v polynomiálním čase ověří, je je pro něj formule pravdivá.
        \end{itemize}
        \item SAT je NP-úplný
        
        \begin{itemize}
            \item Vezmeme libovolný $L\in NP$
            \item nechť $M$ je nedeterministický TM který rozhoduje jazyk $L$ v čase $n^k-3$ pro nějaké $k$.
            \item Vytvoříme tabulku (tableau) $n^k\times n^k$, každý řádek odpovídá konfiguraci $M$ na vstupu $w$
            \item můžeme předpokládat (opatřit) každou konfiguraci ohraničenou zarážkami $\#$.
        
            \begin{tabular}{|c|c|c|c|c|c|c|c|c|c|c|}\hline
            \# & $q_0$ & $w_1$ &$w_2$ & \ldots &$w_n$ & \_  &\_  &  \ldots &\# \\\hline
            \# &       &       &       &       &       &       &       &  &\# \\\hline
            \# & \vdots &       &       &       &       &       &       &&\# \\
            \# & \vdots &       &       &       &       &       &       &&\# \\\hline
            \# &       &       &       &       &       &       &       &  &\# \\\hline
                
            \end{tabular}
            \item Výpočet budeme skládat po okýnkách $2\times 3$.
            \begin{tabular}{|c|c|c|}\hline
     & & \\\hline
     & & \\\hline
            \end{tabular}
            
        \end{itemize}
    \end{itemize}
    \end{proofs}
    \end{frame}
    
    
    \begin{frame}{}
    \begin{proofm}{}
    \begin{itemize}
        \item Vybraná dovolená okénka, ($a,b,c,d\in \Gamma$)
        
        \begin{tabular}{ccc}
        \\[0.1cm]
         \begin{minipage}{3cm} 
          $\delta(q_1,b)\ni (q_2,c,L)$\\
            \begin{tabular}{|c|c|c|}\hline
              a & $q_1$ & b\\\hline
                $q_2$ & a & c\\\hline
            \end{tabular}
            \end{minipage}
     &
         \begin{minipage}{3cm} 
          $\delta(q_1,b)\ni (q_2,c,R)$\\
            \begin{tabular}{|c|c|c|}\hline
              a & $q_1$ & b\\\hline
                a & c & $q_2$ \\\hline
            \end{tabular}
            \end{minipage}
    
     & 
         \begin{minipage}{3cm} 
          $\delta(q_1,b)\ni (q_2,c,R)$\\
            \begin{tabular}{|c|c|c|}\hline
              d & a & $q_1$ \\\hline
                d & a & c  \\\hline
            \end{tabular}
            \end{minipage}
    \\[0.5cm]
         \begin{minipage}{3cm} 
          přenos beze změny\\
            \begin{tabular}{|c|c|c|}\hline
              \# & a & b\\\hline
              \# & a & b\\\hline
            \end{tabular}
            \end{minipage}
     &
         \begin{minipage}{3cm} 
          $\delta(\_,\_)\ni (q_2,\_,L)$\\
            \begin{tabular}{|c|c|c|}\hline
              a & b & c\\\hline
                a & b & $q_2$ \\\hline
            \end{tabular}
            \end{minipage}
     & 
         \begin{minipage}{3cm} 
          $\delta(\_,\_)\ni (\_,c,L)$\\
            \begin{tabular}{|c|c|c|}\hline
              a & b & d\\\hline
                c & b & d \\\hline
            \end{tabular}
            \end{minipage}
        \\[0.5cm]
    \end{tabular}
    
    \begin{itemize}
        \item Přenos beze změny všude, kde není v okolí stav (hlava)
        \item na každém řádku nejvýše jeden stav
        \item okénko musí být částí povoleného přechodu
        \item rozbor technický, udělali za nás jiní.
        \item {\bf Tvrzení:} Pokud je první řádek tabulky počáteční konfigurace a každé okénko je legální, pak každý řádek odpovídá legální konfiguraci dosažitelné jedním krokem z předchozího řádku.
        
        \begin{itemize}
            \item V horním řádku není stav, pak se prostřední symbol musí opsat beze změny.
            \item V horním řádku stav vprostřed: okénko garantuje korektnost přepisu obou stran.
        \end{itemize}
    \end{itemize}
        
        
    \end{itemize}
    \end{proofm}
    \end{frame}
    
    
    \begin{frame}{}
    \begin{proofe}
    \begin{itemize}
        \item Z tabulky vytvoříme formuli $\phi=\phi_{cell}\ \&\ \phi_{start}\ \&\ \phi_{move}\ \&\  \phi_{accept}$.
        \item pro každé políčko tabulky $(i,j) $ a písmeno $a\in \Gamma\cup Q \cup \{\#\}$ vytvoříme výrokovou proměnnou $x_{i,j,a}$
        $$
        \phi_{cell}=\bigwedge_{1\leq i,j \leq n^k} \left[ (\bigvee_{a \in \Gamma\cup Q \cup \{\#\}} x_{i,j,a}) \ 
        \&\bigwedge_{s\neq t\in \Gamma\cup Q \cup \{\#\}} (\overline{x_{i,j,s}}\vee \overline{x_{i,j,t}})\right]\hbox{\hspace{0.1cm}   \#právě jedno $a$}
        $$
        $$\phi_{start}= x_{1,1,\#}\& x_{1,2,q_0} \& x_{1,3,w_1}\& x_{1,4,w_2}\& \ldots \& x_{1,n+2,w_n}\& x_{1,n+3,\_}\&\ldots \& x_{1,n^k,\#}
        $$
    $\phi_{accept}= \bigvee_{1\leq i,j \leq n^k} x_{i,j,q_{accept}}
    $
        \item Celková formule $\phi_{move}$ bude konjunkce, že každé okénko s horním středem na pozici $i,j$ je legální
    $$\phi_{move}=\bigwedge_{1\leq i < n^k, 1<j<n^k} \phi_{i,j}\hbox{\hspace{1cm}   \# okénko $(i,j)$ je legální}
    $$
    \item  legálnost okénka $(i,j)$ zajistíme disjunkcí legálních okének
    $$\phi_{i,j}=\bigvee_{(a_1,\ldots,a_6) \in LEGAL} (x_{i,j-1,a_1}\& x_{i,j,a_2}\& x_{i,j+1,a_3}\& 
    x_{i+1,j-1,a_4}\& x_{i+1,j,a_5}\& x_{i+1,j+1,a_6} )
    $$
    \end{itemize}
    \end{proofe}
    \end{frame}
    
    \begin{frame}{}
    \begin{proofe}
    \begin{itemize}
    \item {\bf Tvrzení:} Převod má polynomiální složitost, konkrétně $O(n^{2k}\log n)$.
    \begin{itemize}
        \item $\phi_{cell}\in O(n^{2k})$, procházíme dvojice buněk
        \item $\phi_{start}\in O(n^{2})$, procházíme první řádek
        \item $\phi_{move},\phi_{accept}\in O(n^{2k})$, 
    \begin{itemize}
        \item 	$|\delta|$ je pro nás konstanta, protože nezávisí na délce vstupu
    \item počet buněk $n^{2k}$, pro každou konstantní velikost formule.
    \end{itemize}
        
    \end{itemize}
    \item pro šťouraly $\log n$ pro zápis indexu proměnných, jehož délka závisí na $n$.
    \end{itemize}
    \end{proofe}
    \end{frame}
    
    \begin{frame}{co-NP, Tautologičnost}
    \begin{definition}
    Jazyk $L\subseteq \Sigma^*$ patří do třídy \alert{co-NP} právě když jeho doplněk $\Sigma^*-L$ patří do NP.
    \end{definition}
    \begin{itemize}
        \item P je částí NP i co-NP.
        \item Domníváme se, že NP-úplné problémy nejsou v co-NP.
        
        \begin{itemize}
            \item pokud $P=NP$, tak jsou.
        \end{itemize}
    \end{itemize}
    \begin{definition}[tautologičnost]
    Problém, zda je daná formule výrokové logiky, nazýváme \alert{tautologičnost TAUT}.
    \end{definition}
    \begin{theorem}
    Problém tautologičnosti TAUT je co-NP.
    \end{theorem}
    \begin{itemize}
        \item Důkaz z pozorování, že doplněk TAUT je SAT a SAT je v NP.
        \item Doplněk SAT je otázka, jestli negace dané formule je tautologie.
    \item Doplněk SAT je co-NP.
    \item SAT rozhoduje všechny formule, tedy i jejich negace.
    \end{itemize}
    \end{frame}
    
    
    \begin{frame}{Prostorová složitost}
    \begin{itemize}
        \item Podobně jako časovou složitost měříme prostor potřebný k výpočtu.
        \item Turingův stroj je jednoduchý a dost podobný reálným počítačům, proto je často používán k definici tříd složitosti.
    \end{itemize}
    \begin{definition}[prostorová složitost]
    Pro deterministický Turingův stroj $M$, který zastaví na každém vstupu, je
    \alert{prostorová složitost} $M$ funkce $f:\mathbb{N}\to \mathbb{N}$, kde $f(n)$ je maximální počet buněk pásky, které  $M$ přečte při jakémkoli vstupu délky $n$.
    
    
    Pro nedeterministický Turingův stroj $M$, který všechy větve výpočtu zastaví na každém vstupu, je 
    \alert{prostorová složitost} $M$ je funkce $f:\mathbb{N}\to \mathbb{N}$, kde $f(n)$ je maximální počet buněk pásky, které  $M$ přečte při jakémkolivstupu délky $n$ na libovolné větvi výpočtu.
    
    \end{definition}
    \end{frame}
    
    \begin{frame}{Třídy prostorové složitosti}
    \begin{definition}[třídy prostorové složitosti]
    Mějme funkci $t: \mathbb{N}\to \mathbb{R}^+$. Definujeme \alert{třídy prostorové složitosti } $SPACE(f(n))$ a $NSPACE(f(n))$: %jakožto množinu všech jazyků, které jsou rozhodnutelné turingovým strojem v~čase $O(t(n))$.
    \begin{itemize}
        \item $SPACE(f(n))=\{L \ |\  L \hbox{je jazyk rozhodnutelný v prostoru $O(f(n))$ deterministickým TM}\}$,
        \item $NSPACE(f(n))=\{L \ |\  L \hbox{je jazyk rozhodnutelný v prostoru $O(f(n))$ nedeterministickým TM}\}$.
    \end{itemize}
    \end{definition}
    
    \begin{theorem}
    $$P \subseteq NP \subseteq PSPACE \subseteq NPSPACE \subseteq EXPTIME = \bigcup_k TIME(2^{n^k})\hbox{.} $$
    \end{theorem}
    
    \end{frame}
    
    
    