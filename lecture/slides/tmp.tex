
  
    
    \begin{frame}{Cook-Levin-ova věta}
    
    \begin{theorem}[Cook-Levin-ova věta]
    $SAT$ je $NP$-úplný.
    \end{theorem}
    \begin{itemize}
        \item idea důkazu úplnosti: převedeme výpočet Turingova stroje na SAT.
    \end{itemize}
    \end{frame}
    
    
    \begin{frame}{}
    \begin{proofs}
    \begin{itemize}
        \item SAT is NP
        
        \begin{itemize}
            \item Nedeterministický TM uhodne správné ohodnocení a v polynomiálním čase ověří, je je pro něj formule pravdivá.
        \end{itemize}
    \pause
     \item SAT je NP-úplný
        
        \begin{itemize}[<+->]
            \item Vezmeme libovolný $L\in NP$
            \item nechť $M$ je nedeterministický TM který rozhoduje jazyk $L$ v čase $n^k-3$ pro nějaké $k$ 
      \item (zde pro jednoduchost NTM s jednostrannou páskou)
            \item Vytvoříme tabulku (tableau) $n^k\times n^k$, každý řádek odpovídá konfiguraci $M$ na vstupu $w$
            \item můžeme předpokládat (opatřit) každou konfiguraci ohraničenou zarážkami $\#$.
        
            \begin{tabular}{|c|c|c|c|c|c|c|c|c|c|c|}\hline
            \# & $q_0$ & $w_1$ &$w_2$ & \ldots &$w_n$ & \_  &\_  &  \ldots &\# \\\hline
            \# &       &       &       &       &       &       &       &  &\# \\\hline
            \# & \vdots &       &       &       &       &       &       &&\# \\
            \# & \vdots &       &       &       &       &       &       &&\# \\\hline
            \# &       &       &       &       &       &       &       &  &\# \\\hline
                
            \end{tabular}
            \item Výpočet budeme skládat po okýnkách $2\times 3$.
            \begin{tabular}{|c|c|c|}\hline
     & & \\\hline
     & & \\\hline
            \end{tabular}
            
        \end{itemize}
    \end{itemize}
    \end{proofs}
    \end{frame}
    
    
    \begin{frame}{}
    \begin{proof}{}
    \begin{itemize}
        \item Vybraná dovolená okénka, ($a,b,c,d\in \Gamma$)
        
        \begin{tabular}{ccc}
        \\[0.1cm]
         \begin{minipage}{3cm} 
          $\delta(q_1,b)\ni (q_2,c,L)$\\
            \begin{tabular}{|c|c|c|}\hline
              a & $q_1$ & b\\\hline
                $q_2$ & a & c\\\hline
            \end{tabular}
            \end{minipage}
     &
         \begin{minipage}{3cm} 
          $\delta(q_1,b)\ni (q_2,c,R)$\\
            \begin{tabular}{|c|c|c|}\hline
              a & $q_1$ & b\\\hline
                a & c & $q_2$ \\\hline
            \end{tabular}
            \end{minipage}
    
     & 
         \begin{minipage}{3cm} 
          $\delta(q_1,b)\ni (q_2,c,R)$\\
            \begin{tabular}{|c|c|c|}\hline
              d & a & $q_1$ \\\hline
                d & a & c  \\\hline
            \end{tabular}
            \end{minipage}
    \\[0.5cm]
         \begin{minipage}{3cm} 
          přenos beze změny\\
            \begin{tabular}{|c|c|c|}\hline
              \# & a & b\\\hline
              \# & a & b\\\hline
            \end{tabular}
            \end{minipage}
     &
         \begin{minipage}{3cm} 
          $\delta(\_,\_)\ni (q_2,\_,L)$\\
            \begin{tabular}{|c|c|c|}\hline
              a & b & c\\\hline
                a & b & $q_2$ \\\hline
            \end{tabular}
            \end{minipage}
     & 
         \begin{minipage}{3cm} 
          $\delta(\_,\_)\ni (\_,c,L)$\\
            \begin{tabular}{|c|c|c|}\hline
              a & b & d\\\hline
                c & b & d \\\hline
            \end{tabular}
            \end{minipage}
        \\[0.5cm]
    \end{tabular}
    
    \begin{itemize}[<+->]
        \item Přenos beze změny všude, kde není v okolí stav (hlava)
        \item na každém řádku nejvýše jeden stav
        \item okénko musí být částí povoleného přechodu
        \item rozbor technický, udělali za nás jiní.
        \item {\bf Tvrzení:} Pokud je první řádek tabulky počáteční konfigurace a každé okénko je legální, pak každý řádek odpovídá legální konfiguraci dosažitelné jedním krokem z předchozího řádku.
        
        \begin{itemize}
            \item V horním řádku není stav, pak se prostřední symbol musí opsat beze změny.
            \item V horním řádku stav vprostřed: okénko garantuje korektnost přepisu obou stran.
        \end{itemize}
    \end{itemize}
        
        
    \end{itemize}
    \end{proof}
    \end{frame}
    
    
    \begin{frame}{}
    \begin{proofe}
    \begin{itemize}[<+->]
        \item Z tabulky vytvoříme formuli $\phi=\phi_{cell}\ \&\ \phi_{start}\ \&\ \phi_{move}\ \&\  \phi_{accept}$.
        \item pro každé políčko tabulky $(i,j) $ a písmeno $a\in \Gamma\cup Q \cup \{\#\}$ vytvoříme výrokovou proměnnou $x_{i,j,a}$
        $$
        \phi_{cell}=\bigwedge_{1\leq i,j \leq n^k} \left[ \left(\bigvee_{a \in \Gamma\cup Q \cup \{\#\}} x_{i,j,a}\right) \ 
        \&\bigwedge_{s\neq t\in \Gamma\cup Q \cup \{\#\}} (\overline{x_{i,j,s}}\vee \overline{x_{i,j,t}})\right]\hbox{\hspace{0.1cm}   \#právě jedno $a$}
        $$
     \pause
        $$\phi_{start}= x_{1,1,\#}\& x_{1,2,q_0} \& x_{1,3,w_1}\& x_{1,4,w_2}\& \ldots \& x_{1,n+2,w_n}\& x_{1,n+3,\_}\&\ldots \& x_{1,n^k,\#}
        $$
     \pause
    $\phi_{accept}= \bigvee_{1\leq i,j \leq n^k} x_{i,j,q_{accept}}
    $
        \item Celková formule $\phi_{move}$ bude konjunkce, že každé okénko s horním středem na pozici $i,j$ je legální
    $$\phi_{move}=\bigwedge_{1\leq i < n^k, 1<j<n^k} \phi_{i,j}\hbox{\hspace{1cm}   \# okénko $(i,j)$ je legální}
    $$
    \item  legálnost okénka $(i,j)$ zajistíme disjunkcí legálních okének
    $$\phi_{i,j}=\bigvee_{(a_1,\ldots,a_6) \in LEGAL} (x_{i,j-1,a_1}\& x_{i,j,a_2}\& x_{i,j+1,a_3}\& 
    x_{i+1,j-1,a_4}\& x_{i+1,j,a_5}\& x_{i+1,j+1,a_6} )
    $$
    \end{itemize}
    \end{proofe}
    \end{frame}
    
    \begin{frame}{}
    \begin{proofe}
    \begin{itemize}[<+->]
    \item {\bf Tvrzení:} Převod má polynomiální složitost, konkrétně $O(n^{2k}\log n)$.
    \begin{itemize}
        \item $\phi_{cell}\in O(n^{2k})$, procházíme dvojice buněk
        \item $\phi_{start}\in O(n^{2})$, procházíme první řádek
        \item $\phi_{move},\phi_{accept}\in O(n^{2k})$, 
    \begin{itemize}
    %	\item 	$|\delta|$ je pro nás konstanta, protože nezávisí na délce vstupu
    \item počet buněk $n^{2k}$, pro každou konstantní velikost formule.
    \end{itemize}
        
    \end{itemize}
    \item pro šťouraly $\log n$ pro zápis indexu proměnných, jehož délka závisí na $n$.
    \end{itemize}
    \end{proofe}
    \end{frame}
    
    

    