\documentclass[handout]{beamer}

\usepackage{a4wide}
\usepackage{amsmath}
\usepackage{amssymb}
\usepackage{amsthm}
\usepackage[czech]{babel}
\usepackage{bookmark}
\usepackage{enumerate}
\usepackage[T1]{fontenc}
\usepackage{hyperref}
\usepackage[utf8]{inputenc}
\usepackage{lmodern}
\usepackage{multicol}
\usepackage{tikz}
    \usetikzlibrary{automata, arrows, positioning}


\theoremstyle{definition}
    \newtheorem{problem}{Příklad}




\title{Lecture 6 -- Chomsky Normal Form, Pumping lemma for context-free languages}


\begin{document}


\frame{\titlepage}


\begin{frame}{Recap of Lecture 5}
	
	\begin{itemize}
		\item Grammars: general, context-sensitive, context-free, right-linear (regular) -- Chomsky hierarchy
		\item The language of a grammar, derivation
		\item Right-linear grammars correspond to FA (and so do left/linear)
		\item Linear grammars are stronger
		\item Context-free grammars: parse tree and its yield
		\item (un)ambiguous grammars, inherently ambiguous languages
	\end{itemize}

\end{frame}


\section{2.6 Chomsky Normal Form}


\begin{frame}{Chomsky normal form}
	
	The \alert{Chomsky normal form (ChNF)} of a context-free grammar:
	
	\begin{itemize}
		\item all rules of the form \alert{$A\rightarrow BC$} or  \alert{$A\rightarrow a$} ($A,B,C\in V$, $a\in T$)
		\item no \alert{useless} symbols
	\end{itemize}

	\begin{theorem}
		For every context-free language $L$ such that $L\setminus \{\epsilon\}\neq \emptyset$ there exists a grammar in ChNF that generates $L\setminus \{\epsilon\}$.
	\end{theorem}
	
	Applications:
	
	\begin{itemize}
		\item Test membership in $L$: the \alert{CYK algorithm} (Sakai 1962) 
		\item Prove the \alert{Pumping lemma for context-free languages}
	\end{itemize}

\end{frame}


\begin{frame}{Converting to ChNF}

	Take any context-free grammar for $L$ and simplify (\alert{in this order!}):
	
	\begin{enumerate}
		\item eliminate \alert{$\epsilon$-productions} $A\rightarrow\epsilon$
		\hfill{\small [here we lose $\epsilon\in L$]}
		\item eliminate \alert{unit productions} $A\rightarrow B$
		\item eliminate \alert{useless} symbols
		\begin{itemize}
			\item[3a.] \alert{unreachable} \hfill[from the start symbol]
			\item[3b.] \alert{nongenerating} \hfill[a word over terminals]
		\end{itemize} 
	\end{enumerate}

	Now we have a \alert{reduced} grammar. To get to ChNF, we further:
		
	\begin{enumerate}\setcounter{enumi}{3}
		\item \alert{separate} terminals from bodies
		\item \alert{break up} longer bodies
	\end{enumerate}

\end{frame}


\begin{frame}{Step 1: Eliminate $\epsilon$-productions}

	A variable $A\in V$ is \alert{nullable} if $A\Rightarrow^* \epsilon$. An algorithm to find them:
	\begin{itemize}
		\item[\textbf{basis:}] for every $\epsilon$-production $A\rightarrow \epsilon$ mark $A$ as nullable
		\item[\textbf{induct:}] if $B\rightarrow C_1 \ldots C_k\in\mathcal P$ where all $C_i$ are nullable, $B$ is nullable
	\end{itemize}

	\textbf{To eliminate $\epsilon$-productions:} 1. find nullable variables, 2. remove $\epsilon$-productions, 3. process every production $A\rightarrow X_1\ldots X_k \in \mathcal P$:
	\begin{itemize}
		\item let $J\subseteq\{1,\dots,k\}$ be the positions of all nullable variables
		\item for every $J'\subseteq J$ create a copy of the production where $X_j$ for $j\in J'$ are deleted, except if $J=\{1,\dots,k\}$ require $J'\neq\emptyset$
	\end{itemize}

	\textbf{Example:} $\mathcal P=\{S\rightarrow AB,A\rightarrow aAB\mid\epsilon,B\rightarrow ABBA\mid\epsilon\}$

	$S\rightarrow AB\mid A\mid B$
	$A\rightarrow aAB\mid aA\mid aB\mid a$
	$B\rightarrow ABBA\mid ABA\mid ABB\mid BBA\mid AA\mid AB\mid BA\mid BB\mid A\mid B$

\end{frame}


\begin{frame}{Step 2: Eliminate unit productions}

	\textbf{Idea}: for a unit production $A\to B$ copy rules for $B$ with head $A$, but unit productions can be composed, we need transitive closure:

	\alert{Unit pairs} $\mathcal U\subseteq V\times V$ are defined as follows:
	\begin{itemize}
		\item $(A,B)\in\mathcal U$ for every unit production $A\to B\in\mathcal P$
		\item if $(A,B)\in\mathcal U$ and $(B,C)\in\mathcal U$, then $(A,C)\in\mathcal U$
	\end{itemize}

	\textbf{To eliminate unit productions:} 
	\begin{enumerate}
		\item find all unit pairs $\mathcal U$
		\item remove all unit productions
		\item for every unit pair $(A,B)\in\mathcal U$ and production $B\to\mathcal\beta\in\mathcal P$ add the production $A\to\beta$ to $\mathcal P$
	\end{enumerate}

\end{frame}
	

\begin{frame}{Step 2: Eliminate unit productions -- an example}

	$E\rightarrow T\mid E+T$\\
	$F\rightarrow I\mid (E)$\\
	$I\rightarrow a\mid b\mid Ia\mid Ib\mid I0\mid I1$\\
	$T\rightarrow F\mid T*F$

	unit pairs:\\		
	$(E,E),(E,F),(E,I),(E,T),$\\
	$(F,F),(F,I),$\\
	$(I,I),$\\
	$(T,F),(T,I),(T,T)$

	the result:\\
	$E\rightarrow E+T\mid T*F\mid (E)\mid a\mid b\mid Ia\mid Ib\mid I0\mid I1$
	$I\rightarrow a\mid b\mid Ia\mid Ib\mid I0\mid I1$\\
	$F\rightarrow (E)\mid a\mid b\mid Ia\mid Ib\mid I0\mid I1$\\
	$T\rightarrow T*F\mid (E)\mid a\mid b\mid Ia\mid Ib\mid I0\mid I1$\\
	

\end{frame}


\begin{frame}{Step 3: Eliminate useless symbols}

	\begin{itemize}
		\item $X\in V\cup T$ is a \alert{useful} symbol (in $G$) if there exists a derivation of the form $S\Rightarrow^* \alpha X \beta \Rightarrow^* w$ for some $w \in T^*$
		\item $X$ is \alert{useless} if it is not useful
		\item $X$ is \alert{generating} if $X \Rightarrow^* w$ for some  $w\in T^*$
		\item $X$ is \alert{reachable} if $S\Rightarrow^* \alpha X \beta$ for some $\alpha,\beta\in(V\cup T)^*$
	\end{itemize}

	Observe:
	\begin{itemize}
		\item useful $\Leftrightarrow$ generating and reachable

		\item useless $\Leftrightarrow$ nongenerating or unreachable (we eliminate both)
		\item all terminals are generating
	\end{itemize}

\end{frame}


\begin{frame}{Step 3: Eliminate useless symbols -- the algorithm}

	\begin{enumerate}
		\item Find all generating symbols:
		\begin{itemize}
			\item[\textbf{basis:}] mark all terminals $a\in T$ as generating
			\item[\textbf{induct:}] for every production $A\rightarrow\beta$ where every symbol in the body $\beta$ is generating, mark the head $A$ as generating (incl. $A\rightarrow \epsilon$)
		\end{itemize}
		\item Remove all \alert{nongenerating} symbols and rules containing them
		\item Find all reachable symbols
		\begin{itemize}
			\item[\textbf{basis:}] mark $S$ as reachable
			\item[\textbf{induct:}] for every production $A\rightarrow\beta$ where the head $A$ is reachable mark every symbol in the body $\beta$ as reachable
		\end{itemize}
		\item Remove all \alert{unreachable} symbols and rules containing them		
	\end{enumerate}

	\begin{itemize}
		\item The order is important! Eliminating unreachable symbols can create new nongenerating symbols, but not vice versa.
		\item \textbf{Example:} eliminate nongenerating $B$, then unreachable $A$
		\smallskip
		\begin{center}
			\begin{minipage}{0.25\textwidth}
				$S\rightarrow AB\mid a$\\
				$A\rightarrow b$
			\end{minipage}
			\begin{minipage}{0.25\textwidth}
				$S\rightarrow a$\\
				$A\rightarrow b$
			\end{minipage}
			\begin{minipage}{0.25\textwidth}
				$S\rightarrow a$
			\end{minipage}
		\end{center}
	\end{itemize}

\end{frame}


\begin{frame}{Steps 4 \& 5: Separate terminals and break up long bodies}

	\textbf{Step 4: Separate terminals from bodies}

	For every terminal $a\in T$, introduce a new variable $V_a$.

	For every rule $A\to\beta$ with $|\beta|\geq 2$, replace every terminal $a$ by $V_a$.
	
	\bigskip

	\textbf{Step 5: Break up longer bodies}

	Replace every rule $A\to B_1\dots B_k$ with $k\geq 3$ with:
	\begin{align*}
		A&\to B_1 C_1\\
		C_1&\to B_2 C_2\\
		&\ \,\vdots\\
		C_{k-2}&\to B_{k-1}B_k
	\end{align*}
	where $C_1,\dots,C_{k-2}$ are new variables (only used for this purpose).
	
\end{frame}


\begin{frame}{Conversion to Chomsky Normal Form}

	ChNF: only useful symbols and rules \alert{$A\rightarrow BC$} or \alert{$A\rightarrow a$}
	
	\begin{theorem}
		For every context-free language $L$ such that $L\setminus \{\epsilon\}\neq \emptyset$ there exists a grammar in ChNF that generates $L\setminus \{\epsilon\}$.
	\end{theorem}
	\begin{proof}
		Take a context-free grammar $G$ for $L$. Modify it by applying steps 1, 2, 3a, 3b, 4, and 5, in order. Clearly, the result is in ChNF. After step 1 we get $G'$ such that $L(G')=L(G)\setminus \{\epsilon\}$; the remaining steps produce equivalent grammars. Steps 2-5 don't add any $\epsilon$-productions, 3-5 don't add unit productions, 3b-5 don't add nongenerating symbols, 4-5 don't add useless, etc.
	\end{proof}

	Note: If we only apply 1, 2, 3a, and 3b, we get a \alert{reduced} grammar: only useful symbols, no $\epsilon$-productions, no unit productions.

\end{frame}


\begin{frame}{Example}

	\begin{multicols}{2}
		
		$I\rightarrow a\mid b\mid Ia\mid Ib\mid I0\mid I1$\\
		$F\rightarrow I\mid (E)$
		
		$T\rightarrow F\mid T*F$\\
		$E\rightarrow T\mid E+T$
	\end{multicols}

	\bigskip
	\hrule

	\begin{multicols}{2}
		\small

		\textbf{reduce + separate}

		$I\rightarrow a\mid b\mid IA\mid IB\mid IZ\mid IU$\\
		$F\rightarrow LER\mid a\mid b\mid IA\mid IB\mid IZ\mid IU$\\
		$T\rightarrow TMF\mid LER\mid a\mid b\mid IA\mid IB\mid IZ\mid IU$\\
		$E\rightarrow EPT\mid TMF\mid LER\mid a\mid b\mid IA\mid IB\mid IZ\mid IU$\\

		\bigskip
		
		$A\rightarrow a$, $B\rightarrow b$, $Z\rightarrow 0$, $U\rightarrow 1$,\\
		$P\rightarrow +$, $M\rightarrow *$, $L\rightarrow ($, $R\rightarrow )$
		
		\bigskip\bigskip

		\textbf{break up longer bodies}

		$F\rightarrow LC_3\mid a\mid b\mid IA\mid IB\mid IZ\mid IU$\\
		$T\rightarrow TC_2\mid LC_3\mid a\mid b\mid IA\mid IB\mid IZ\mid IU$\\
		$E\rightarrow EC_1\mid TC_2\mid LC_3\mid a\mid b\mid IA\mid IB\mid IZ\mid IU$\\
		$C_1\rightarrow PT$\\
		$C_2\rightarrow MF$\\
		$C_3\rightarrow ER$\\
		$I,A,B,Z,U,P,M,L,R$ same as on the left side
	\end{multicols}

\end{frame}


\section{2.7 Pumping lemma for context-free languages}


\begin{frame}{Pumping lemma for context-free languages}

	\begin{theorem}[Pumping Lemma for Context Free Languages]
		Let $L\subseteq\Sigma^*$ be context-free. Then there exists $n\in\mathbb N$ s.t. for any $z\in L, |z|\geq n$ there are $u,v,w,x,y\in\Sigma^*$ s.t. $z=uvwxy$ and:
		
		(i) $|vwx|\leq n $ \hfill (ii) $|vx|>0$ \hfill (iii) $uv^iwx^iy\in L$ for all $i\geq 0$
	\end{theorem}

	\textbf{Proof idea:} Take a ChNF grammar for $L$. If $z\in L$ is long enough, a parse tree for $z$ must contain a path from $S$ to a leaf (terminal) of length $|V|+1$. Some nonterminal $N\in V$ repeats on this path giving two subtrees with root $N$: a larger one containing a smaller one. Replace the larger with a copy of the smaller ($i=0$) or the smaller with a copy of the larger ($i=2$).

	\textbf{What is long enough?} If $|z|>2^{k-1}$, then the depth of the tree is $k+1$. (All inner nodes not immediately above a leaf are binary!) 

\end{frame}


\begin{frame}{The proof in a picture}

	\vspace{-6pt}
	\begin{center}
		\includegraphics[height=0.92\textheight]{files/pumping-lemma-for-context-free-languages.png}
	\end{center}
	\vspace{-12pt}
	{\footnotesize Image by Jochen Burghardt, CC BY-SA 3.0, from Wikipedia}
		
	
\end{frame}


\begin{frame}{The proof}

	If $L=\emptyset$ and $L=\{\epsilon\}$ trivial, take $n=1$. Otherwise take a ChNF grammar for $L$. Set $n=2^{|V|-1}+1$ . Let $z\in L$ with $|z|\geq n$.
	
	A parse tree for $z$ contains a path from $S$ to a terminal $t$ of length at least $|V|+1$. At least two of the last $|V|+1$ nonterminals on this path must be the same. Let $A^1, A^2$ be such a pair that is closest to $t$. Let $T^1,T^2$ be the subtrees rooted at $A^1, A^2$.
		
	The path from $A^1$ to $t$ is the longest one in $T^1$ and has length at most $(k+1)$. Thus $|vwx|\leq n $.

	There are two paths from $A^1$ (ChNF!): one leads to $T^2$, the other to the rest, it must generate at least one letter (no $\epsilon$-productions). Thus $|vx|>0$.

\end{frame}


\begin{frame}{The proof cont'd}

	The word $z=uvwxy$ is derived as follows:
	\begin{itemize}
		\item $A^2\Rightarrow^* w$
		\item $A^1\Rightarrow^*vA^2x\Rightarrow^* vwx$
		\item $S\Rightarrow^*uA^1y\Rightarrow^*uvA^2xy\Rightarrow^*uvwxy$
	\end{itemize}

	For $i=0$: replace $T^1$ by $T^2$
	$$
	S\Rightarrow^*uA^2y\Rightarrow^*uwy
	$$

	For $i=2$: replace $T^2$ by a copy of $T^1$
	$$
	S\Rightarrow^*uA^1y\Rightarrow^*uvA^1xy\Rightarrow^*uvvA^2xxy\Rightarrow^*uvvwxxy
	$$

	For $i\geq 3$ repeat the above.\hfill\qedsymbol
	
\end{frame}


\begin{frame}{Application: proving nonregularity}

	TODO
	

\end{frame}


\section{2.8 The CYK algorithm}


\end{document}


\begin{example}
	The language $L=\{0^n1^n2^n\mid n\geq 1\}$ is not CFL.
	\begin{itemize}
	\item assume it were CFL
	\item we get $n$ from the Pumping Lemma
		\item choose $z=0^n1^n2^n$ $|z|\geq n $
		\item $|vwx|\leq n $
		\item we pump at most two different symbols
		\item the equality of symbols is violated -- CONTRADICTION.
	\end{itemize}
	\end{example}
	
	
	\begin{example}
	The language $L=\{0^i1^j2^k\mid 0\leq i\leq j\leq k\}$ is not CFL.
	
	\begin{itemize}
	\item assume it were CFL
	\item we get $n$ from the Pumping Lemma
		\item choose $z=0^n1^n2^n$, then $|z|\geq n $
		\item $|vwx|\leq n$
		\item we pump at most two different symbols
		\item in the case of $0$ (or $1$), pump up -- CONTRADICTION $i\leq j$ (or $j\leq n$)
		\item if $2$ (or $1$), pump down -- CONTRADICTION $j\leq n$ (or $i\leq j$)
	\end{itemize}
	\end{example}
	
	\begin{example}
	The language $L=\{0^i1^j2^i3^j\mid i,j\geq 1\}$ is not CFL.
	\begin{itemize}
	\item assume it were CFL
	\item we get $n$ from the Pumping Lemma
		\item choose $z=0^n1^n2^n3^n$, then $|z|\geq n$
		\item $|vwx|\leq n $
		\item we do pump both $0$ and 2 nor 1 and 3
		\item the equality of symbols is violated -- CONTRADICTION.
	\end{itemize}
	\end{example}
	
	\begin{example}
	The language $L=\{ww\mid w\hbox{ is in }\{0,1\}^*\}$ is not CFL.
	\begin{itemize}
	\item assume it were CFL
	\item we get $n$ from the Pumping Lemma
		\item choose $z=0^n1^n0^n1^n$, then $|z|\geq n$
		\item $|vwx|\leq n $
		\item we do not reach both positions of 0 nor both positions of 1
		\item the equality of symbols is violated -- CONTRADICTION.
	\end{itemize}
	\end{example}
	
	
	\subsection*{Generalizations of the Pumping Lemma}{\,}
	
	Pumping lemma for CFL is again only an implication!
	\begin{example}
	$L=\{a^ib^jc^kd^l| i=0 \vee j=k=l\}$ is not a CFL but it can be pumped.
	
	\begin{tabular}{l l}
	$i=0: b^jc^kd^l$ & can be pumped in any letter\\
	$i>0:a^ib^nc^nd^n$ & can be pumped in $a^*$
	\end{tabular}
	\end{example}
	
	What to do?
	\begin{itemize}
		\item generalisations (Ogden lemma and others)
		\begin{itemize}
			\item pumping marked symbols
		\end{itemize}
		\item closure properties (later).
	\end{itemize}
	
	
	
	\section*{Testing emptiness of CFLs}
	Test whether the start symbol $S$ of $G$ is generating: can be done in $O(|G|)$ time.
	\begin{figure}[H]\includegraphics[width=0.8\textwidth]{files/emptyCFL.PNG}
	\end{figure}
	\begin{itemize}
		\item Create an indexed list of all variables (left)
		\item Add links: 
		\begin{itemize}
			\item $\forall$ variable a chain of all the positions in which that variable appears (full line).
			\item $\forall$ production, a count of the number of positions holding variables whose ability to generate a terminal string has not yet been taken into account (dashed and dotted lines)
	\item Establishing $B$ is generating follow full links and decrease the count by $1$ for each occurrence.
	\item If a count reaches $0$, then we know the head variable is generating.
	\item Keep all generating variables on a stack and procceed them.
		\end{itemize}
	\end{itemize}
	
	
	\section*{Testing finiteness of CFLs}
	Let $G$ be a Chomsky normal form grammar for $L$, i.e. $L\setminus\{\lambda\}=L(G)$. Construct the following oriented graph: 
	\begin{itemize}
		\item nodes: variables in $G$
		\item edges: $\{(A,B),(A,C) \mid A\rightarrow BC\text{ is a production rule in }G\}$
	\end{itemize}
	Now $L$ is infinite, if and only if the graph contains an oriented cycle. Can be tested in $O(|G|)$.
	
	
	
	
	
	
	
	\section*{Testing membership in a CFL: the CYK algorithm}
	
	The Cocke-Younger-Kasami algorithm for testing membership in a CFL (inefficient: exponential in $|w|$ -- check all derivation trees of appropriate depth of a ChNF grammar for $L$)
	
	\begin{definition}[CYK Algorithm, time $O(n^3)$]
	\begin{itemize}
		\item The input is CNF grammar $G=(V,T,P,S)$ for language $L$ and a string $w=a_1a_2\ldots a_n\in T^*$.
		\item Create triangular table 
		
		\begin{itemize}
			\item horizontal axis is $w$
			\item $X_{ij}$ is the set of variables $A$ such that $A\gen a_ia_{i+1}\ldots a_j$.
		\end{itemize}
		\item Fill the table upwards.
	\begin{itemize}
	\item BASIS: $$X_{ii}=\{A;A\rightarrow a_i \in P\}$$
		 \item INDUCTION:
		$$X_{ij}=\{A\rightarrow BC\mid B\in X_{ik}, C\in X_{k+1,j}\text{ for some $k$ such that }i\leq k<j\}$$
	%	$X_{ij}=\{A\rightarrow BC;B\gen a_i\ldots a_k, C\gen a_{k+1}\ldots a_j\}$
	\item QUESTION: Is $S\in X_{1n}$?
	\end{itemize}
	\end{itemize}
	\end{definition}
	
	\begin{center}
	\begin{tabular}{|c c c c c c c c}
	 $X_{15}$ \\ 
	%\rowcolor{LightCyan}%\cellcolor{LightCyan}
	$X_{14}$ &  $X_{25}$  \\ 
	$X_{13}$ &  $X_{24}$ &  $X_{35}$ \\ %\\[-\normalbaselineskip]
	 $X_{12}$ &  $X_{23}$ & $X_{34}$ &  $X_{45}$ \\ 
	 $X_{11}$ & $X_{22}$ &  $X_{33}$ &$X_{44}$ &$X_{55}$ \\ \hline %\cline{1-7}
	$a_1$&$a_2$&$a_3$&$a_4$&$a_5$
	\end{tabular}
		
	\end{center}
	
	
	\begin{example}
	$G=(V,T,P,S)$, $P=\{S \to AB\mid BC, A \to BA\mid a, B\to CC\mid b, C\to AB\mid a\}$
	
	\bigskip
	
	Start filling from the bottom:
	
	\bigskip
	
	\begin{tabular}{|c c c c c c c c}
	 $\{S,A,C\}$ \\ 
	- &  $\{S,A,C\}$  \\ 
	- &  $\{B\}$ &  $\{B\}$ \\ 
	 $\{S,A\}$ &  $\{B\}$ & $\{S,C\}$ &  $\{S,A\}$ \\ 
	 $\{B\}$ & $\{A,C\}$ &  $\{A,C\}$ &$\{B\}$ &$\{A,C\}$ \\ \hline 
	$b$&$a$&$a$&$b$&$a$
	\end{tabular}
	\end{example}