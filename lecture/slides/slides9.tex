\documentclass[handout]{beamer}

\usetheme[progressbar=frametitle]{metropolis}
\metroset{block=fill}

\subtitle{NTIN071 Automata and Grammars}
\author{Jakub Bulín (KTIML MFF UK)}

\date{Spring 2025\\ 
    \vspace{1in} 
    \begin{flushleft}
        \it \footnotesize * Adapted from the Czech-lecture slides by Marta Vomlelová with gratitude. The translation, some modifications, and all errors are mine.
    \end{flushleft}
}

%% packages

\usepackage{amsmath}
\usepackage{amssymb}
\usepackage{amsthm}
\usepackage{cancel}
\usepackage{color}
\usepackage{colortbl}
\usepackage{forest}
\usepackage[utf8x]{inputenc}
\usepackage{multicol}
\usepackage{multirow}

%% colors
\definecolor{Gray}{gray}{0.9}

%% TikZ
\usepackage{tikz}
    \usetikzlibrary{
        automata,
        arrows,
        backgrounds,
        decorations.pathmorphing,
        fit,
        positioning,
        shapes,
        shapes.geometric,
        tikzmark
    } 
    \tikzset{>=stealth',shorten >=1pt,auto,node distance=2cm}
    \tikzset{initial text={}}
    \tikzset{elliptic state/.style={draw,ellipse}}

%% amsthm
\theoremstyle{plain}
    \newtheorem*{algorithm}{Algorithm}    
    \newtheorem*{observation}{Observation}
    \newtheorem*{proposition}{Proposition}

\theoremstyle{remark}
    \newtheorem*{exercise}{Exercise}
    \newtheorem*{remark}{Remark}

%% macros
\DeclareMathOperator{\RegE}{RegE}
\DeclareMathOperator{\RL}{RL}

% Just for Lecture 2
\newcommand{\x}{$\times$}
\newcommand{\nx}{\ }



\title{Lecture 9 -- Closure properties of context-free languages, Dyck languages}


\begin{document}


\frame{\titlepage}


\begin{frame}{Recap of Lecture 8}
	
    \begin{itemize}
        \item Pushdown automata accept exactly context-free languages (constructions: CFG to PDA and PDA to CFG)
        \item A deterministic pushdown automaton (DPDA)
        \item DPDA recognize a proper subclass of context-free languages,\\ accepts by empty stack iff prefix-free and accepts by final state\\
        (Deterministic PDA + acceptance by empty stack does not even cover regular languages!)
        \item Deterministic PDA have unambiguous grammars
        \item The landscape of languages
        \item Converting between representations of context-free languages
        \item Undecidable problems about context-free languages (preview)
	\end{itemize}

\end{frame}


\section*{2.12 Closure properties of context-free languages}


\begin{frame}{Closed under union, concatenation, iteration, reverse}

        

\end{frame}


\begin{frame}{Not closed under intesection}

    

\end{frame}


\begin{frame}{Intersection of a context-free and a regular language}
    

\end{frame}


\begin{frame}{An application}


\end{frame}


\begin{frame}{Difference and complement}

    

\end{frame}


\section*{Substitution and homomorphism}



\section*{2.13 Dyck languages}


\begin{frame}{Dyck languages}

    
\end{frame}


\begin{frame}{A characterization of context-free languages}


\end{frame}


\begin{frame}{The proof}

    

\end{frame}


\end{document}