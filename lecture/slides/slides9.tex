\documentclass[handout]{beamer}

\usepackage{a4wide}
\usepackage{amsmath}
\usepackage{amssymb}
\usepackage{amsthm}
\usepackage[czech]{babel}
\usepackage{bookmark}
\usepackage{enumerate}
\usepackage[T1]{fontenc}
\usepackage{hyperref}
\usepackage[utf8]{inputenc}
\usepackage{lmodern}
\usepackage{multicol}
\usepackage{tikz}
    \usetikzlibrary{automata, arrows, positioning}


\theoremstyle{definition}
    \newtheorem{problem}{Příklad}




\title{Lecture 9 -- Closure properties of context-free languages, Dyck languages}


\begin{document}


\frame{\titlepage}


\begin{frame}{Recap of Lecture 8}
	
    \begin{itemize}
        \item Pushdown automata accept exactly context-free languages (constructions: CFG to PDA and PDA to CFG)
        \item A deterministic pushdown automaton (DPDA)
        \item DPDA recognize a proper subclass of context-free languages,\\ accepts by empty stack iff prefix-free and accepts by final state\\
        (Deterministic PDA + acceptance by empty stack does not even cover regular languages!)
        \item Deterministic PDA have unambiguous grammars
        \item The landscape of languages
        \item Converting between representations of context-free languages
        \item Undecidable problems about context-free languages (preview)
	\end{itemize}

\end{frame}


\section*{2.12 Closure properties of context-free languages}


\begin{frame}{Closed under union, concatenation, iteration, reverse}

        

\end{frame}


\begin{frame}{Not closed under intesection}

    

\end{frame}


\begin{frame}{Intersection of a context-free and a regular language}
    

\end{frame}


\begin{frame}{An application}


\end{frame}


\begin{frame}{Difference and complement}

    

\end{frame}


\section*{Substitution and homomorphism}



\section*{2.13 Dyck languages}


\begin{frame}{Dyck languages}

    
\end{frame}


\begin{frame}{A characterization of context-free languages}


\end{frame}


\begin{frame}{The proof}

    

\end{frame}


\end{document}