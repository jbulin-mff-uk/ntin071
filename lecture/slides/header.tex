\usetheme[progressbar=frametitle]{metropolis}
\metroset{block=fill}

\subtitle{NTIN071 Automata and Grammars}
\author{Jakub Bulín (KTIML MFF UK)}

\date{Spring 2024\\ 
    \vspace{1in} 
    \begin{flushleft}
        \it \footnotesize * Adapted from the Czech-lecture slides by Marta Vomlelová with gratitude. The translation, some modifications, and all errors are mine.
    \end{flushleft}
}

%% packages

\usepackage{amsmath,amssymb,amsthm}
\usepackage{cancel}
\usepackage{color}
\usepackage{colortbl}
\usepackage{forest}
\usepackage[utf8x]{inputenc}
\usepackage{multicol}
\usepackage{multirow}

%% colors
\definecolor{Gray}{gray}{0.9}

%% TikZ
\usepackage{tikz}
    \usetikzlibrary{
        automata,
        arrows,
        backgrounds,
        decorations.pathmorphing,
        fit,
        positioning,
        shapes,
        shapes.geometric,
        tikzmark
    } 
    \tikzset{>=stealth',shorten >=1pt,auto,node distance=2cm}
    \tikzset{initial text={}}
    \tikzset{elliptic state/.style={draw,ellipse}}

%% amsthm
\theoremstyle{plain}
    \newtheorem*{algorithm}{Algorithm}    
    \newtheorem*{observation}{Observation}
    \newtheorem*{proposition}{Proposition}

\theoremstyle{remark}
    \newtheorem*{exercise}{Exercise}
    \newtheorem*{remark}{Remark}

%% macros
\DeclareMathOperator{\RegE}{RegE}
\DeclareMathOperator{\RL}{RL}

% Just for Lecture 2
\newcommand{\x}{$\times$}
\newcommand{\nx}{\ }
