\usetheme[progressbar=frametitle]{metropolis}
\metroset{block=fill}

\subtitle{NTIN071 Automata and Grammars}
\author{Jakub Bulín (KTIML MFF UK)}

\date{Spring 2024\\ 
    \vspace{1in} 
    \begin{flushleft}
        \it \footnotesize * Adapted from the Czech-lecture slides by Marta Vomlelová with gratitude. The translation, some modifications, and all errors are mine.
    \end{flushleft}
}

%% packages

\usepackage{amsmath,amssymb,amsthm}
\usepackage{cancel}
\usepackage{color}
\usepackage{colortbl}
\usepackage{forest}
\usepackage[utf8x]{inputenc}
\usepackage{multicol}

%% colors
\definecolor{Gray}{gray}{0.9}


%% TikZ
\usepackage{tikz}
    \usetikzlibrary{shapes,shapes.geometric,arrows,fit,automata}
    \usetikzlibrary{positioning}
    \usetikzlibrary{decorations.pathmorphing}
    \tikzset{>=stealth',shorten >=1pt,auto,node distance=2cm}
    \tikzset{initial text={}}
    \tikzset{elliptic state/.style={draw,ellipse}}


%% amsthm
\theoremstyle{plain}
    \newtheorem*{algorithm}{Algorithm}
    \newtheorem*{observation}{Observation}

\theoremstyle{remark}
    \newtheorem*{exercise}{Exercise}
    \newtheorem*{remark}{Remark}

%% macros


\DeclareMathOperator{\RegE}{RegE}
\DeclareMathOperator{\RL}{RL}

% Just for Lecture 2
\newcommand{\x}{$\times$}
\newcommand{\nx}{\ }


%% Fix thm title spacing
% % map t block to oldblock
% \let\oldblock\block
% \let\endoldblock\endblock

% % change block by adding smallskip
% \renewenvironment{block}[1]
%     {\begin{oldblock}{#1}
%         \smallskip
%     }
%     { 
%     \end{oldblock}
%     }



% \usepackage{bookmark}
% \usepackage{booktabs}
% \usepackage{cancel}
% \usepackage[czech]{babel}
% \usepackage{enumerate}
% \usepackage[T1]{fontenc}
% \usepackage{forest}
% \usepackage{lmodern}
% \usepackage{tikz}
%     \usetikzlibrary{arrows.meta}
% \usepackage[utf8x]{inputenc}
% \usepackage{xfrac}

% %% macros and definitions
% %\DeclareMathOperator{\Aut}{Aut}

% \newcommand{\A}{\textbf{A}}
