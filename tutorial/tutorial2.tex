\documentclass[a4paper,12pt]{amsart}

\usetheme[progressbar=frametitle]{metropolis}
\metroset{block=fill}

\subtitle{NTIN071 Automata and Grammars}
\author{Jakub Bulín (KTIML MFF UK)}

\date{Spring 2025\\ 
    \vspace{1in} 
    \begin{flushleft}
        \it \footnotesize * Adapted from the Czech-lecture slides by Marta Vomlelová with gratitude. The translation, some modifications, and all errors are mine.
    \end{flushleft}
}

%% packages

\usepackage{amsmath}
\usepackage{amssymb}
\usepackage{amsthm}
\usepackage{cancel}
\usepackage{color}
\usepackage{colortbl}
\usepackage{forest}
\usepackage[utf8x]{inputenc}
\usepackage{multicol}
\usepackage{multirow}

%% colors
\definecolor{Gray}{gray}{0.9}

%% TikZ
\usepackage{tikz}
    \usetikzlibrary{
        automata,
        arrows,
        backgrounds,
        decorations.pathmorphing,
        fit,
        positioning,
        shapes,
        shapes.geometric,
        tikzmark
    } 
    \tikzset{>=stealth',shorten >=1pt,auto,node distance=2cm}
    \tikzset{initial text={}}
    \tikzset{elliptic state/.style={draw,ellipse}}

%% amsthm
\theoremstyle{plain}
    \newtheorem*{algorithm}{Algorithm}    
    \newtheorem*{observation}{Observation}
    \newtheorem*{proposition}{Proposition}

\theoremstyle{remark}
    \newtheorem*{exercise}{Exercise}
    \newtheorem*{remark}{Remark}

%% macros
\DeclareMathOperator{\RegE}{RegE}
\DeclareMathOperator{\RL}{RL}

% Just for Lecture 2
\newcommand{\x}{$\times$}
\newcommand{\nx}{\ }


\begin{document}

\thispagestyle{empty}

\section*{NTIN071 A\&G: Tutorial 2 -- Pumping Lemma, Myhill--Nerode theorem}

\medskip

\subsection*{Teaching goals:} The student is able to

    \begin{itemize}\setlength{\itemsep}{0pt}
        \item state and prove the Pumping Lemma
        \item apply the Pumping Lemma to prove nonregularity of a given language
        \item state and prove the Myhill-Nerode theorem
        \item apply the Myhill-Nerode theorem to prove regularity, to construct a DFA
        \item apply the Myhill-Nerode theorem to prove nonregularity
    \end{itemize}

\section*{In-class problems}


\medskip\begin{problem}[Pumping Lemma: statement]
    
    \begin{enumerate}[(a)]\setlength\itemsep{6pt}
        \item Formulate the Pumping Lemma for regular languages (without consulting your notes).
        \item How is the number $n$ form its statement related to a recognizing automaton?
        \item Prove it (without consulting your notes).
        \item Demonstrate pumping on $L=\{w\in\{a,b\}^* \,\mid\,\text{$w$ contains $abba$ as a subword}\}$.
    \end{enumerate}

\end{problem}
    
    
\medskip\begin{problem}[Pumping Lemma: application]
    
    Use the Pumping Lemma to prove that the following languages are not regular. (The alphabet is $\Sigma=\{a,b\}$.)
    
    \medskip
      
    \begin{enumerate}[(a)]\setlength\itemsep{6pt}        
        \item $L=\{a^ib^j\ \mid\ i\geq j\}$       
        \item $L=\{a^{i^2}\ \mid\ i\geq 0\}$        
        \item $L=\{a^ib^{i+j}a^j\ \mid\ i,j\geq 0\}$
        \item $L=\{ww^R\ \mid \ w\in\Sigma^*\}$, where $w^R$ is $w$ reversed
    \end{enumerate}
      
\end{problem}


\medskip\begin{problem}[Myhill--Nerode theorem: statement]
    
    \begin{enumerate}[(a)]
        \item Formulate the Myhill--Nerode theorem and recall the idea of its proof (without consulting your notes).
        \item Show that if we forget any of the conditions on the equivalence $\sim$, the resulting statement is not true.
    \end{enumerate}
    

\end{problem}


\medskip\begin{problem}[Myhill--Nerode theorem: application]

    Prove or disprove using the Myhill--Nerode theorem that the following languages are regular.
    \begin{enumerate}[(a)]\setlength\itemsep{6pt}
        \item $L=\{aa, ab, ba\}$        
        \item $L=\{a^ib^j\ \mid\ i\geq j\}$        
        \item $L=\{a^{i^2}\ \mid\ i\geq 0\}$ 
        \item $L=\{ww^R\ \mid \ w\in\Sigma^*\}$, where $w^R$ is $w$ reversed
        \item $L=\{a^ib^{i+j}a^j\ \mid\ i,j\geq 0\}$        
    \end{enumerate}

\end{problem}


\section*{Extra Practice and Thinking}


\medskip\begin{problem}
    
    Use the Pumping Lemma to prove that the following languages are not regular. (The alphabet is $\Sigma=\{a,b\}$.)
          
    \begin{enumerate}[(a)]\setlength\itemsep{6pt}
        \item $L=\{a^ib^j\ \mid\ i\leq j\}$        
        \item $L=\{a^{2^i}\ \mid\ i\geq 0\}$
        \item $L=\{ww\ \mid \ w\in\Sigma^*\}$
    \end{enumerate}
    
\end{problem}


\medskip\begin{problem}

    Prove or disprove using the Myhill--Nerode theorem that the following languages are regular.
    \begin{enumerate}[(a)]\setlength\itemsep{6pt}
        \item $L=\{a^ib^j\ \mid\ i\leq j\}$
        \item $L_k=\{a^ib^j\ \mid\ i\leq j\leq k\}$ for a fixed $k\in\mathbb N$
        \item $L=\{a^{2^i}\ \mid\ i\geq 0\}$
        \item $L=\{ww\ \mid \ w\in\Sigma^*\}$
    \end{enumerate}

\end{problem}


\medskip\begin{problem}[Pumping Lemma: generalization]
    
    \begin{enumerate}[(a)]\setlength\itemsep{6pt}
        \item Can we change the condition $|uv|\leq n$ with $|vw|\leq n$, that is, \emph{iterate near the end}? Prove or disprove.
        \item Can we iterate near a chosen position in the word? How to formulate (and prove) such a generalization?
    \end{enumerate}
    
\end{problem}


\medskip\begin{problem}[Equivalences on words]

    Give an example of an equivalence relation $\sim$ on $\Sigma^*$ which:

    \medskip
    
    \begin{enumerate}[(a)]\setlength\itemsep{12pt}
        \item is a right and a left congruence
        \item is a right but not a left congruence
        \item is of finite index
    \end{enumerate}

\end{problem}


\end{document}