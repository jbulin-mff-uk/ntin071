\documentclass[a4paper,12pt]{amsart}

\usetheme[progressbar=frametitle]{metropolis}
\metroset{block=fill}

\subtitle{NTIN071 Automata and Grammars}
\author{Jakub Bulín (KTIML MFF UK)}

\date{Spring 2025\\ 
    \vspace{1in} 
    \begin{flushleft}
        \it \footnotesize * Adapted from the Czech-lecture slides by Marta Vomlelová with gratitude. The translation, some modifications, and all errors are mine.
    \end{flushleft}
}

%% packages

\usepackage{amsmath}
\usepackage{amssymb}
\usepackage{amsthm}
\usepackage{cancel}
\usepackage{color}
\usepackage{colortbl}
\usepackage{forest}
\usepackage[utf8x]{inputenc}
\usepackage{multicol}
\usepackage{multirow}

%% colors
\definecolor{Gray}{gray}{0.9}

%% TikZ
\usepackage{tikz}
    \usetikzlibrary{
        automata,
        arrows,
        backgrounds,
        decorations.pathmorphing,
        fit,
        positioning,
        shapes,
        shapes.geometric,
        tikzmark
    } 
    \tikzset{>=stealth',shorten >=1pt,auto,node distance=2cm}
    \tikzset{initial text={}}
    \tikzset{elliptic state/.style={draw,ellipse}}

%% amsthm
\theoremstyle{plain}
    \newtheorem*{algorithm}{Algorithm}    
    \newtheorem*{observation}{Observation}
    \newtheorem*{proposition}{Proposition}

\theoremstyle{remark}
    \newtheorem*{exercise}{Exercise}
    \newtheorem*{remark}{Remark}

%% macros
\DeclareMathOperator{\RegE}{RegE}
\DeclareMathOperator{\RL}{RL}

% Just for Lecture 2
\newcommand{\x}{$\times$}
\newcommand{\nx}{\ }


\begin{document}

\thispagestyle{empty}

\section*{NTIN071 A\&G: Tutorial 4 -- Closure properties of regular languages}

\medskip

\subsection*{Teaching goals:} The student is able to

    \begin{itemize}\setlength{\itemsep}{0pt}
        \item formally describe a construction of an automaton based on other automata
        \item decide whether regular languages are closed under various set and string operations, including more complex ones, and prove or disprove it
    \end{itemize}


\section*{In-class problems}


\medskip\begin{problem}[Closure under set and string operations] 
    
    Given DFAs $A,B$, construct an automaton $C$ recognizing the given language. (Give a formal description of the automaton.)
    
    \begin{multicols}{3}

        \begin{enumerate}[(a)]\setlength\itemsep{6pt}
            \item $L(A)-L(B)$
            \item $L(A).L(B)$
            \item $L(A)^+$
            \item $L(A)^*$
            \item $L(A)^R$
        \end{enumerate}

        \begin{tabular}{ r | c c }
            A & a & b \\ \hline
            $\to$ 0 & 1 & 2 \\  
            $\ast$ 1 & 3 & 0 \\
            2 & 4 & 5 \\
            3 & 0 & 2 \\
            4 & 2 & 5 \\
            5 & 0 & 3
        \end{tabular}    
        
        \begin{tabular}{ r | c c }
            B & a & b \\ \hline
            $\to$ 0 & 0 & 5 \\  
            $\ast$ 1 & 1 & 3 \\
            2 & 2 & 5 \\
            3 & 3 & 2 \\
            $\ast$ 4 & 6 & 1 \\
            5 & 5 & 1 \\
            $\ast$ 6 & 4 & 2
        \end{tabular}

    \end{multicols}
    
\end{problem}


\medskip\begin{problem}[Delete]
    
    Let $L$ be a regular language over the alphabet $\Sigma=\{a,b\}$. Desribe the following languages in set notation. Decide if they are (necessarily) also regular, prove or disprove.    
    The language of all words obtained from words of the language $L$ by\dots

    \medskip

    \begin{enumerate}[(a)]\setlength\itemsep{6pt} 
        \item \dots deleting all occurrences of the letter $a$. 
        \item \dots deleting the initial letter and writing this letter at the end of the word. 
        \item \dots deleting the longest contiguous sequence of $a$'s from the beginning of the word.
    \end{enumerate}

\end{problem}


\section*{Extra Practice and Thinking}


\medskip\medskip\begin{problem}[Prefixes]
    Are regular languages closed under the following operations? Prove or disprove. (In the following, $L$ is a regular language over an alphabet $\Sigma$.)
    \begin{enumerate}[(a)]
        \item $\mathrm{init}(L)=\{w\in\Sigma^*\mid \text{there is }u\in\Sigma^*\text{ such that }wu\in L\}$
        \item $\min(L)=\{w\in L\mid \text{there is no }u\in L,v\in\Sigma^+\text{ such that }w=uv\}$
        \item $\max(L)=\{w\in L\mid \text{there is no }u\in\Sigma^+\text{ such that }wu\in L\}$
    \end{enumerate}
\end{problem}  




\medskip\begin{problem}[Shift] 
    Given a regular language $L$ over an alphabet $\Sigma$, define the language $L'$ as follows. Is the language $L'$ necessarily regular? 
    $$
    L'=\{uv\mid u,v\in\Sigma^*,vu\in L\}
    $$ 
\end{problem}  

\medskip\begin{problem}[Cut] 
    Consider two regular languages $L$ and $M$ over an alphabet $\Sigma$, and define the language $K$ as follows. Is the language $K$ necessarily regular? 
    $$  
    K=\{uw\mid u,w\in\Sigma^*, (\exists v\in M)\, uvw\in L\}  
    $$
\end{problem}  


\medskip\begin{problem}[Switch final and nonfinal states]
    
    If we switch accepting and nonaccepting states in a given NFA, will the language recognized by the resulting automaton be the complement of the language recognized by the original NFA? Justify your answer.

\end{problem}


\medskip\begin{problem}[Iterations of unary languages]
    
    Show that for any language $L$ over the alphabet $\Sigma=\{a\}$, the language $L^*$ is regular. 

    % Hint: Use Bézout's identity.

\end{problem}


\end{document}