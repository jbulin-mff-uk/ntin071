\documentclass[a4paper,12pt]{amsart}

\usetheme[progressbar=frametitle]{metropolis}
\metroset{block=fill}

\subtitle{NTIN071 Automata and Grammars}
\author{Jakub Bulín (KTIML MFF UK)}

\date{Spring 2025\\ 
    \vspace{1in} 
    \begin{flushleft}
        \it \footnotesize * Adapted from the Czech-lecture slides by Marta Vomlelová with gratitude. The translation, some modifications, and all errors are mine.
    \end{flushleft}
}

%% packages

\usepackage{amsmath}
\usepackage{amssymb}
\usepackage{amsthm}
\usepackage{cancel}
\usepackage{color}
\usepackage{colortbl}
\usepackage{forest}
\usepackage[utf8x]{inputenc}
\usepackage{multicol}
\usepackage{multirow}

%% colors
\definecolor{Gray}{gray}{0.9}

%% TikZ
\usepackage{tikz}
    \usetikzlibrary{
        automata,
        arrows,
        backgrounds,
        decorations.pathmorphing,
        fit,
        positioning,
        shapes,
        shapes.geometric,
        tikzmark
    } 
    \tikzset{>=stealth',shorten >=1pt,auto,node distance=2cm}
    \tikzset{initial text={}}
    \tikzset{elliptic state/.style={draw,ellipse}}

%% amsthm
\theoremstyle{plain}
    \newtheorem*{algorithm}{Algorithm}    
    \newtheorem*{observation}{Observation}
    \newtheorem*{proposition}{Proposition}

\theoremstyle{remark}
    \newtheorem*{exercise}{Exercise}
    \newtheorem*{remark}{Remark}

%% macros
\DeclareMathOperator{\RegE}{RegE}
\DeclareMathOperator{\RL}{RL}

% Just for Lecture 2
\newcommand{\x}{$\times$}
\newcommand{\nx}{\ }


\usepackage{mdframed}

\begin{document}

% \thispagestyle{empty}



\section*{NTIN071 A\&G: Tutorial 14 -- TODO Intro to complexity}

% after Lecture 13
% spring 2024

\medskip

% \noindent\emph{Solve 1abc only for A\&B, 2, 3a (the rest is for practice).}

\medskip






% \bigskip\begin{problem}
%     Determine the correct inclusion relationship between the following pairs of classes. That is, fill in one of the relation symbols $\subsetneq$, $\subseteq$, $=$, $\supseteq$, $\supsetneq$, or $?$ (if we cannot say based on our knowledge). Justify your answers.
    
%     \renewcommand{\arraystretch}{2.5}
%     \begin{tabular}{p{0.5cm}  p{3cm} p{2.5cm} | p{0.5cm} p{3cm} p{2.5cm}}
%     (a) & $\SPACE(n\log n)$ & $\TIME(2^{n^2})$ & (f) & $\SPACE(n\log n)$ & $\TIME(2^{n\log n})$\\
%     (b) & $\TIME(2^{n^2})$ & $\TIME(2^{n\log n})$ & (g) & $\TIME(2^{n^2})$ & $\NSPACE(\log^2 n)$\\
%     (c) & $\TIME(2^{n\log n})$ & $\NSPACE(\log^2 n)$ & (h) & $\TIME(2^{n\log n})$ & $\NTIME(n)$\\
%     (d) & $\NSPACE(\log^2 n)$ & $\NTIME(n)$ & (i) & $\NSPACE(\log^2 n)$ & $\SPACE(n\log n)$ \\
%     (e) & $\NTIME(n)$ & $\SPACE(n\log n)$ & (j) & $\NTIME(n)$ & $\TIME(2^{n^2})$
%     \end{tabular}
%     \end{problem}
    
    
%     \bigskip\begin{problem}
%     Determine the correct inclusion relationship between the following pairs of classes. That is, fill in one of the relation symbols $\subsetneq$, $\subseteq$, $=$, $\supseteq$, $\supsetneq$, or $?$ (if we cannot say based on our knowledge). Justify your answers.
    
%     \renewcommand{\arraystretch}{2.5}
%     \begin{tabular}{p{0.5cm}  p{3cm} p{2.5cm} | p{0.5cm} p{3cm} p{2.5cm}}
%     (a) & $\SPACE(n)$ & $\TIME(2^n)$ & (f) & $\SPACE(n)$ & $\TIME(2^{n\log n})$\\
%     (b) & $\TIME(2^n)$ & $\TIME(2^{n\log n})$ & (g) & $\TIME(2^n)$ & $\NSPACE(\log^3 n)$\\
%     (c) & $\TIME(2^{n\log n})$ & $\NSPACE(\log^3 n)$ & (h) & $\TIME(2^{n\log n})$ & $\NTIME(2^n)$\\
%     (d) & $\NSPACE(\log^3 n)$ & $\NTIME(2^n)$ & (i) & $\NSPACE(\log^3 n)$ & $\SPACE(n)$ \\
%     (e) & $\NTIME(2^n)$ & $\SPACE(n)$ & (j) & $\NTIME(2^n)$ & $\TIME(2^n)$
%     \end{tabular}
%     \end{problem}
    
    
    \bigskip\begin{problem}
    Show that the class P is closed under union, intersection, and complement.
    \end{problem}
    
    
    \bigskip\begin{problem}
    Show that the class NP is closed under union and intersection.
    \end{problem}
    
    \bigskip\begin{problem}
    Show that the class P is closed under Kleene star. That is, if $A\in P$, then
    $$
    A^* = \{w_1\dots w_k\mid k\geq 0, w_i\in A\text { for all }i\leq k\} \in \text{P}.
    $$
    (Hint: Design a dynamic algorithm filling a table $T$ where $T[i,j]=1$ if and only if $w_i\dots w_j\in A^*$.) 
    \end{problem}
    
    \bigskip\begin{problem}
    Show that the class NP is closed under Kleene star.
    \end{problem}
    
    \bigskip\hrule\hrule
    
    
    % Determine the correct inclusion relationship between the following pairs of classes. That is, fill in one of the relation symbols $\subsetneq$, $\subseteq$, $=$, $\supseteq$, $\supsetneq$, or $?$ (if we cannot say based on our knowledge). Justify your answers.
    
    % \renewcommand{\arraystretch}{2.5}
    % \begin{tabular}{p{0.5cm}  p{3cm} p{2.5cm} | p{0.5cm} p{3cm} p{2.5cm}}
    % (a) & $\SPACE(n)$ & $\TIME(2^{\log^3 n})$ & (f) & $\SPACE(n)$ & $\NSPACE(\log^2 n)$\\
    % (b) & $\TIME(2^{\log^3 n})$ & $\NSPACE(\log^2 n)$ & (g) & $\TIME(2^{\log^3 n})$ & $\NTIME(2^{\log^3 n})$\\
    % (c) & $\NSPACE(\log^2 n)$ & $\NTIME(2^{\log^3 n})$ & (h) & $\NSPACE(\log^2 n)$ & $\NTIME(2^{n\log n})$\\
    % (d) & $\NTIME(2^{\log^3 n})$ & $\NTIME(2^{n\log n})$ & (i) & $\NTIME(2^{\log^3 n})$ & $\SPACE(n)$ \\
    % (e) & $\NTIME(2^{n\log n})$ & $\SPACE(n)$ & (j) & $\NTIME(2^{n\log n})$ & $\TIME(2^{\log^3 n})$
    % \end{tabular}



    \medskip\begin{problem}
        Show that the problems \textsc{clique}, \textsc{independent-set}, and \textsc{vertex-cover} defined below are polynomially inter-reducible. 
        
        \bigskip
        \begin{quote}
        \begin{mdframed}
        \textsc{clique}
        \medskip\hrule\medskip
        \begin{itemize}
            \item[\textsc{In:}] A graph $G=(V,E)$ and an integer $k\geq 0$.
            \item[\textsc{Q:}] Does $G$ contain (as a subgraph) the complete graph (clique) on at least $k$ vertices?
        \end{itemize}
        \end{mdframed}
        \end{quote}
        
        \bigskip
        \begin{quote}
        \begin{mdframed}
        \textsc{independent-set}
        \medskip\hrule\medskip
        \begin{itemize}
            \item[\textsc{In:}] A graph $G=(V,E)$ and an integer $k\geq 0$.
            \item[\textsc{Q:}] Does $G$ contain an independent set of size at least $k$, i.e., $S\subseteq V$, $|S|\geq k$ with no edge connecting a pair of vertices from $S$?
        \end{itemize}
        \end{mdframed}
        \end{quote}
        
        \bigskip
        \begin{quote}
        \begin{mdframed}
        \textsc{vertex-cover}
        \medskip\hrule\medskip
        \begin{itemize}
            \item[\textsc{In:}] A graph $G=(V,E)$ and an integer $k\geq 0$.
            \item[\textsc{Q:}] Does $G$ have a vertex cover of size at most $k$, i.e., $S\subseteq V$, $|S|\leq k$ containing at least one vertex from every edge?
        \end{itemize}
        \end{mdframed}
        \end{quote}
        
        \end{problem}
        
        
        \medskip\begin{problem}
        Show that the problem \textsc{vertex-cover} is polynomially reducible to the problem \textsc{dominating-set} defined below.
        
        \bigskip
        \begin{quote}
        \begin{mdframed}
        \textsc{dominating-set}
        \medskip\hrule\medskip
        \begin{itemize}
            \item[\textsc{In:}] A graph $G=(V,E)$ and an integer $k\geq 0$.
            \item[\textsc{Q:}] Does $G$ contain a set of vertices $S\subseteq V$ of size at most $k$ such that every $v\in V\setminus S$ has a neighbor in $S$?
        \end{itemize}
        \end{mdframed}
        \end{quote}
        
        
        \end{problem}
        
        
        \medskip\begin{problem}
        It is well known that the problem \textsc{hamiltonian-cycle} defined below is an NP-complete problem. Use this fact to show that the problems \textsc{oriented-hamiltonian-cycle}, \textsc{$(s,t)$-hamiltonian-path}, and \textsc{hamiltonian-path} defined further below are NP-complete as well.
        
        \bigskip
        \begin{quote}
        \begin{mdframed}
        \textsc{hamiltonian-cycle}
        \medskip\hrule\medskip
        \begin{itemize}
            \item[\textsc{In:}] An (unoriented) graph $G=(V,E)$.
            \item[\textsc{Q:}] Does $G$ contain a Hamiltonian cycle, i.e., a cycle containing every vertex?
        \end{itemize}
        \end{mdframed}
        \end{quote}
        
        
        \bigskip
        \begin{quote}
        \begin{mdframed}
        \textsc{oriented-hamiltonian-cycle}
        \medskip\hrule\medskip
        \begin{itemize}
            \item[\textsc{In:}] An oriented graph $G=(V,E)$.
            \item[\textsc{Q:}] Does $G$ contain an oriented Hamiltonian cycle, i.e., an oriented cycle containing every vertex?
        \end{itemize}
        \end{mdframed}
        \end{quote}
        
        \bigskip
        \begin{quote}
        \begin{mdframed}
        \textsc{$(s,t)$-hamiltonian-path}
        \medskip\hrule\medskip
        \begin{itemize}
            \item[\textsc{In:}] An (unoriented) graph $G=(V,E)$ and a pair of vertices $s,t\in V$.
            \item[\textsc{Q:}] Does $G$ contain a Hamiltonian path from $s$ to $t$, i.e., a path that starts in $s$, ends in $t$, and visits every vertex exactly once?
        \end{itemize}
        \end{mdframed}
        \end{quote}
        
        \bigskip
        \begin{quote}
        \begin{mdframed}
        \textsc{hamiltonian-path}
        \medskip\hrule\medskip
        \begin{itemize}
            \item[\textsc{In:}] An (unoriented) graph $G=(V,E)$.
            \item[\textsc{Q:}] Does $G$ contain a Hamiltonian path, i.e., a path that visits every vertex exactly once?
        \end{itemize}
        \end{mdframed}
        \end{quote}
        
        \end{problem}
        
        
        \medskip\begin{problem}
        Show that the problem \textsc{hamiltonian-cycle} is polynomially reducible to the problem \textsc{sat} defined below.
        
        \bigskip
        \begin{quote}
        \begin{mdframed}
        \textsc{sat}
        \medskip\hrule\medskip
        \begin{itemize}
            \item[\textsc{In:}] A propositional formula $\varphi$ in conjunctive normal form (CNF).
            \item[\textsc{Q:}] Is $\varphi$ satisfiable?
        \end{itemize}
        \end{mdframed}
        \end{quote}
        
        \end{problem}
        
        
        \medskip\begin{problem}
        Show that the problem \textsc{hamiltonian-cycle} is polynomially reducible to the problem \textsc{traveling-salesperson} defined below.
        
        \bigskip
        \begin{quote}
        \begin{mdframed}
        \textsc{traveling-salesperson}
        \medskip\hrule\medskip
        \begin{itemize}
            \item[\textsc{In:}] A list of cities $C=\{c_1,\dots,c_n\}$, distances $d(c_i,c_j)\in\mathbb N$ between each pair of cities, and $D\in\mathbb N$.
            \item[\textsc{Q:}] Is there a route of length at most $D$ that visits every city exactly once and returns to the origin city?
        \end{itemize}
        \end{mdframed}
        \end{quote}
        
        \end{problem}
        
        
        \medskip\begin{problem}
        Show that the problems \textsc{integer-programming} and \textsc{binary-integer-programming} defined below are NP-hard. You can use a reduction from one of the following problems: \textsc{sat}, \textsc{clique}, \textsc{independent-set}, or \textsc{vertex-cover}.
        
        Moreover, show that the problem \textsc{binary-integer-programming} is in NP (and thus it is NP-complete). The problem \textsc{integer-programming} is in NP as well but it is not that easy to prove. Why? 
        
        \bigskip
        \begin{quote}
        \begin{mdframed}
        \textsc{integer-programming}
        \medskip\hrule\medskip
        \begin{itemize}
            \item[\textsc{In:}] A $k\times n$ integer matrix $A$ and an integer vector $b$ of length $k$.
            \item[\textsc{Q:}] Is there an integer vector $x$ of length $n$ such that $Ax\geq b$?
        \end{itemize}
        \end{mdframed}
        \end{quote}
        
        \bigskip
        \begin{quote}
        \begin{mdframed}
        \textsc{binary-integer-programming}
        \medskip\hrule\medskip
        \begin{itemize}
            \item[\textsc{In:}] A $k\times n$ integer matrix $A$ and an integer vector $b$ of length $k$.
            \item[\textsc{Q:}] Is there a vector $x\in\{0,1\}^n$ such that $Ax\geq b$?
        \end{itemize}
        \end{mdframed}
        \end{quote}
        
        \end{problem}
        
        
        \medskip\begin{problem}
        Show that the problem \textsc{graph-coloring} defined below is NP-complete.
        
        \bigskip
        \begin{quote}
        \begin{mdframed}
        \textsc{graph-coloring}
        \medskip\hrule\medskip
        \begin{itemize}
            \item[\textsc{In:}] A graph $G=(V,E)$ and $k\in\mathbb N$.
            \item[\textsc{Q:}] Can we color vertices of $G$ with at most $k$ colors so that there are no monochromatic edges?
        \end{itemize}
        \end{mdframed}
        \end{quote}
        
        \end{problem}
        
        
        Show that the problem \textsc{cover-oriented-cycles} defined below is NP-complete. (Hint: Use a reduction from \textsc{vertex-cover}. Do not forget to show that \textsc{cover-oriented-cycles} is in NP.)
        
        \bigskip
        \begin{quote}
        \begin{mdframed}
        \textsc{cover-oriented-cycles}
        \medskip\hrule\medskip
        \begin{itemize}
            \item[\textsc{In:}] An oriented graph $G=(V,E)$ and an integer $k\geq 0$.
            \item[\textsc{Q:}] Is there a set of vertices $S\subseteq V$, $|S|\leq k$, containing at least one vertex from every oriented cycle in $G$?
        \end{itemize}
        \end{mdframed}
        \end{quote}
        

        
        \bigskip

        Consider the \textsc{partition-problem} defined below.
        \begin{enumerate}[(a)]
            \item Show that the \textsc{partition-problem} is polynomially reducible to the \textsc{knapsack} problem defined further below.
            \item Show that the \textsc{partition-problem} is polynomially reducible to the \textsc{scheduling} problem defined even further below.
        \end{enumerate}
        
        \bigskip
        \begin{quote}
        \begin{mdframed}
        \textsc{partition-problem}
        \medskip\hrule\medskip
        \begin{itemize}
            \item[\textsc{In:}] A (finite) set $A$ and a value $s(a)\in\mathbb N$ associated with each $a\in A$.
            \item[\textsc{Q:}] Is it possible to partition $A$ into two parts of the same value? More precisely, is there $A'\subseteq A$ such that $\sum_{a\in A'} s(a)=\sum_{a\in A\setminus A'} s(a)$?
        \end{itemize}
        \end{mdframed}
        \end{quote}
        
        
        \bigskip
        \begin{quote}
        \begin{mdframed}
        \textsc{knapsack}
        \medskip\hrule\medskip
        \begin{itemize}
            \item[\textsc{In:}] A (finite) set $A$, where each $a\in A$ has an associated size $s(a)\in\mathbb N$ and value $v(a)\in\mathbb N$, and two positive integers: capacity $C$ and value $V$.
            \item[\textsc{Q:}] Is there a subset $A'\subseteq A$ such that $\sum_{a\in A'} s(a)\leq C$ and $\sum_{a\in A'} v(a)\geq V$?
        \end{itemize}
        \end{mdframed}
        \end{quote}
        
        
        \bigskip
        \begin{quote}
        \begin{mdframed}
        \textsc{scheduling}
        \medskip\hrule\medskip
        \begin{itemize}
            \item[\textsc{In:}] A number of processors $m\in\mathbb N$, a (finite) set of tasks $T$ where each task $x\in T$ has an associated duration $d(x)\in\mathbb N$, and a positive integer $D$.
            \item[\textsc{Q:}] Is it possible to partition the set $T$ into $m$ parts $T_1,\dots,T_m$ (i.e., pairwise disjoint sets covering $T$) such that $\sum_{x\in T_i} d(x)\leq D$ for every $1\leq i\leq m$?
        \end{itemize}
        \end{mdframed}
        \end{quote}
        



\end{document}