\documentclass[a4paper,12pt]{amsart}

\usetheme[progressbar=frametitle]{metropolis}
\metroset{block=fill}

\subtitle{NTIN071 Automata and Grammars}
\author{Jakub Bulín (KTIML MFF UK)}

\date{Spring 2025\\ 
    \vspace{1in} 
    \begin{flushleft}
        \it \footnotesize * Adapted from the Czech-lecture slides by Marta Vomlelová with gratitude. The translation, some modifications, and all errors are mine.
    \end{flushleft}
}

%% packages

\usepackage{amsmath}
\usepackage{amssymb}
\usepackage{amsthm}
\usepackage{cancel}
\usepackage{color}
\usepackage{colortbl}
\usepackage{forest}
\usepackage[utf8x]{inputenc}
\usepackage{multicol}
\usepackage{multirow}

%% colors
\definecolor{Gray}{gray}{0.9}

%% TikZ
\usepackage{tikz}
    \usetikzlibrary{
        automata,
        arrows,
        backgrounds,
        decorations.pathmorphing,
        fit,
        positioning,
        shapes,
        shapes.geometric,
        tikzmark
    } 
    \tikzset{>=stealth',shorten >=1pt,auto,node distance=2cm}
    \tikzset{initial text={}}
    \tikzset{elliptic state/.style={draw,ellipse}}

%% amsthm
\theoremstyle{plain}
    \newtheorem*{algorithm}{Algorithm}    
    \newtheorem*{observation}{Observation}
    \newtheorem*{proposition}{Proposition}

\theoremstyle{remark}
    \newtheorem*{exercise}{Exercise}
    \newtheorem*{remark}{Remark}

%% macros
\DeclareMathOperator{\RegE}{RegE}
\DeclareMathOperator{\RL}{RL}

% Just for Lecture 2
\newcommand{\x}{$\times$}
\newcommand{\nx}{\ }


\begin{document}

\thispagestyle{empty}

\section*{NTIN071 A\&G: Tutorial 10 -- Coversion between PDA and context-free grammars}

\medskip

\subsection*{Teaching goals:} The student is able to

    \begin{itemize}\setlength{\itemsep}{0pt}
        \item 
    \end{itemize}


\section*{In-class problems}


\medskip\begin{problem}

\end{problem}


\section*{Extra Practice and Thinking}


\medskip\begin{problem}

\end{problem}


\end{document}


\documentclass[a4paper,12pt]{amsart}

\usetheme[progressbar=frametitle]{metropolis}
\metroset{block=fill}

\subtitle{NTIN071 Automata and Grammars}
\author{Jakub Bulín (KTIML MFF UK)}

\date{Spring 2025\\ 
    \vspace{1in} 
    \begin{flushleft}
        \it \footnotesize * Adapted from the Czech-lecture slides by Marta Vomlelová with gratitude. The translation, some modifications, and all errors are mine.
    \end{flushleft}
}

%% packages

\usepackage{amsmath}
\usepackage{amssymb}
\usepackage{amsthm}
\usepackage{cancel}
\usepackage{color}
\usepackage{colortbl}
\usepackage{forest}
\usepackage[utf8x]{inputenc}
\usepackage{multicol}
\usepackage{multirow}

%% colors
\definecolor{Gray}{gray}{0.9}

%% TikZ
\usepackage{tikz}
    \usetikzlibrary{
        automata,
        arrows,
        backgrounds,
        decorations.pathmorphing,
        fit,
        positioning,
        shapes,
        shapes.geometric,
        tikzmark
    } 
    \tikzset{>=stealth',shorten >=1pt,auto,node distance=2cm}
    \tikzset{initial text={}}
    \tikzset{elliptic state/.style={draw,ellipse}}

%% amsthm
\theoremstyle{plain}
    \newtheorem*{algorithm}{Algorithm}    
    \newtheorem*{observation}{Observation}
    \newtheorem*{proposition}{Proposition}

\theoremstyle{remark}
    \newtheorem*{exercise}{Exercise}
    \newtheorem*{remark}{Remark}

%% macros
\DeclareMathOperator{\RegE}{RegE}
\DeclareMathOperator{\RL}{RL}

% Just for Lecture 2
\newcommand{\x}{$\times$}
\newcommand{\nx}{\ }


\begin{document}

% \thispagestyle{empty}

\section*{NTIN071 A\&G: Tutorial 10 -- Pushdown automata}

% after Lecture 9
% spring 2024

\medskip

\noindent\emph{Solve 1a-h, 2ab, 3a first (the rest is for practice).}

\medskip

\medskip\begin{problem}[Constructing PDA]

    Design pushdown automata for the following languages. (Acceptance can be either by final state or by empty stack, in some cases construct both.)

    \bigskip

    \begin{enumerate}[(a)]\setlength\itemsep{12pt}
        \item $L=\{w\mid w\in\{0,1\}^*,|w|_1\geq 3\}$
        \item $L=\{ww^R\mid w\in \{0,1\}^*\}$
        \item $L=\{w\in\{(,)\}^*\mid w\text{ is correctly parenthesized}\}$
        \item $L=\{w\in \{0,1\}^*\mid w=w^R\}$
        \item $L=\{a^ib^jc^k\mid i=j \text{ or } j=k\} $
        \item $L=\{a^ib^jc^k\mid i+j=k\}$
        \item $L=\{a^{2n}b^{3n}\mid n\geq 0\}$
        \item $L=\{w\in \{0,1\}^*\mid  |w|_0=|w|_1\} $
        \item $L=\{u2v\mid u,v\in \{0,1\}^*\text{ and }|u|\neq |v|\}$
        \item $L=\{w\in\{(,),[,]\}^*\mid w\text{ is correctly parenthesized}\}$           
    \end{enumerate}

\end{problem}


\medskip\begin{problem}[Acceptance by final state vs. empty stack]

    Convert selected PDA constructed in the previous problem from acceptance by final state to acceptance by empty stack, and vice versa. (Try both constructions.)

\end{problem}


\medskip\begin{problem}[CFG to PDA]

    For a given context-free grammar $G$ construct pushdown automata $P_1,P_2$ such that $L(G)=N(P_1)=L(P_2)$.

    \bigskip

    \begin{enumerate}[(a)]\setlength\itemsep{12pt}
        \item $G=(\{S,T,X\},\{a,b\},\mathcal P,S)$
            \begin{align*}
        \mathcal P=\{S&\rightarrow aTXb, \\
        T&\rightarrow XTS\mid \epsilon,\\ 
        X&\rightarrow a\mid b\}
        \end{align*}
        \item $G=(\{S,T,X\},\{(,),*,+,1\},\mathcal P,S)$
            \begin{align*}
        \mathcal P=\{S&\rightarrow S+T\mid T, \\
        T&\rightarrow T*X\mid X,\\ 
        X&\rightarrow 1\mid (S)\}
        \end{align*}
    \end{enumerate}

    For a reasonably long word $w\in L(G)$ find its leftmost derivation from $G$ and simulate the automaton  $P_1$ on the input $w$.

\end{problem}
    

\medskip\begin{problem}[PDA to CFG]

    Convert selected (small) pushdown automata constructed in Problem 1 to context-free grammars. For a reasonably long word $w$ accepted by the automaton find a leftmost derivation of $w$ from the grammar.

\end{problem}


\section*{Bonus: Context grammars}


\medskip\begin{problem}[A context grammar]
    
    Consider the grammar $G=(\{S,A,B,C\},\{a,b,c\},S,P)$, where
    \begin{align*}
        P=\{&S\rightarrow aSBC\mid aBC, B\rightarrow BBC,  C\rightarrow CC, CB\rightarrow BC,\\ 
        &aB\rightarrow ab, bB\rightarrow bb, bC\rightarrow bc, cC\rightarrow cc\}
    \end{align*}
    What language does it generate? Is $G$ a context grammar? If not, find an equivalent context grammar.
    
\end{problem}


\end{document}
