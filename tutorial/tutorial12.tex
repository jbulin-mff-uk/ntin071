\documentclass[a4paper,12pt]{amsart}

\usepackage{a4wide}
\usepackage{amsmath}
\usepackage{amssymb}
\usepackage{amsthm}
\usepackage[czech]{babel}
\usepackage{bookmark}
\usepackage{enumerate}
\usepackage[T1]{fontenc}
\usepackage{hyperref}
\usepackage[utf8]{inputenc}
\usepackage{lmodern}
\usepackage{multicol}
\usepackage{tikz}
    \usetikzlibrary{automata, arrows, positioning}


\theoremstyle{definition}
    \newtheorem{problem}{Příklad}



\begin{document}

\thispagestyle{empty}

\section*{NTIN071 A\&G: Tutorial 12 -- Intro to complexity theory}

\medskip

\subsection*{Teaching goals:} The student is able to

    \begin{itemize}\setlength{\itemsep}{0pt}
        \item give formal definitions of $\mathtt{TIME}(f(n))$ and $\mathtt{SPACE}(f(n))$
        \item define the complexity classes $\mathtt{P}$, $\mathtt{NP}$ (both verifier and NTM-based), co-$\mathtt{NP}$
        \item define polynomial-time reductions, $\mathtt{NP}$-hardness, $\mathtt{NP}$-completeness
        \item design polynomial-time reductions between problems
        \item decide whether complexity classes are closed under various operations
    \end{itemize}


\section*{In-class problems}


\medskip
\begin{problem}
    Show that the problems \textsc{clique}, \textsc{independent-set}, and \textsc{vertex-cover} defined below are polynomial-time inter-reducible. 
    
    \smallskip
    \begin{center}
        \fbox{\parbox{0.8\textwidth}{                
        \textsc{clique}
        \medskip\hrule\medskip
        \begin{itemize}
            \item[\textsc{In:}] A graph $G=(V,E)$ and an integer $k\geq 0$.
            \item[\textsc{Q:}] Does $G$ contain (as a subgraph) the complete graph (clique) on at least $k$ vertices?
        \end{itemize}
        }}
    \end{center}
    
    \smallskip
    \begin{center}
        \fbox{\parbox{0.8\textwidth}{ 
        \textsc{independent-set}
        \medskip\hrule\medskip
        \begin{itemize}
            \item[\textsc{In:}] A graph $G=(V,E)$ and an integer $k\geq 0$.
            \item[\textsc{Q:}] Does $G$ contain an independent set of size at least $k$, i.e., $S\subseteq V$, $|S|\geq k$ with no edge connecting a pair of vertices from $S$?
        \end{itemize}
        }}
    \end{center}
    
    \smallskip
    \begin{center}
        \fbox{\parbox{0.8\textwidth}{ 
        \textsc{vertex-cover}
        \medskip\hrule\medskip
        \begin{itemize}
            \item[\textsc{In:}] A graph $G=(V,E)$ and an integer $k\geq 0$.
            \item[\textsc{Q:}] Does $G$ have a vertex cover of size at most $k$, i.e., $S\subseteq V$, $|S|\leq k$ containing at least one vertex from every edge?
        \end{itemize}
        }}
    \end{center}
    
\end{problem}


\medskip
\begin{problem}

    Use the well-known fact that \textsc{hamiltonian-cycle} is $\mathtt{NP}$-complete to show that \textsc{oriented-hamiltonian-cycle}, \textsc{$(s,t)$-hamiltonian-path}, and \textsc{hamiltonian-path} are $\mathtt{NP}$-complete as well.
    
    \smallskip
    \begin{center}
        \fbox{\parbox{0.8\textwidth}{ 
        \textsc{hamiltonian-cycle}
        \medskip\hrule\medskip
        \begin{itemize}
            \item[\textsc{In:}] An (unoriented) graph $G=(V,E)$.
            \item[\textsc{Q:}] Does $G$ contain a Hamiltonian cycle, i.e., a cycle containing every vertex?
        \end{itemize}
        }}
    \end{center}
        
    \smallskip
    \begin{center}
        \fbox{\parbox{0.8\textwidth}{ 
        \textsc{oriented-hamiltonian-cycle}
        \medskip\hrule\medskip
        \begin{itemize}
            \item[\textsc{In:}] An oriented graph $G=(V,E)$.
            \item[\textsc{Q:}] Does $G$ contain an oriented Hamiltonian cycle, i.e., an oriented cycle containing every vertex?
        \end{itemize}
        }}
    \end{center}
       
    \smallskip
    \begin{center}
        \fbox{\parbox{0.8\textwidth}{ 
        \textsc{$(s,t)$-hamiltonian-path}
        \medskip\hrule\medskip
        \begin{itemize}
            \item[\textsc{In:}] An (unoriented) graph $G=(V,E)$ and a pair of vertices $s,t\in V$.
            \item[\textsc{Q:}] Does $G$ contain a Hamiltonian path from $s$ to $t$, i.e., a path that starts in $s$, ends in $t$, and visits every vertex exactly once?
        \end{itemize}
        }}
    \end{center}
        
    \smallskip
    \begin{center}
        \fbox{\parbox{0.8\textwidth}{ 
        \textsc{hamiltonian-path}
        \medskip\hrule\medskip
        \begin{itemize}
            \item[\textsc{In:}] An (unoriented) graph $G=(V,E)$.
            \item[\textsc{Q:}] Does $G$ contain a Hamiltonian path, i.e., a path that visits every vertex exactly once?
        \end{itemize}
        }}
    \end{center}
        
\end{problem}


\medskip
\begin{problem}
    Show that the class $\mathtt{P}$ is closed under union, intersection, and complement.
\end{problem}


\medskip    
\begin{problem}
    Show that the class $\mathtt{NP}$ is closed under union and intersection.
\end{problem}


\section*{Extra Practice and Thinking}


\medskip
\begin{problem}
    Show that \textsc{vertex-cover} is polynomial-time reducible to \textsc{dominating-set}.
    
    \smallskip
    \begin{center}
        \fbox{\parbox{0.8\textwidth}{ 
        \textsc{dominating-set}
        \medskip\hrule\medskip
        \begin{itemize}
            \item[\textsc{In:}] A graph $G=(V,E)$ and an integer $k\geq 0$.
            \item[\textsc{Q:}] Does $G$ contain a set of vertices $S\subseteq V$ of size at most $k$ such that every $v\in V\setminus S$ has a neighbor in $S$?
        \end{itemize}
        }}
    \end{center}
    
\end{problem}


\medskip
\begin{problem}

    Show that \textsc{hamiltonian-cycle} is polynomial-time reducible to \textsc{traveling-salesperson}.        

    \smallskip
    \begin{center}
        \fbox{\parbox{0.8\textwidth}{ 
        \textsc{traveling-salesperson}
        \medskip\hrule\medskip
        \begin{itemize}
            \item[\textsc{In:}] A list of cities $C=\{c_1,\dots,c_n\}$, distances $d(c_i,c_j)\in\mathbb N$ between each pair of cities, and $D\in\mathbb N$.
            \item[\textsc{Q:}] Is there a route of length at most $D$ that visits every city exactly once and returns to the origin city?
        \end{itemize}
        }}
    \end{center}
        
\end{problem}


\medskip
\begin{problem}

    Show that \textsc{hamiltonian-cycle} is polynomial-time reducible to \textsc{sat}.        
        
\end{problem}


\begin{problem}

    Show that \textsc{graph-coloring} is NP-complete.

    \smallskip
    \begin{center}
        \fbox{\parbox{0.8\textwidth}{ 
        \textsc{graph-coloring}
        \medskip\hrule\medskip
        \begin{itemize}
            \item[\textsc{In:}] A graph $G=(V,E)$ and $k\in\mathbb N$.
            \item[\textsc{Q:}] Can we color vertices of $G$ with at most $k$ colors so that there are no monochromatic edges?
        \end{itemize}
        }}
    \end{center}
        
\end{problem}
       

\begin{problem}

    Show that the class $\mathtt{P}$ is closed under iteration. That is, if $L\in\mathtt{P}$, then $L^*$ is also in $\mathtt{P}$. (Hint: Design a table-filling algorithm where $T[i,j]=1$ iff $a_i\dots a_j\in L^*$.)

\end{problem}


\end{document}
