\documentclass[a4paper,12pt]{amsart}

\usetheme[progressbar=frametitle]{metropolis}
\metroset{block=fill}

\subtitle{NTIN071 Automata and Grammars}
\author{Jakub Bulín (KTIML MFF UK)}

\date{Spring 2025\\ 
    \vspace{1in} 
    \begin{flushleft}
        \it \footnotesize * Adapted from the Czech-lecture slides by Marta Vomlelová with gratitude. The translation, some modifications, and all errors are mine.
    \end{flushleft}
}

%% packages

\usepackage{amsmath}
\usepackage{amssymb}
\usepackage{amsthm}
\usepackage{cancel}
\usepackage{color}
\usepackage{colortbl}
\usepackage{forest}
\usepackage[utf8x]{inputenc}
\usepackage{multicol}
\usepackage{multirow}

%% colors
\definecolor{Gray}{gray}{0.9}

%% TikZ
\usepackage{tikz}
    \usetikzlibrary{
        automata,
        arrows,
        backgrounds,
        decorations.pathmorphing,
        fit,
        positioning,
        shapes,
        shapes.geometric,
        tikzmark
    } 
    \tikzset{>=stealth',shorten >=1pt,auto,node distance=2cm}
    \tikzset{initial text={}}
    \tikzset{elliptic state/.style={draw,ellipse}}

%% amsthm
\theoremstyle{plain}
    \newtheorem*{algorithm}{Algorithm}    
    \newtheorem*{observation}{Observation}
    \newtheorem*{proposition}{Proposition}

\theoremstyle{remark}
    \newtheorem*{exercise}{Exercise}
    \newtheorem*{remark}{Remark}

%% macros
\DeclareMathOperator{\RegE}{RegE}
\DeclareMathOperator{\RL}{RL}

% Just for Lecture 2
\newcommand{\x}{$\times$}
\newcommand{\nx}{\ }


\begin{document}

\thispagestyle{empty}

\section*{NTIN071 A\&G: Tutorial 7 -- Chomsky normal form, The CYK algorithm}

\medskip

\subsection*{Teaching goals:} The student is able to

    \begin{itemize}\setlength{\itemsep}{0pt}
        \item give the formal definition of Chomsky Normal Form and related notions
        \item convert a given context-free grammar to ChNF
        \item explain the CYK algorithm, apply to a given word and context-free grammar
    \end{itemize}

\medskip

\section*{In-class problems}


\medskip\begin{problem}[About the conversion to ChNF]
    
    Recall the process of converting a context-free grammar to Chomsky Normal Form. Then answer the following questions. Justify.
    
    \begin{enumerate}[(a)]\setlength{\itemsep}{6pt}
        \item Find an example of a grammar in which there is a generating variable only reachable via nongenerating variables.
        \item When reducing a grammar, which variables do we need to remove first: nongenerating or unreachable?
        \item Is it possible for a reachable generating variable to become nongenerating after the removal of unreachable variables?
        \item When we want to break up a production rule with long body, what is the minimal number of Chomsky Normal Form rules we need to create?
    \end{enumerate}

\end{problem}


\medskip\begin{problem}[Convert to ChNF]

    \label{prob:chnf}
    Convert the following context-free grammars to Chomsky normal form:
    
    \begin{multicols}{2}
        \begin{enumerate}[(a)]    
            \item $G_1=(\{S,A,B\},\{0,1\},S,\mathcal P)$, where
            \begin{align*}
                \mathcal P=\{&S\rightarrow 0AB, \\
                &A\rightarrow 0A0\mid 11,\\
                &B\rightarrow 0\}
            \end{align*}
    
            \item $G_2=(\{S,A,B\},\{0,1\},S,\mathcal P)$, where
            \begin{align*}            
                \mathcal P=\{
                &S\rightarrow 0A10B10, \\
                &A\rightarrow 1A0\mid \epsilon,\\
                &B\rightarrow 1B00\mid \epsilon\} 
            \end{align*}
        \end{enumerate}
    \end{multicols}
        
\end{problem}


\medskip\begin{problem}[The CYK algorithm]
    
    Using the CYK algorithm determine if $w\in L(G)$.

    \begin{enumerate}[(a)]\setlength{\itemsep}{6pt}
        
        \item $w=0110$, $G=(\{S,A,B\},\{0, 1\},S,\mathcal P)$, where      
        
        \begin{align*}
            \mathcal P=\{S&\rightarrow 0\mid AB, \\
            A&\rightarrow 1\mid SA\mid SB, \\
            B&\rightarrow AS \mid BA \mid 0\}
        \end{align*}

        \item $w=001100$, $G=G_1$ is the grammar from Problem \ref{prob:chnf}(a)
        
        \item $w=110011$, $G=G_1$ is the grammar from Problem \ref{prob:chnf}(a)
        
    \end{enumerate}
       

\end{problem}


\section*{Extra Practice and Thinking}


\medskip\begin{problem}[Convert to ChNF]

    Convert the following to Chomsky normal form:
    \begin{multicols}{2}
        \begin{enumerate}[(a)]
    
            \item $G=(\{S,A,B\},\{0,1\},S,\mathcal P)$
            \begin{align*}
                \mathcal P=\{&S\rightarrow A\mid 0SA\mid \epsilon, \\
                &A\rightarrow 1A\mid 1\mid B1,\\
                &B\rightarrow 0B\mid 0\mid \epsilon\} 
            \end{align*}
    
            \item $G=(\{S,E,F\},\{(,),*,+,,1\},S,\mathcal P)$
            \begin{align*}
                \mathcal P=\{&
                S\rightarrow (E), \\
                &E\rightarrow F+F\mid F*F,\\
                &F\rightarrow S\mid 1\}
            \end{align*}

        \end{enumerate}
    \end{multicols}
        
\end{problem}


\medskip\begin{problem}[The CYK algorithm]
    
    Using the CYK algorithm determine if $w\in L(G)$.

    \begin{enumerate}[(a)]\setlength{\itemsep}{6pt}

        \item $w=abcbb$, $G=(\{S,A,B,C\},\{a,b,c\},S,\mathcal P)$, where
        
        \begin{align*}
            \mathcal P=\{S&\rightarrow CA\mid CB, \\
            B&\rightarrow CBA\mid CB\mid BA\mid BB, \\
            C&\rightarrow ABC\mid BC,\\
            A&\rightarrow a, B\rightarrow b, C\rightarrow c\}
        \end{align*}    

        \item $w=01010010$, $G=G_2$ is the grammar from Problem \ref{prob:chnf}(b)
        \item $w=01010011$, $G=G_2$ is the grammar from Problem \ref{prob:chnf}(b)

    \end{enumerate}

\end{problem}


\end{document}
   