\documentclass[a4paper,12pt]{article}
\usepackage{a4wide}
\usepackage{amsmath}
\usepackage[utf8]{inputenc}
\usepackage[english]{babel}

\begin{document}


\thispagestyle{empty}

\begin{center}
    \large{NTIN071 A\&G: List of requirements for the tutorial test}    
\end{center}

\bigskip

\begin{itemize}
    \item Finite automaton for a given regular language (DFA, NFA, $\epsilon$-NFA), extended transition function.
    \item Proof of non-regularity (Pumping lemma for regular languages, Myhill-Nerode theorem, closure properties).
    \item State equivalence algorithm, construction of reduced DFA.
    \item Conversion of $\epsilon$-NFA or NFA to DFA (subset construction).
    \item Conversion from regular expression to finite automaton and vice versa (including state elimination algorithm).
    \item Right-linear grammar for a given regular language, derivation.
    \item Conversion of right-linear grammar to finite automaton and vice versa.
    \item Context-free grammar for a given context-free language, derivation.
    \item Conversion of context-free grammar to Chomsky normal form.
    \item The CYK algorithm.
    \item Proof of non-context-freeness (Pumping lemma for context-free languages, closure properties).
    \item Construction of a pushdown automaton (acceptance by final state, empty stack, conversion between them), sequence of configurations.
    \item Conversion of context-free grammar to pushdown automaton.
    \item Turing machine for a given language, sequence of configurations.
    \item Classification of a language into Chomsky hierarchy: it is enough to  prove regularity, prove non-regularity and context-freeness, or prove non-context-freeness.
\end{itemize}


\end{document}

