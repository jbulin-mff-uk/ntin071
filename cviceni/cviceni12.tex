\documentclass[a4paper,12pt]{amsart}

\usepackage[framemethod=TikZ]{mdframed}
\usetheme[progressbar=frametitle]{metropolis}
\metroset{block=fill}

\subtitle{NTIN071 Automata and Grammars}
\author{Jakub Bulín (KTIML MFF UK)}

\date{Spring 2025\\ 
    \vspace{1in} 
    \begin{flushleft}
        \it \footnotesize * Adapted from the Czech-lecture slides by Marta Vomlelová with gratitude. The translation, some modifications, and all errors are mine.
    \end{flushleft}
}

%% packages

\usepackage{amsmath}
\usepackage{amssymb}
\usepackage{amsthm}
\usepackage{cancel}
\usepackage{color}
\usepackage{colortbl}
\usepackage{forest}
\usepackage[utf8x]{inputenc}
\usepackage{multicol}
\usepackage{multirow}

%% colors
\definecolor{Gray}{gray}{0.9}

%% TikZ
\usepackage{tikz}
    \usetikzlibrary{
        automata,
        arrows,
        backgrounds,
        decorations.pathmorphing,
        fit,
        positioning,
        shapes,
        shapes.geometric,
        tikzmark
    } 
    \tikzset{>=stealth',shorten >=1pt,auto,node distance=2cm}
    \tikzset{initial text={}}
    \tikzset{elliptic state/.style={draw,ellipse}}

%% amsthm
\theoremstyle{plain}
    \newtheorem*{algorithm}{Algorithm}    
    \newtheorem*{observation}{Observation}
    \newtheorem*{proposition}{Proposition}

\theoremstyle{remark}
    \newtheorem*{exercise}{Exercise}
    \newtheorem*{remark}{Remark}

%% macros
\DeclareMathOperator{\RegE}{RegE}
\DeclareMathOperator{\RL}{RL}

% Just for Lecture 2
\newcommand{\x}{$\times$}
\newcommand{\nx}{\ }


\begin{document}

\thispagestyle{empty}

\section*{NTIN071 A\&G: Cvičení 12 -- Úvod do výpočetní složitosti}

\medskip

\subsection*{Cíle výuky:} Po absolvování student umí

\begin{itemize}\setlength{\itemsep}{0pt}
    \item uvést formální definice tříd $\mathtt{TIME}(f(n))$ a $\mathtt{SPACE}(f(n))$
    \item definovat složitostní třídy $\mathtt{P}$, $\mathtt{NP}$ (jak na základě verifikátoru, tak NTM), co-$\mathtt{NP}$
    \item definovat polynomiální redukci, $\mathtt{NP}$-těžkost a $\mathtt{NP}$-úplnost
    \item zkonstruovat polynomiální redukci mezi problémy
    \item určit, zda jsou třídy složitosti uzavřené na různé operace
\end{itemize}


\section*{Příklady na cvičení}


\medskip
\begin{problem}
    Ukažte, že problémy \textsc{clique}, \textsc{independent-set} a \textsc{vertex-cover}, definované níže, jsou na sebe navzájem polynomiálně redukovatelné.
    \smallskip  
    \begin{center}  
        \fbox{\parbox{0.8\textwidth}{                
        \textsc{clique}  
        \medskip\hrule\medskip  
        \begin{itemize}  
            \item[\textsc{In:}] Graf $G=(V,E)$ a celé číslo $k\geq 0$.  
            \item[\textsc{Q:}] Obsahuje $G$ (jako podgraf) úplný podgraf na alespoň $k$ vrcholech?  
        \end{itemize}  
        }}  
    \end{center}  

    \smallskip  
    \begin{center}  
        \fbox{\parbox{0.8\textwidth}{  
        \textsc{independent-set}  
        \medskip\hrule\medskip  
        \begin{itemize}  
            \item[\textsc{In:}] Graf $G=(V,E)$ a celé číslo $k\geq 0$.  
            \item[\textsc{Q:}] Obsahuje $G$ nezávislou množinu alespoň $k$ vrcholů, tj. množinu $S\subseteq V$, $|S|\geq k$, kde žádné dva vrcholy nejsou spojeny hranou?  
        \end{itemize}  
        }}  
    \end{center}  

    \smallskip  
    \begin{center}  
        \fbox{\parbox{0.8\textwidth}{  
        \textsc{vertex-cover}  
        \medskip\hrule\medskip  
        \begin{itemize}  
            \item[\textsc{In:}] Graf $G=(V,E)$ a celé číslo $k\geq 0$.  
            \item[\textsc{Q:}] Obsahuje $G$ vrcholové pokrytí velikosti nejvýše $k$, tj. množinu $S\subseteq V$, $|S|\leq k$, která má alespoň jeden vrchol z každé hrany?  
        \end{itemize}  
        }}  
    \end{center}  

\end{problem}

\medskip
\begin{problem}

    Použijte známý fakt, že \textsc{hamiltonian-cycle} je $\mathtt{NP}$-úplný, a ukažte, že \textsc{oriented-hamiltonian-cycle}, \textsc{$(s,t)$-hamiltonian-path} a \textsc{hamiltonian-path} jsou také $\mathtt{NP}$-úplné.  

    \smallskip  
    \begin{center}  
        \fbox{\parbox{0.8\textwidth}{  
        \textsc{hamiltonian-cycle}  
        \medskip\hrule\medskip  
        \begin{itemize}  
            \item[\textsc{In:}] Neorientovaný graf $G=(V,E)$.  
            \item[\textsc{Q:}] Obsahuje $G$ Hamiltonovskou kružnici, tj. kružnici obsahující každý vrchol?  
        \end{itemize}  
        }}  
    \end{center}  

    \smallskip  
    \begin{center}  
        \fbox{\parbox{0.8\textwidth}{  
        \textsc{oriented-hamiltonian-cycle}  
        \medskip\hrule\medskip  
        \begin{itemize}  
            \item[\textsc{In:}] Orientovaný graf $G=(V,E)$.  
            \item[\textsc{Q:}] Obsahuje $G$ orientovanou Hamiltonovskou kružnici, tj. orientovanou kružnici obsahující každý vrchol?  
        \end{itemize}  
        }}  
    \end{center}  

    \smallskip  
    \begin{center}  
        \fbox{\parbox{0.8\textwidth}{  
        \textsc{$(s,t)$-hamiltonian-path}  
        \medskip\hrule\medskip  
        \begin{itemize}  
            \item[\textsc{In:}] Neorientovaný graf $G=(V,E)$ a dvojice vrcholů $s,t\in V$.  
            \item[\textsc{Q:}] Obsahuje $G$ Hamiltonovskou cestu z $s$ do $t$, tj. cestu, která začíná v $s$, končí v $t$ a prochází každý vrchol právě jednou?  
        \end{itemize}  
        }}  
    \end{center}  

    \smallskip  
    \begin{center}  
        \fbox{\parbox{0.8\textwidth}{  
        \textsc{hamiltonian-path}  
        \medskip\hrule\medskip  
        \begin{itemize}  
            \item[\textsc{In:}] Neorientovaný graf $G=(V,E)$.  
            \item[\textsc{Q:}] Obsahuje $G$ Hamiltonovskou cestu, tj. cestu, která prochází každý vrchol právě jednou?  
        \end{itemize}  
        }}  
    \end{center}  

\end{problem}


\medskip
\begin{problem}
    Ukažte, že třída $\mathtt{P}$ je uzavřená na sjednocení, průnik a doplněk.
\end{problem}

\medskip
\begin{problem}
    Ukažte, že třída $\mathtt{NP}$ je uzavřená na sjednocení a průnik.
\end{problem}


\section*{K procvičení a k zamyšlení}


\medskip
\begin{problem}
    
    Ukažte, že \textsc{vertex-cover} má polynomiální redukci na \textsc{dominating-set}.

    \smallskip  
    \begin{center}  
        \fbox{\parbox{0.8\textwidth}{  
        \textsc{dominating-set}  
        \medskip\hrule\medskip  
        \begin{itemize}  
            \item[\textsc{In:}] Graf $G=(V,E)$ a celé číslo $k\geq 0$.  
            \item[\textsc{Q:}] Obsahuje $G$ množinu vrcholů $S\subseteq V$ o velikosti nejvýše $k$ takovou, že každý vrchol $v\in V\setminus S$ má souseda v $S$?  
        \end{itemize}  
        }}  
    \end{center}  

\end{problem}

\medskip
\begin{problem}

    Sestrojte polynomiální redukci \textsc{hamiltonian-cycle} na \textsc{traveling-salesperson}.        

    \smallskip  
    \begin{center}  
        \fbox{\parbox{0.8\textwidth}{  
        \textsc{traveling-salesperson}  
        \medskip\hrule\medskip  
        \begin{itemize}  
            \item[\textsc{In:}] Množina měst $C=\{c_1,\dots,c_n\}$, vzdálenosti $d(c_i,c_j)\in\mathbb N$ mezi každou dvojicí měst a číslo $D\in\mathbb N$.  
            \item[\textsc{Q:}] Existuje cesta délky nejvýše $D$, která navštíví každé město právě jednou a vrátí se do výchozího města?  
        \end{itemize}  
        }}  
    \end{center} 

\end{problem}


\begin{problem}

    Ukažte, že \textsc{hamiltonian-cycle} je polynomiálně  redukovatelný na \textsc{sat}.        

\end{problem}


\begin{problem}

    Ukažte, že \textsc{graph-coloring} je NP-úplný.  

    \smallskip  
    \begin{center}  
        \fbox{\parbox{0.8\textwidth}{  
        \textsc{graph-coloring}  
        \medskip\hrule\medskip  
        \begin{itemize}  
            \item[\textsc{In:}] Graf $G=(V,E)$ a číslo $k\in\mathbb N$.  
            \item[\textsc{Q:}] Lze obarvit vrcholy grafu $G$ nejvýše $k$ barvami tak, aby žádná hrana nespojovala dva vrcholy stejné barvy?  
        \end{itemize}  
        }}  
    \end{center}  

\end{problem}


\begin{problem}

    Ukažte, že třída $\mathtt{P}$ je uzavřená na iteraci. To znamená, že pokud $L\in\mathtt{P}$, pak také $L^*$ patří do $\mathtt{P}$. (Nápověda: Navrhněte dynamický program vyplňující tabulku, kde $T[i,j]=1$ právě tehdy, když $a_i\dots a_j\in L^*$.)

\end{problem}


\end{document}
