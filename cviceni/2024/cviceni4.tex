\documentclass[a4paper,12pt]{amsart}

\usetheme[progressbar=frametitle]{metropolis}
\metroset{block=fill}

\subtitle{NTIN071 Automata and Grammars}
\author{Jakub Bulín (KTIML MFF UK)}

\date{Spring 2025\\ 
    \vspace{1in} 
    \begin{flushleft}
        \it \footnotesize * Adapted from the Czech-lecture slides by Marta Vomlelová with gratitude. The translation, some modifications, and all errors are mine.
    \end{flushleft}
}

%% packages

\usepackage{amsmath}
\usepackage{amssymb}
\usepackage{amsthm}
\usepackage{cancel}
\usepackage{color}
\usepackage{colortbl}
\usepackage{forest}
\usepackage[utf8x]{inputenc}
\usepackage{multicol}
\usepackage{multirow}

%% colors
\definecolor{Gray}{gray}{0.9}

%% TikZ
\usepackage{tikz}
    \usetikzlibrary{
        automata,
        arrows,
        backgrounds,
        decorations.pathmorphing,
        fit,
        positioning,
        shapes,
        shapes.geometric,
        tikzmark
    } 
    \tikzset{>=stealth',shorten >=1pt,auto,node distance=2cm}
    \tikzset{initial text={}}
    \tikzset{elliptic state/.style={draw,ellipse}}

%% amsthm
\theoremstyle{plain}
    \newtheorem*{algorithm}{Algorithm}    
    \newtheorem*{observation}{Observation}
    \newtheorem*{proposition}{Proposition}

\theoremstyle{remark}
    \newtheorem*{exercise}{Exercise}
    \newtheorem*{remark}{Remark}

%% macros
\DeclareMathOperator{\RegE}{RegE}
\DeclareMathOperator{\RL}{RL}

% Just for Lecture 2
\newcommand{\x}{$\times$}
\newcommand{\nx}{\ }


\begin{document}

\thispagestyle{empty}

\section*{NTIN071 A\&G: Cvičení 4 -- Nedeterminismus, uzavřenost na operace}

% po 3. přednášce
% jaro 2024

\medskip

\noindent\emph{Vyřešte nejprve 1a, 2, 3 (zbytek je na procvičení, bude-li naopak čas, vraťte se k nedodělaným příkladům z minule).}

\medskip


\medskip\begin{problem}[Podmnožinová konstrukce]

    Pro daný nedeterministický automat s $\epsilon$-přechody sestrojte ekvivalentní redukovaný DFA.
    
    \begin{multicols}{2}
    
        \begin{enumerate}[(a)]
            \item
            \begin{tabular}{ r | c c c }
            & a & b & $\epsilon$ \\ \hline
            $\ast A$ & $\{A,C\}$ & $\{B\}$ & $\emptyset$ \\
            $B$ & $\{B,D\}$ & $\emptyset$ & $\emptyset$ \\
            $\ast C$ & $\{E\}$ & $\{D\}$ & $\emptyset$ \\
            $D$ & $\{A\}$ & $\{C,D\}$ & $\emptyset$ \\
            $\to\ast E$ & $\emptyset$ & $\emptyset$ & $\{A,C\}$
            \end{tabular}
            
            \item
            \begin{tabular}{ r | c c c }
            & a & b & $\epsilon$ \\ \hline
            $\to A$ & $\{E\}$ & $\{B\}$ & $\emptyset$ \\
            $B$ & $\emptyset$ & $\{C\}$ & $\{D\}$ \\
            $\to C$ & $\emptyset$ & $\{D\}$ & $\emptyset$ \\
            $\ast D$ & $\emptyset$ & $\emptyset$ & $\emptyset$ \\
            $E$ & $\{F\}$ & $\emptyset$ & $\{B,C\}$\\
            $F$ & $\{D\}$ & $\emptyset$ & $\emptyset$
            \end{tabular}
            
        \end{enumerate}
    
    \end{multicols}
\end{problem}


\medskip\begin{problem}[Uzavřenost na operace]

    Pro danou dvojici automatů $A,B$ sestrojte automat, který přijímá daný jazyk.

    \begin{multicols}{3}
    
        \begin{enumerate}[(a)]\setlength\itemsep{12pt}
            \item $L(A)-L(B)$
            \item $L(A).L(B)$
            \item $L(A)^+$
            \item $L(A)^*$
            \item $L(A)^R$
        \end{enumerate}
        
        \begin{tabular}{ r | c c }
        A & a & b \\ \hline
        $\to\ast$ 0 & 1 & 2 \\  
        1 & 3 & 0 \\
        2 & 4 & 5 \\
        3 & 0 & 2 \\
        4 & 2 & 5 \\
        5 & 0 & 3
        \end{tabular}    
        
        \begin{tabular}{ r | c c }
        B & a & b \\ \hline
        $\to\ast$ 0 & 0 & 5 \\  
        1 & 1 & 3 \\
        2 & 2 & 7 \\
        3 & 3 & 2 \\
        $\ast$ 4 & 6 & 1 \\
        5 & 5 & 1 \\
        $\ast$ 6 & 4 & 2 \\
        7 & 7 & 0
        \end{tabular}
    
    \end{multicols}
    
\end{problem}
    

\medskip\begin{problem}[Mazání]

    Mějme nějaký regulární jazyk $L$ nad abecedou $\Sigma=\{a,b\}$. Popište následující jazyky v množinovém zápisu. Jsou tyto jazyky (nutně) také regulární? Dokažte nebo vyvraťte. 
    
    \medskip

    Jazyk sestávající ze všech slov vzniklých ze slov jazyka $L$\dots
    
    \medskip
    \begin{enumerate}[(a)]\setlength\itemsep{12pt}
            \item \dots smazáním všech výskytů písmene $a$.
            \item \dots smazáním počátečního písmene a zapsáním tohoto písmene na konec slova.
            \item \dots smazáním nejdelší souvislé posloupnosti $a$ček ze začátku slova.
    \end{enumerate}

\end{problem}

    
\medskip\begin{problem}[Záměna přijímajících a nepřijímajících stavů]
        
    Zaměníme-li u daného NFA přijímající a nepřijímající stavy, bude jazyk přijímaný výsledným automatem doplňkem jazyka přijímaného původním automatem? Zdůvodněte.
        
\end{problem}




\end{document}
