\documentclass[a4paper,12pt]{amsart}

\usepackage{a4wide}
\usepackage{amsmath}
\usepackage{amssymb}
\usepackage{amsthm}
\usepackage[czech]{babel}
\usepackage{bookmark}
\usepackage{enumerate}
\usepackage[T1]{fontenc}
\usepackage{hyperref}
\usepackage[utf8]{inputenc}
\usepackage{lmodern}
\usepackage{multicol}
\usepackage{tikz}
    \usetikzlibrary{automata, arrows, positioning}


\theoremstyle{definition}
    \newtheorem{problem}{Příklad}



\begin{document}

%\thispagestyle{empty}

\section*{NTIN071 A\&G: Cvičení 2 -- Regulární jazyky, Pumping lemma}

% po 1. přednášce
% jaro 2024

\medskip

\noindent\emph{Vyřešte nejprve 1ab, 2, 3a, 4a-f (zbytek je na procvičení).}

\medskip\begin{problem}[Regulární jazyky a množinové operace]

    Mějme dva regulární jazyky, $L,L'$ nad stejnou abecedou. Ukažte, že platí následující:
  
    \medskip
    
    \begin{enumerate}[(a)]\setlength\itemsep{12pt}
        \item $\Sigma^*\setminus L$ je regulární jazyk
        \item $L\cup L'$ je regulární jazyk  
        \item $L\cap L'$ je regulární jazyk
        \item Co když mají $L$ a $L'$ různé abecedy (které ale mohou sdílet některé symboly)?
        \item Uměli byste ukázat, že $L^R$ (tj. slova z $L$ napsaná pozpátku) je také regulární jazyk?
    \end{enumerate}
    
\end{problem}


\medskip\begin{problem}[Pumping lemma: formulace]

    \begin{enumerate}[(a)]\setlength\itemsep{12pt}
        \item Zformulujte pumping lemma pro regulární jazyky (bez nahlížení do poznámek z přednášky).
        \item Jak souvisí $n$ z lemmatu s automatem rozpoznávajícím daný jazyk?
        \item Dokažte je (bez nahlížení do poznámek z přednášky).
    \end{enumerate}

\end{problem}


\medskip\begin{problem}[Pumping lemma: zobecnění]

    \begin{enumerate}[(a)]\setlength\itemsep{12pt}
        \item Můžeme podmínku $|uv|\leq n$ v Pumping lemmatu nahradit za $|vw|\leq n$, tedy \emph{iterovat blízko konce}? Dokažte nebo vyvraťte.
        \item Můžeme iterovat blízko předem zvoleného místa ve slově? Jak zformulovat (a dokázat) takové zesílení?
    \end{enumerate}

\end{problem}


\medskip\begin{problem}[Pumping lemma: aplikace]

    Určete, které z následujících jazyků nejsou regulární. Dokažte to pomocí pumping lemmatu. (Jazyky jsou nad abecedou $\Sigma=\{a,b\}$.)

    \medskip
  
    \begin{enumerate}[(a)]\setlength\itemsep{12pt}
        \item $L=\{aa, ab, ba\}$
        \item $L=\{a^ib^j\ \mid\ i\leq j\}$
        \item $L=\{a^ib^j\ \mid\ i\geq j\}$
        \item $L_k=\{a^ib^j\ \mid\ i\leq j\leq k\}$ pro dané $k\in\mathbb N$
        \item $L=\{a^{2^i}\ \mid\ i\geq 0\}$
        \item $L=\{ww^R\ \mid \ w\in\Sigma^*\}$, kde $w^R$ označuje slovo $w$ napsané pozpátku
        \item $L=\{a^ib^{i+j}a^j\ \mid\ i,j\geq 0\}$
        \item $L=\{ww\ \mid \ w\in\Sigma^*\}$
        
    \end{enumerate}
  
\end{problem}


\end{document}
