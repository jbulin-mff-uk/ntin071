\documentclass[a4paper,12pt]{article}
\usepackage{a4wide}
\usepackage{amsmath}
\usepackage[utf8]{inputenc}
\usepackage[czech]{babel}

\begin{document}


\thispagestyle{empty}

\begin{center}
    \large{NTIN071 A\&G: Seznam požadavků na zápočtový test}    
\end{center}

\bigskip

\begin{itemize}
    \item Konečný automat pro daný regulární jazyk (DFA, NFA, $\epsilon$-NFA), rozšířená přechodová funkce.
    \item Důkaz neregularity (Pumping lemma pro regulární jazyky, Myhill-Nerodeova věta, uzávěrové vlastnosti).
    \item Algoritmus ekvivalence stavů, konstrukce redukovaného DFA.
    \item Převod $\epsilon$-NFA resp. NFA na DFA (podmnožinová konstrukce).
    \item Převod z regulárního výrazu na konečný automat a naopak (včetně algoritmu eliminace stavů).
    \item Pravá lineární gramatika pro daný regulární jazyk, derivace.
    \item Převod pravé lineární gramatiky na konečný automat a naopak.
    \item Bezkontextová gramatika pro daný bezkontextový jazyk, derivace.
    \item Převod bezkontextové gramatiky do Chomského normální formy.
    \item Algoritmus CYK.
    \item Důkaz nebezkontextovosti (Pumping lemma pro bezkontextové jazyky, uzávěrové vlastnosti).
    \item Konstrukce zásobníkového automatu (přijímání koncovým stavem, prázdným zásobníkem, převod mezi nimi), posloupnost konfigurací.
    \item Převod bezkontextové gramatiky na zásobníkový automat.
    \item Turingův stroj pro daný jazyk, posloupnost konfigurací.
    \item Zařazení jazyka do Chomského hierarchie: stačí důkaz regularity, důkaz neregularity a bezkontextovosti, nebo důkaz nebezkontextovosti.
\end{itemize}


\end{document}

