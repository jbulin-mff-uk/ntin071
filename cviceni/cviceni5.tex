\documentclass[a4paper,12pt]{amsart}

\usepackage{a4wide}
\usepackage{amsmath}
\usepackage{amssymb}
\usepackage{amsthm}
\usepackage[czech]{babel}
\usepackage{bookmark}
\usepackage{enumerate}
\usepackage[T1]{fontenc}
\usepackage{hyperref}
\usepackage[utf8]{inputenc}
\usepackage{lmodern}
\usepackage{multicol}
\usepackage{tikz}
    \usetikzlibrary{automata, arrows, positioning}


\theoremstyle{definition}
    \newtheorem{problem}{Příklad}



\begin{document}

% \thispagestyle{empty}

\section*{NTIN071 A\&G: Cvičení 5 -- Regulární výrazy}

% po 4. přednášce
% jaro 2024

\medskip

\noindent\emph{Vyřešte nejprve 1a-f, 2a-e, 3abc, 4a (zbytek je na procvičení).}

\medskip

\medskip\begin{problem}[Konstrukce regulárních výrazů]

    Najděte regulární výrazy reprezentující jazyky nad abecedou $\Sigma = \{a, b\}$ sestávající ze slov, která:

    \medskip
    
    \begin{multicols}{2}
    
        \begin{enumerate}[(a)]\setlength\itemsep{12pt}
            \item začínají `abba',
            \item končí  `abba',
            \item začínají `ab' a končí `ba',
            \item obsahují `abba' nebo `bab' jako podslovo,
            \item neobsahují `aa' jako podslovo,
            \item obsahují sudý počet výskytů písmene `a',
            \item mají alespoň 2 písmena a začínají a končí stejným písmenem,
            \item mají alespoň 4 písmena a začínají a končí stejnou dvojicí písmen.
        \end{enumerate}

    \end{multicols}

\end{problem}
    
    
\medskip\begin{problem}[Převod regexu na automat]

    Zkonstruujte konečné automaty přijímající jazyky popsané následujícími regulárními výrazy.
    
    \begin{multicols}{2}
    
        \begin{enumerate}[(a)]\setlength\itemsep{12pt}
            \item $ab + ba$
            \item $a^2 + b^2 + ab$
            \item $a + b^*$
            \item $(ab + c)^*$
            \item $((ab + c)+a(bc)^* + b)^*$
            \item $((ab + c)^*a(bc)^* + b)^*$
            \item $(01^* + 101)^*0^*1$
            \item $(01)^*11(01)^*(0 + 1)^*00$
        \end{enumerate}

    \end{multicols}
    
\end{problem}


\medskip\begin{problem}[Převod automatu na regex]

    Sestrojte regulární výrazy pro jazyky přijímané následujícími automaty.

    \bigskip
    
    % \begin{multicols}{2}

        \begin{enumerate}[(a)]
        
            \item \begin{tikzpicture}[>=stealth',shorten >=1pt,auto,node distance=1.2cm]
                    \tikzset{every state/.style={minimum size=0.2cm}}
                        
                        \node[initial,state]  (a1)      {1};
                        \node[state,accepting] (b1)  [right of=a1]    {2};
                            \node[state] [right of=b1](c1)      {3};
                        \node[state,accepting] [right of=c1] (d1)      {4};
                        \node[state] (e1)  [right of=d1]    {5};
                        \node[state,accepting] (f1)  [right of=e1]    {6};
                        \node[state] (g1)  [below left of=f1]    {7};
                    \path[->]
                            (a1)  edge  node {a} (b1)
                            (b1)  edge  node {a} (c1)
                            (c1)  edge  node {a} (d1)
                            (d1)  edge  node {a} (e1)
                            (e1)  edge  node {a} (f1)
                            (f1)  edge  node {a} (g1)
                            (g1)  edge  node {a} (e1)
                            ;
                    \end{tikzpicture}            
            
            \item \begin{tikzpicture}[>=stealth',shorten >=1pt,auto,node distance=1.5cm]
                    \tikzset{every state/.style={minimum size=0.2cm}}
                        
                        \node[initial,state]  (a1)      {1};
                        \node[state,accepting] (b1)  [above right of=a1]    {2};
                            \node[state,accepting] [below of=b1](c1)      {3};
                    \path[->]
                            (a1)  edge  node {b} (b1)
                            (a1)  edge[loop above]  node {a} (a1)
                            (b1)  edge[bend left]  node {a} (c1)
                            (b1)  edge[loop above]  node {b} (b1)
                            (c1)  edge  node {a} (a1)
                            (c1)  edge  node {b} (b1)
                            ;
                    \end{tikzpicture}

            \item \begin{tikzpicture}[>=stealth',shorten >=1pt,auto,node distance=1.5cm]
                \tikzset{every state/.style={minimum size=0.2cm}}
                            
                            \node[initial,state,accepting]  (a1)      {1};
                            \node[state] (b1)  [above right of=a1]    {2};
                                \node[state] [below of=b1](c1)      {3};
                        \path[->]
                                (a1)  edge [bend left] node {a} (b1)
                                (c1)  edge[loop right]  node {a,b} (c1)
                                (b1)  edge  node {a} (c1)
                                (b1)  edge  node {b} (a1)
                                (c1)  edge  node {b} (a1)
                                ;
                    \end{tikzpicture}

            \item \begin{tikzpicture}[>=stealth',shorten >=1pt,auto,node distance=1.5cm]
                    \tikzset{every state/.style={minimum size=0.2cm}}
                        
                        \node[initial,state]  (a1)      {1};
                        \node[state] (b1)  [right of=a1]    {2};
                            \node[state,accepting] [right of=b1](c1)      {3};
                    \path[->]
                            (a1)  edge  node {a} (b1)
                            (a1)  edge[loop above]  node {b} (a1)
                            (b1)  edge  node {b} (c1)
                            (b1)  edge[loop above]  node {a} (b1)
                            (c1)  edge[loop right]  node {a,b} (c1)
                            ;
                    \end{tikzpicture}
            
            \item \begin{tikzpicture}[>=stealth',shorten >=1pt,auto,node distance=1.5cm]
                    \tikzset{every state/.style={minimum size=0.2cm}}			
                        \node[initial,state]  (a1)      {1};
                        \node[state,accepting] (b1)  [right of=a1]    {2};
                    \path[->]
                            (a1)  edge[bend left]  node {b} (b1)
                            (a1)  edge[loop above]  node {a} (a1)
                            (b1)  edge[bend left]  node {a} (a1)
                            (b1)  edge[loop above]  node {b} (b1)
                            ;
                    \end{tikzpicture}

            \item \begin{tikzpicture}[>=stealth',shorten >=1pt,auto,node distance=1.5cm]
                    \tikzset{every state/.style={minimum size=0.2cm}}
                        
                        \node[initial,state]  (a1)      {1};
                        \node[state,accepting] (b1)  [above right of=a1]    {2};
                            \node[state] [below right of=a1](c1)      {3};
                    \path[->]
                            (a1)  edge  node {a} (b1)
                            (b1)  edge[loop above]  node {a,b} (b1)
                            (c1)  edge  node {b} (a1)
                            (c1)  edge  node {a,b} (b1)
                            ;
                    \end{tikzpicture}
                    
        \end{enumerate}

    % \end{multicols}

\end{problem}
    

\medskip\begin{problem}[Testování ekvivalence regulárních výrazů]

    \phantom{}

    \medskip

    \begin{enumerate}[(a)]\setlength\itemsep{12pt}
        \item Popište algoritmus na testování ekvivalence regulárních výrazů. 
        \item Aplikujte ho na následující dvojici regulárních výrazů:
        $$
        (a + b)(a + b)^* \qquad\text{a}\qquad a(a + b)^* + b(a + b)^*
        $$
    \end{enumerate}

\end{problem}
   

\medskip\begin{problem}[Jsou regulární výrazy regulární?]
    
    Mějme konečnou abecedu $\Sigma$. Je jazyk sestávající ze všech regulárních výrazů nad abecedou $\Sigma$ regulárním jazykem?
    
\end{problem}


\end{document}
