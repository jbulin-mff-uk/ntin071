\documentclass[a4paper,12pt]{amsart}

\usetheme[progressbar=frametitle]{metropolis}
\metroset{block=fill}

\subtitle{NTIN071 Automata and Grammars}
\author{Jakub Bulín (KTIML MFF UK)}

\date{Spring 2025\\ 
    \vspace{1in} 
    \begin{flushleft}
        \it \footnotesize * Adapted from the Czech-lecture slides by Marta Vomlelová with gratitude. The translation, some modifications, and all errors are mine.
    \end{flushleft}
}

%% packages

\usepackage{amsmath}
\usepackage{amssymb}
\usepackage{amsthm}
\usepackage{cancel}
\usepackage{color}
\usepackage{colortbl}
\usepackage{forest}
\usepackage[utf8x]{inputenc}
\usepackage{multicol}
\usepackage{multirow}

%% colors
\definecolor{Gray}{gray}{0.9}

%% TikZ
\usepackage{tikz}
    \usetikzlibrary{
        automata,
        arrows,
        backgrounds,
        decorations.pathmorphing,
        fit,
        positioning,
        shapes,
        shapes.geometric,
        tikzmark
    } 
    \tikzset{>=stealth',shorten >=1pt,auto,node distance=2cm}
    \tikzset{initial text={}}
    \tikzset{elliptic state/.style={draw,ellipse}}

%% amsthm
\theoremstyle{plain}
    \newtheorem*{algorithm}{Algorithm}    
    \newtheorem*{observation}{Observation}
    \newtheorem*{proposition}{Proposition}

\theoremstyle{remark}
    \newtheorem*{exercise}{Exercise}
    \newtheorem*{remark}{Remark}

%% macros
\DeclareMathOperator{\RegE}{RegE}
\DeclareMathOperator{\RL}{RL}

% Just for Lecture 2
\newcommand{\x}{$\times$}
\newcommand{\nx}{\ }


\begin{document}

\thispagestyle{empty}

\section*{NTIN071 A\&G: Cvičení 10 -- Převod mezi PDA a bezkontextovými gramatikami}

\medskip

\subsection*{Cíle výuky:} Po absolvování student umí

\begin{itemize}\setlength{\itemsep}{0pt}
    \item 
\end{itemize}

\section*{Příklady na cvičení}

\medskip\begin{problem} 

\end{problem}


\section*{K procvičení a k zamyšlení}


\medskip\begin{problem}

\end{problem}


\end{document}


\documentclass[a4paper,12pt]{amsart}

\usetheme[progressbar=frametitle]{metropolis}
\metroset{block=fill}

\subtitle{NTIN071 Automata and Grammars}
\author{Jakub Bulín (KTIML MFF UK)}

\date{Spring 2025\\ 
    \vspace{1in} 
    \begin{flushleft}
        \it \footnotesize * Adapted from the Czech-lecture slides by Marta Vomlelová with gratitude. The translation, some modifications, and all errors are mine.
    \end{flushleft}
}

%% packages

\usepackage{amsmath}
\usepackage{amssymb}
\usepackage{amsthm}
\usepackage{cancel}
\usepackage{color}
\usepackage{colortbl}
\usepackage{forest}
\usepackage[utf8x]{inputenc}
\usepackage{multicol}
\usepackage{multirow}

%% colors
\definecolor{Gray}{gray}{0.9}

%% TikZ
\usepackage{tikz}
    \usetikzlibrary{
        automata,
        arrows,
        backgrounds,
        decorations.pathmorphing,
        fit,
        positioning,
        shapes,
        shapes.geometric,
        tikzmark
    } 
    \tikzset{>=stealth',shorten >=1pt,auto,node distance=2cm}
    \tikzset{initial text={}}
    \tikzset{elliptic state/.style={draw,ellipse}}

%% amsthm
\theoremstyle{plain}
    \newtheorem*{algorithm}{Algorithm}    
    \newtheorem*{observation}{Observation}
    \newtheorem*{proposition}{Proposition}

\theoremstyle{remark}
    \newtheorem*{exercise}{Exercise}
    \newtheorem*{remark}{Remark}

%% macros
\DeclareMathOperator{\RegE}{RegE}
\DeclareMathOperator{\RL}{RL}

% Just for Lecture 2
\newcommand{\x}{$\times$}
\newcommand{\nx}{\ }


\begin{document}

\thispagestyle{empty}

\section*{NTIN071 A\&G: Cvičení 10 -- Zásobníkové automaty}

% po 10. přednášce
% jaro 2024

\medskip

\noindent\emph{Vyřešte nejprve 1a-h, 2ab, 3a (zbytek je na procvičení).}

\medskip


\medskip\begin{problem}[Konstrukce PDA]

    Navrhněte zásobníkové automaty pro následující jazyky. (Mohou přijímat koncovým stavem i prázdným zásobníkem, u některých sestrojte obojí.)

    \medskip

    \begin{enumerate}[(a)]\setlength\itemsep{9pt}
        \item $L=\{w\mid w\in\{0,1\}^*,|w|_1\geq 3\}$
        \item $L=\{ww^R\mid w\in \{0,1\}^*\}$
        \item $L=\{w\in\{(,)\}^*\mid w\text{ je korektní uzávorkování}\}$
        \item $L=\{w\in \{0,1\}^*\mid w=w^R\}$
        \item $L=\{a^ib^jc^k\mid i=j \text{ nebo } j=k\} $
        \item $L=\{a^ib^jc^k\mid i+j=k\}$
        \item $L=\{a^{2n}b^{3n}\mid n\geq 0\}$
        \item $L=\{w\in \{0,1\}^*\mid  |w|_0=|w|_1\} $
        \item $L=\{u2v\mid u,v\in \{0,1\}^*\text{ a }|u|\neq |v|\} $
        \item $L=\{w\in\{(,),[,]\}^*\mid w\text{ je korektní uzávorkování}\}$   
        %\item* $L=\{u2v\ \mid \ u,v\in \{0,1\}^*\text{ a } u\neq v\}$ (jde-li o bezkontextový jazyk)          
    \end{enumerate}

\end{problem}


\medskip\begin{problem}[Koncový stav vs. prázdný zásobník]

    Vybrané zásobníkové automaty sestrojené v předchozím příkladu převeďte z přijímání koncovým stavem na přijímání prázdným zásobníkem, a naopak. (Vyzkoušejte si obě konstrukce.)

\end{problem}


\medskip\begin{problem}[Z gramatiky na PDA]

    Pro danou gramatiku $G$ sestrojte zásobníkové automaty $Z_1,Z_2$ že $L(Z_1)=L(G)$ a $N(Z_2)=L(G)$.

    \medskip

    \begin{enumerate}[(a)]\setlength\itemsep{6pt}
        \item $G=(\{S,T,X\},\{a,b\},P,S)$
            \begin{align*}
        P=\{S&\rightarrow aTXb, \\
        T&\rightarrow XTS\mid \epsilon,\\ 
        X&\rightarrow a\mid b\}
        \end{align*}
        \item $G=(\{S,T,X\},\{(,),*,+,1\},P,S)$
            \begin{align*}
        P=\{S&\rightarrow S+T\mid T, \\
        T&\rightarrow T*X\mid X,\\ 
        X&\rightarrow 1\mid (S)\}
        \end{align*}
    \end{enumerate}

    Pro nějaké rozumně dlouhé slovo $w\in L(G)$ najděte levou derivaci a simulujte výpočet automatu $Z_2$.

\end{problem}
    

\medskip\begin{problem}[Z PDA na gramatiku]

    Vybrané (malé) zásobníkové automaty sestrojené na předchozím cvičení převeďte na bezkontextové gramatiky. Pro nějaké rozumně dlouhé slovo $w$ přijímané daným automatem najděte levou derivaci tohoto slova v zkonstruované gramatice.

\end{problem}


\section*{Bonus: Kontextové gramatiky}


\medskip\begin{problem}[Kontextová gramatika]
    
    Uvažme $G=(\{S,A,B,C\},\{a,b,c\},S,P)$, kde:
    \begin{align*}
        P=\{&S\rightarrow aSBC\mid aBC, B\rightarrow BBC,  C\rightarrow CC, CB\rightarrow BC,\\ 
        &aB\rightarrow ab, bB\rightarrow bb, bC\rightarrow bc, cC\rightarrow cc\}
    \end{align*}
    Jaký jazyk generuje? Je gramatika $G$ kontextová? Pokud ne, najděte ekvivalentní kontextovou gramatiku.
    
\end{problem}



\end{document}
