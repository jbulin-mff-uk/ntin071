\documentclass[a4paper,12pt]{amsart}

\usetheme[progressbar=frametitle]{metropolis}
\metroset{block=fill}

\subtitle{NTIN071 Automata and Grammars}
\author{Jakub Bulín (KTIML MFF UK)}

\date{Spring 2025\\ 
    \vspace{1in} 
    \begin{flushleft}
        \it \footnotesize * Adapted from the Czech-lecture slides by Marta Vomlelová with gratitude. The translation, some modifications, and all errors are mine.
    \end{flushleft}
}

%% packages

\usepackage{amsmath}
\usepackage{amssymb}
\usepackage{amsthm}
\usepackage{cancel}
\usepackage{color}
\usepackage{colortbl}
\usepackage{forest}
\usepackage[utf8x]{inputenc}
\usepackage{multicol}
\usepackage{multirow}

%% colors
\definecolor{Gray}{gray}{0.9}

%% TikZ
\usepackage{tikz}
    \usetikzlibrary{
        automata,
        arrows,
        backgrounds,
        decorations.pathmorphing,
        fit,
        positioning,
        shapes,
        shapes.geometric,
        tikzmark
    } 
    \tikzset{>=stealth',shorten >=1pt,auto,node distance=2cm}
    \tikzset{initial text={}}
    \tikzset{elliptic state/.style={draw,ellipse}}

%% amsthm
\theoremstyle{plain}
    \newtheorem*{algorithm}{Algorithm}    
    \newtheorem*{observation}{Observation}
    \newtheorem*{proposition}{Proposition}

\theoremstyle{remark}
    \newtheorem*{exercise}{Exercise}
    \newtheorem*{remark}{Remark}

%% macros
\DeclareMathOperator{\RegE}{RegE}
\DeclareMathOperator{\RL}{RL}

% Just for Lecture 2
\newcommand{\x}{$\times$}
\newcommand{\nx}{\ }


\begin{document}

\thispagestyle{empty}

\section*{NTIN071 A\&G: Cvičení 10 -- Vlastnosti bezkontextových jazyků, DPDA}

\medskip

\subsection*{Cíle výuky:} Po absolvování student umí

\begin{itemize}\setlength{\itemsep}{0pt}
    \item rozhodnout, zda jsou bezkontextové jazyky (CFL) uzavřené na různé operace
    \item uvést formální definici deterministického zásobníkového automatu (DPDA)
    \item vysvětlit rozdíl mezi přijímáním koncovým stavem a prázdným zásobníkem pro DPDA, tj. třídy jazyků $\mathrm{L_{DPDA}}$ a $\mathrm{N_{DPDA}}$
    \item vysvětlit a ilustrovat vztah $\mathrm{L_{DPDA}}$ a $\mathrm{N_{DPDA}}$ k ostatním třídám jazyků
    \item rozhodnout, zda jsou $\mathrm{L_{DPDA}}$ a $\mathrm{N_{DPDA}}$ uzavřené na různé operace
\end{itemize}

\section*{Příklady na cvičení}


\medskip\begin{problem}[Nepalidromy]
    Uvažme jazyk palindromů $L=\{w\in\{a,b\}^*\mid w=w^R\}$.
    \begin{enumerate}[(a)]
        \item Je $L$ bezkontextový?
        \item Je jeho doplněk $\overline{L}$ bezkontextový?
    \end{enumerate}
\end{problem}


\medskip\begin{problem}[Uzávěrové vlasnosti]
    
    Jsou (I) $\mathrm{CFL}$, (II) $\mathrm{L_{DPDA}}$, (III) $\mathrm{N_{DPDA}}$ uzavřené na následující operace? Dokažte nebo vyvraťte.
    
    \begin{multicols}{2}
        \begin{enumerate}[(a)]
            \item průnik
            \item doplněk
            \item průnik s regulárním jazykem
            \item homomorfismus
        \end{enumerate}
    \end{multicols}    

\end{problem}


\medskip\begin{problem}
    Ukažte, že:
    \begin{enumerate}[(a)]
        \item $\{a^nb^{n+1}\mid n\geq 0\}\in\mathrm{N_{DPDA}}$
        \item $\{a^nb^{n+k}\mid n\geq 0,k\in\{0,1\}\}\in\mathrm{L_{DPDA}}\setminus\mathrm{N_{DPDA}}$
        \item $\{a^nb^{n\cdot 2}\mid n\geq 0\}\in\mathrm{L_{DPDA}}$
        \item $\{a^nb^{n\cdot k}\mid n\geq 0,k\in\{1,2\}\}\in\mathrm{CFL}\setminus\mathrm{L_{DPDA}}$
    \end{enumerate}
\end{problem}


\section*{K procvičení a k zamyšlení}


\medskip\begin{problem}[Uzávěrové vlasnosti]
    
    Jsou (I) $\mathrm{CFL}$, (II) $\mathrm{L_{DPDA}}$, (III) $\mathrm{N_{DPDA}}$ uzavřené na následující operace? Dokažte nebo vyvraťte       

    \begin{multicols}{2}
        \begin{enumerate}[(a)]
            \item sjednocení
            \item zřetězení
            \item iterace
            \item reverz
            \item substituce regulárního jazyka
            \item inverzní homomorfismus
            \item pravá/levá derivace
            \item sjednocení s regulárním jazykem
        \end{enumerate}    
    \end{multicols}    

\end{problem}


\begin{problem}[Bonus: Kontextová gramatika]
    
    Mějme $G=(\{S,A,B,C\},\{a,b,c\},\mathcal P,S)$, kde:
    \begin{align*}
        \mathcal P=\{&S\rightarrow aSBC\mid aBC, B\rightarrow BBC,  C\rightarrow CC, CB\rightarrow BC,\\ 
        &aB\rightarrow ab, bB\rightarrow bb, bC\rightarrow bc, cC\rightarrow cc\}
    \end{align*}
    Jaký jazyk generuje? Je gramatika $G$ kontextová? Pokud ne, najděte ekvivalentní kontextovou gramatiku.
    
\end{problem}


\end{document}








\end{document}
