\documentclass[a4paper,12pt]{amsart}

\usepackage{a4wide}
\usepackage{amsmath}
\usepackage{amssymb}
\usepackage{amsthm}
\usepackage[czech]{babel}
\usepackage{bookmark}
\usepackage{enumerate}
\usepackage[T1]{fontenc}
\usepackage{hyperref}
\usepackage[utf8]{inputenc}
\usepackage{lmodern}
\usepackage{multicol}
\usepackage{tikz}
    \usetikzlibrary{automata, arrows, positioning}


\theoremstyle{definition}
    \newtheorem{problem}{Příklad}



\begin{document}

\thispagestyle{empty}

\section*{NTIN071 A\&G: Cvičení 3 -- Ekvivalentní a minimální reprezentace, testování vlastností, nedeterminismus, podmnožinová konstrukce}

\medskip

\subsection*{Cíle výuky:} Po absolvování student umí

\begin{itemize}\setlength{\itemsep}{0pt}
    \item definovat dosažitelné stavy, ekvivalenci stavů, redukovaný automat, homomorfismus automatů, NFA, $\epsilon$-NFA
    \item aplikovat algoritmy pro dosažitelné stavy a ekvivalenci stavů k redukci DFA
    \item navrhovat efektivní algoritmy pro testování základních vlastností automatů
    \item aplikovat konstrukci podmnožin k převodu NFA nebo $\epsilon$-NFA na DFA
\end{itemize}

\section*{Příklady na cvičení}

\medskip\begin{problem}[Ekvivalentní a minimální reprezentace]    
    
    Pro následující automaty:

    \begin{enumerate}[(a)]\setlength\itemsep{6pt}
        \item Najděte a odstraňte nedosažitelné stavy.
        \item Určete relaci ekvivalence (nerozlišitelnosti) stavů. (Navíc pro každou rozlišitelnou dvojici stavů najděte všechna nejkratší rozlišující slova.)
        \item Zkonstruujte jejich redukty.        
    \end{enumerate}
        
    \begin{multicols}{2}

        \centering
    
        \begin{tabular}{ r | c c }
            A & a & b \\ \hline
            $\to\ast$ 0 & 1 & 2 \\  
            1 & 3 & 0 \\
            2 & 4 & 5 \\
            3 & 0 & 2 \\
            4 & 2 & 5 \\
            5 & 0 & 3
        \end{tabular}
            
        \begin{tabular}{ r | c c }
            B  & a & b \\ \hline
            $\to\ast$ 0 & 0 & 5 \\  
            1 & 1 & 3 \\
            2 & 2 & 7 \\
            3 & 3 & 2 \\
            $\ast$ 4 & 6 & 1 \\
            5 & 5 & 1 \\
            $\ast$ 6 & 4 & 2 \\
            7 & 7 & 0
        \end{tabular}

    \end{multicols}

\end{problem}


\begin{problem}[Testování vlastností]

    Mějme konečné automaty $A,B$. Navrhněte algoritmy, které rozhodnou, zda platí daná vlasnost. (Umíte odhadnout jejich časovou složitost?)
    
    \begin{multicols}{4}
    
        \begin{enumerate}[(a)]\setlength\itemsep{6pt}
            \item $L(A)=\emptyset$,
            \item $L(A)=L(B)$,
            \item $L(A)\subseteq L(B)$,
            \item $L(A)$ je konečný.
        \end{enumerate}

    \end{multicols}

    Aplikujte algoritmus na automaty $A,B$ z předchozího problému.

    \smallskip

\end{problem}

\begin{problem}[Podmnožinová konstrukce]

    Pro daný nedeterministický automat s $\epsilon$-přechody sestrojte ekvivalentní redukovaný DFA.
    
    \begin{center}
        \begin{tabular}{ r | c c c }
            & a & b & $\epsilon$ \\ \hline
            $\to A$ & $\{E\}$ & $\{B\}$ & $\emptyset$ \\
            $B$ & $\emptyset$ & $\{C\}$ & $\{D\}$ \\
            $\to C$ & $\emptyset$ & $\{D\}$ & $\emptyset$ \\
            $\ast D$ & $\emptyset$ & $\emptyset$ & $\emptyset$ \\
            $E$ & $\{F\}$ & $\emptyset$ & $\{B,C\}$\\
            $F$ & $\{D\}$ & $\emptyset$ & $\emptyset$
        \end{tabular}
    \end{center}

\end{problem}


\section*{K procvičení a k zamyšlení}


\medskip\begin{problem}[Redukce DFA]    
    
    Zredukujte následující DFA:
    
    \begin{center}
        \begin{tabular}{ r | c c }
            C & a & b \\ \hline
            $\to$ 1 & 2 & 3 \\
            2 & 2 & 4 \\
            $\ast$ 3 & 3 & 5 \\
            4 & 2 & 7 \\
            $\ast$ 5 & 6 & 3 \\
            $\ast$ 6 & 6 & 6 \\
            7 & 7 & 4 \\
            8 & 2 & 3 \\
            9 & 9 & 4
        \end{tabular}
    \end{center}

\end{problem}


\medskip\begin{problem}[Podmnožinová konstrukce]

    Zkonstruujte ekvivalentní redukovaný DFA.    
    
    \begin{center}
        \begin{tabular}{ r | c c c }
            & a & b & $\epsilon$ \\ \hline
            $\ast A$ & $\{A,C\}$ & $\{B\}$ & $\emptyset$ \\
            $B$ & $\{B,D\}$ & $\emptyset$ & $\emptyset$ \\
            $\ast C$ & $\{E\}$ & $\{D\}$ & $\emptyset$ \\
            $D$ & $\{A\}$ & $\{C,D\}$ & $\emptyset$ \\
            $\to\ast E$ & $\emptyset$ & $\emptyset$ & $\{A,C\}$
        \end{tabular}
    \end{center}

\end{problem}


\medskip\begin{problem}[Homomorfismus automatů]

    Najděte DFA $A,B$ takové, že:  

    \begin{enumerate}[(a)]\setlength\itemsep{6pt}
        \item Jsou oba redukované, a nejsou izomorfní. 
        \item $A$ je homomorfní na $B$, ale nejsou izomorfní.
        \item Jsou ekvivalentní, ale ne izomorfní.
        \item Jsou oba homomorfní na, ale ne izomorfní s $C$, a zároveň $A$ není homomorfní na $B$ ani $B$ na $A$.
        $$
        C=(\{p,q\},\{0,1\},\{((p,0),q),((p,1),p),((q,0),p),((q,1),q)\},p,\{q\})
        $$
    \end{enumerate}

\end{problem}


\medskip\begin{problem}[Regulární? Zredukuj]
    
   Uvažme jazyk $L$ nad abecedou $\{a,b\}$ sestávající ze všech slov, která neobsahují trojici po sobě jdoucích stejných písmen. Rozhodněte, zda je $L$ regulární. Pokud ano, najděte regulární DFA, který ho rozpoznává.

\end{problem}


\medskip\begin{problem}[Podmnožinová konstrukce a redukt]
    
    Je výsledek podmnožinové konstrukce (kde generujeme jen dosažitelné stavy) nutně reukovaný automat? Dokažte, nebo vyvraťte.

\end{problem}


\end{document}
