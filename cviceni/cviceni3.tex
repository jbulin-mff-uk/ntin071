\documentclass[a4paper,12pt]{amsart}

\usepackage{a4wide}
\usepackage{amsmath}
\usepackage{amssymb}
\usepackage{amsthm}
\usepackage[czech]{babel}
\usepackage{bookmark}
\usepackage{enumerate}
\usepackage[T1]{fontenc}
\usepackage{hyperref}
\usepackage[utf8]{inputenc}
\usepackage{lmodern}
\usepackage{multicol}
\usepackage{tikz}
    \usetikzlibrary{automata, arrows, positioning}


\theoremstyle{definition}
    \newtheorem{problem}{Příklad}



\begin{document}

%\thispagestyle{empty}

\section*{NTIN071 A\&G: Cvičení 3 -- Myhill-Nerodeova věta, ekvivalentní a minimální reprezentace, testování vlastností}

% po 2. přednášce
% jaro 2024

\medskip

\noindent\emph{Vyřešte nejprve 1, 2, 3a-f, 4abc pro A\&B, 5 (zbytek je na procvičení).}

\medskip




\vspace{-12pt}

\medskip\begin{problem}[Ekvivalentní a minimální reprezentace]    
    
    Pro následující automaty:

    \begin{enumerate}[(a)]\setlength\itemsep{6pt}
        \item Najděte a odstraňte nedosažitelné stavy.
        \item Určete relaci ekvivalence (nerozlišitelnosti) stavů. (Navíc pro každou rozlišitelnou dvojici stavů najděte všechna nejkratší rozlišující slova.)
        \item Zkonstruujte jejich redukty.
        \item Jsou některé dva z automatů ekvivalentní? Použijte algoritmus z přednášky.
    \end{enumerate}
    
    \begin{multicols}{3}\small\centering
    
        \begin{tabular}{ r | c c }
            A & a & b \\ \hline
            $\to\ast$ 0 & 1 & 2 \\  
            1 & 3 & 0 \\
            2 & 4 & 5 \\
            3 & 0 & 2 \\
            4 & 2 & 5 \\
            5 & 0 & 3
        \end{tabular}
            
        \begin{tabular}{ r | c c }
            B & a & b \\ \hline
            $\to\ast$ 0 & 0 & 5 \\  
            1 & 1 & 3 \\
            2 & 2 & 7 \\
            3 & 3 & 2 \\
            $\ast$ 4 & 6 & 1 \\
            5 & 5 & 1 \\
            $\ast$ 6 & 4 & 2 \\
            7 & 7 & 0
        \end{tabular}
                
        \begin{tabular}{ r | c c }
            C & a & b \\ \hline
            $\to$ 1 & 2 & 3 \\
            2 & 2 & 4 \\
            $\ast$ 3 & 3 & 5 \\
            4 & 2 & 7 \\
            $\ast$ 5 & 6 & 3 \\
            $\ast$ 6 & 6 & 6 \\
            7 & 7 & 4 \\
            8 & 2 & 3 \\
            9 & 9 & 4
        \end{tabular}

    \end{multicols}

\end{problem}


\medskip\begin{problem}[Testování vlastností]

    Mějme konečné automaty $A,B$. Navrhněte algoritmy, které rozhodnou, zda platí následující vlastnosti. (Umíte odhadnout jejich časovou složitost?)

    \medskip
    
    \begin{multicols}{2}
    
        \begin{enumerate}[(a)]\setlength\itemsep{12pt}
            \item $L(A)=\emptyset$,
            \item $L(A)=L(B)$,
            \item $L(A)\subseteq L(B)$,
            \item $L(A)$ je konečný.
        \end{enumerate}

    \end{multicols}

\end{problem}


\medskip\begin{problem}[Homomorfismus automatů]

    Najděte DFA $A,B$ takové, že:  

    \begin{enumerate}[(a)]\setlength\itemsep{12pt}
        \item Jsou oba redukované, a nejsou izomorfní. 
        \item $A$ je homomorfní na $B$, ale nejsou izomorfní.
        \item Jsou ekvivalentní, ale ne izomorfní.
        \item Jsou oba homomorfní na, ale ne izomorfní s $C$, a zároveň $A$ není homomorfní na $B$ ani $B$ na $A$.
        $$
        C=(\{p,q\},\{0,1\},\{((p,0),q),((p,1),p),((q,0),p),((q,1),q)\},p,\{q\})
        $$
    \end{enumerate}

\end{problem}


\medskip\begin{problem}[Regulární? Zredukuj]
    
   Uvažme jazyk $L$ nad abecedou $\{a,b\}$ sestávající ze všech slov, která neobsahují trojici po sobě jdoucích stejných písmen. Rozhodněte, zda je $L$ regulární. Pokud ano, najděte regulární DFA, který ho rozpoznává.

\end{problem}


\end{document}
