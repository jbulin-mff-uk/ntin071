\documentclass[a4paper,12pt]{amsart}

\usetheme[progressbar=frametitle]{metropolis}
\metroset{block=fill}

\subtitle{NTIN071 Automata and Grammars}
\author{Jakub Bulín (KTIML MFF UK)}

\date{Spring 2025\\ 
    \vspace{1in} 
    \begin{flushleft}
        \it \footnotesize * Adapted from the Czech-lecture slides by Marta Vomlelová with gratitude. The translation, some modifications, and all errors are mine.
    \end{flushleft}
}

%% packages

\usepackage{amsmath}
\usepackage{amssymb}
\usepackage{amsthm}
\usepackage{cancel}
\usepackage{color}
\usepackage{colortbl}
\usepackage{forest}
\usepackage[utf8x]{inputenc}
\usepackage{multicol}
\usepackage{multirow}

%% colors
\definecolor{Gray}{gray}{0.9}

%% TikZ
\usepackage{tikz}
    \usetikzlibrary{
        automata,
        arrows,
        backgrounds,
        decorations.pathmorphing,
        fit,
        positioning,
        shapes,
        shapes.geometric,
        tikzmark
    } 
    \tikzset{>=stealth',shorten >=1pt,auto,node distance=2cm}
    \tikzset{initial text={}}
    \tikzset{elliptic state/.style={draw,ellipse}}

%% amsthm
\theoremstyle{plain}
    \newtheorem*{algorithm}{Algorithm}    
    \newtheorem*{observation}{Observation}
    \newtheorem*{proposition}{Proposition}

\theoremstyle{remark}
    \newtheorem*{exercise}{Exercise}
    \newtheorem*{remark}{Remark}

%% macros
\DeclareMathOperator{\RegE}{RegE}
\DeclareMathOperator{\RL}{RL}

% Just for Lecture 2
\newcommand{\x}{$\times$}
\newcommand{\nx}{\ }


\begin{document}

%\thispagestyle{empty}

\section*{NTIN071 A\&G: Cvičení 3 -- Myhill-Nerodeova věta, ekvivalentní a minimální reprezentace, testování vlastností}

% po 2. přednášce
% jaro 2024

\medskip

\noindent\emph{Vyřešte nejprve 1, 2, 3a-f, 4abc pro A\&B, 5 (zbytek je na procvičení).}

\medskip


\medskip\begin{problem}[Ekvivalence na slovech]

    Uveďte příklad ekvivalence $\sim$ na $\Sigma^*$, která:

    \begin{enumerate}[(a)]\setlength\itemsep{6pt}
        \item je pravá a levá kongruence
        \item je pravá, ale ne levá kongruence
        \item je konečného indexudl
    \end{enumerate}

\end{problem}


\medskip\begin{problem}[Myhill--Nerodeova věta: formulace]
    
    Zformuluje Myhill--Nerodeovu větu a připomeňte si myšlenku důkazu (bez nahlížení do poznámek z přednášky).

\end{problem}


\medskip\begin{problem}[Myhill--Nerodeova věta: aplikace]

    Pomocí Myhill--Nerodeovy věty dokažte nebo vyvraťte, že je jazyk regulární.

    \begin{multicols}{2}

        \begin{enumerate}[(a)]\setlength\itemsep{12pt}
            \item $L=\{aa, ab, ba\}$
            \item $L=\{a^ib^j\ \mid\ i\leq j\}$
            \item $L=\{a^ib^j\ \mid\ i\geq j\}$
            \item $L_k=\{a^ib^j\ \mid\ i\leq j\leq k\}$ pro dané $k\in\mathbb N$
            \item $L=\{a^{2^i}\ \mid\ i\geq 0\}$
            \item $L=\{ww^R\ \mid \ w\in\Sigma^*\}$%, kde $w^R$ je $w$ pozpátku
            \item $L=\{a^ib^{i+j}a^j\ \mid\ i,j\geq 0\}$
            \item $L=\{ww\ \mid \ w\in\Sigma^*\}$
        \end{enumerate}
        
    \end{multicols}    

\end{problem}

\vspace{-12pt}

\medskip\begin{problem}[Ekvivalentní a minimální reprezentace]    
    
    Pro následující automaty:

    \begin{enumerate}[(a)]\setlength\itemsep{6pt}
        \item Najděte a odstraňte nedosažitelné stavy.
        \item Určete relaci ekvivalence (nerozlišitelnosti) stavů. (Navíc pro každou rozlišitelnou dvojici stavů najděte všechna nejkratší rozlišující slova.)
        \item Zkonstruujte jejich redukty.
        \item Jsou některé dva z automatů ekvivalentní? Použijte algoritmus z přednášky.
    \end{enumerate}
    
    \begin{multicols}{3}\small\centering
    
        \begin{tabular}{ r | c c }
            A & a & b \\ \hline
            $\to\ast$ 0 & 1 & 2 \\  
            1 & 3 & 0 \\
            2 & 4 & 5 \\
            3 & 0 & 2 \\
            4 & 2 & 5 \\
            5 & 0 & 3
        \end{tabular}
            
        \begin{tabular}{ r | c c }
            B & a & b \\ \hline
            $\to\ast$ 0 & 0 & 5 \\  
            1 & 1 & 3 \\
            2 & 2 & 7 \\
            3 & 3 & 2 \\
            $\ast$ 4 & 6 & 1 \\
            5 & 5 & 1 \\
            $\ast$ 6 & 4 & 2 \\
            7 & 7 & 0
        \end{tabular}
                
        \begin{tabular}{ r | c c }
            C & a & b \\ \hline
            $\to$ 1 & 2 & 3 \\
            2 & 2 & 4 \\
            $\ast$ 3 & 3 & 5 \\
            4 & 2 & 7 \\
            $\ast$ 5 & 6 & 3 \\
            $\ast$ 6 & 6 & 6 \\
            7 & 7 & 4 \\
            8 & 2 & 3 \\
            9 & 9 & 4
        \end{tabular}

    \end{multicols}

\end{problem}


\medskip\begin{problem}[Testování vlastností]

    Mějme konečné automaty $A,B$. Navrhněte algoritmy, které rozhodnou, zda platí následující vlastnosti. (Umíte odhadnout jejich časovou složitost?)

    \medskip
    
    \begin{multicols}{2}
    
        \begin{enumerate}[(a)]\setlength\itemsep{12pt}
            \item $L(A)=\emptyset$,
            \item $L(A)=L(B)$,
            \item $L(A)\subseteq L(B)$,
            \item $L(A)$ je konečný.
        \end{enumerate}

    \end{multicols}

\end{problem}


\medskip\begin{problem}[Homomorfismus automatů]

    Najděte DFA $A,B$ takové, že:  

    \begin{enumerate}[(a)]\setlength\itemsep{12pt}
        \item Jsou oba redukované, a nejsou izomorfní. 
        \item $A$ je homomorfní na $B$, ale nejsou izomorfní.
        \item Jsou ekvivalentní, ale ne izomorfní.
        \item Jsou oba homomorfní na, ale ne izomorfní s $C$, a zároveň $A$ není homomorfní na $B$ ani $B$ na $A$.
        $$
        C=(\{p,q\},\{0,1\},\{((p,0),q),((p,1),p),((q,0),p),((q,1),q)\},p,\{q\})
        $$
    \end{enumerate}

\end{problem}


\medskip\begin{problem}[Regulární? Zredukuj]
    
   Uvažme jazyk $L$ nad abecedou $\{a,b\}$ sestávající ze všech slov, která neobsahují trojici po sobě jdoucích stejných písmen. Rozhodněte, zda je $L$ regulární. Pokud ano, najděte regulární DFA, který ho rozpoznává.

\end{problem}


\end{document}
