\documentclass[a4paper,12pt]{article}
\usepackage{a4wide}
\usepackage{amsmath,amsthm}
\usepackage[utf8]{inputenc}
\usepackage[czech]{babel}
\usepackage{enumerate}
\usepackage{tikz}
    \usetikzlibrary{arrows,automata,positioning}

\theoremstyle{definition}
    \newtheorem{problem}{Problem}

\begin{document}

\begin{center}
    \large{NTIN071 A\&G: Domácí úkol}    
\end{center}


{\it Za každý příklad lze získat maximálně 5 bodů, celkem 20 bodů. Řešení musí být zcela vaší vlastní prací. Svůj postup, zkonstruované objekty, a důkazy popište dostatečně podrobně a formálně.}

\bigskip

\begin{problem}

    Vyřešte následující vzorové zadání zápočtového testu. 

    \begin{enumerate}[(a)]
        \item Sestrojte bezkontextovou gramatiku generující jazyk $L=\{a^nb^ka^{3n}\mid n,k\geq 0\}$. Napište derivaci pro slovo $w=abbaaa$.
        \item Převeďte gramatiku z předchozího příkladu do Chomského normální formy.
        \item Dokažte, že jazyk $L=\{a^{n^5}\mid n\geq 0\}$ není regulární.
        \item Sestrojte zásobníkový automat přijímající, prázdným zásobníkem, jazyk $L=\{w\in\{0,1\}^*\mid |w|_0\geq |w|_1 + 1\}$. Napište posloupnost konfigurací pro slovo $w=10001$.
        \item Dokažte, že jazyk $L=\{0^i1^j2^k3^\ell\mid i=j=k\text{ nebo }\ell=0\}$ není bezkontextový.
        \item Sestrojte deterministický konečný automat, který přijímá právě ta slova nad abecedou $\{0,1\}$, která končí posloupností $010$.
    \end{enumerate}

\end{problem}


\bigskip


\begin{problem}

    Mějme dva regulární jazyky $L,M$ nad abecedou $\{0,1\}$. Předpokládejme, že máme DFA $A,B$ takové, že $L=L(A)$ a $M=L(B)$. Definujme jazyk $K$ takto:
    $$K=\{uvw\mid uw\in L, v\in M\}$$
    
    \smallskip
    \begin{enumerate}[(a)]    
        \item Sestrojte $\epsilon$-NFA $C$ rozpoznávající jazyk $K$. 
        (Konstrukce musí být formálně popsaná, ale nemusíte zdůvodňovat, proč $C$ přijímá právě jazyk $K$.)
    \end{enumerate}
    Ukažte vaši konstrukci také na konkrétním příkladě: Nechť $L=\{w\in\{0,1\}^*\mid |w|\text{ je dělitelná 3}\}$ a $M=\{0^{2n+1}11\mid n\geq 0\}$.
    
    \smallskip
    \begin{enumerate}[(b)]
        \item Nakreslete stavové diagramy nějakých DFA $A,B$ přijímajících $L,M$ (resp.). 
        \item[(c)] Nakreslete stavový diagram $\epsilon$-NFA $C$ vytvořeného vaší konstrukcí.  
        \item[(d)] Napište posloupnost stavů, ve kterých se $C$ ocitá v průběhu nějakého přijímajícího výpočtu slova $w=001100$.
    \end{enumerate}

\end{problem}


% \bigskip


% \begin{problem}

%     Mějme následující regulární výrazy nad abecedou $\Sigma=\{a,b\}$:
%     \begin{align*}
%         R_1&=((a + b)(a + b)(a + b))^*\\
%         R_2&=(a+b)^*a
%     \end{align*}
%     \begin{enumerate}[(a)]
%         \item Sestrojte \emph{deterministický} konečný automat $A$ rozpoznávající průnik jazyků popsaných výrazy $R_1$ a $R_2$, tj. takový, že $L(A)= L(R_1) \cap L(R_2)$.
%         \item Pomocí algoritmu eliminace stavů převeďte automat $A$ na regulární výraz.
%     \end{enumerate}

% \end{problem}


\bigskip


\begin{problem}
    
    Uvažte následující jazyk nad abecedou $\{0,1,\#\}$:
    $$
    L = \{w \# s^R \mid w,s\in\{0,1\}^*\text{ a slovo $s$ je podslovem slova $w$}\}
    $$
    
    (Poznámka: $s^R$ označuje slovo $s$ napsané pozpátku; podslovo je souvislý podřetězec, vč. prázdného a celého slova.)

    \begin{enumerate}[(a)]  
        \item Dokažte, že $L$ není regulární.    
        \item Sestrojte nějakou bezkontextovou gramatiku $G$ generující jazyk $L$.
        \item Napište derivaci slova $1001\#00$.
        \item Převeďte gramatiku $G$ z části (a) na zásobníkový automat přijímající prázdným zásobníkem.
        \item Napište posloupnost konfigurací pro slovo $1001\#00$.
    \end{enumerate}

\end{problem}


\bigskip


\begin{problem}
    
    Uvažme následující jazyky nad abecedou $\Sigma = \{0,1\}$:
    \begin{itemize}
        \item $L_1$ je jazyk generovaný bezkontextovou gramatikou $G = (\{S\}, \Sigma, \mathcal P, S)$ s množinou pravidel $\mathcal P = \{\,S \rightarrow SS \mid 0S1 \mid \epsilon\,\}$ (kde $\epsilon$ značí prázdné slovo),
        \item $L_2$ je jazyk rozpoznávaný deterministickým konečným automatem 
        $$
        A=(\{q_0,q_1,q_2,q_3\},\Sigma,\delta_A,q_0,\{q_0,q_1,q_2,q_3\}),
        $$ jehož přechodová funkce $\delta_A$ je dána následujícím stavovým diagramem:
        \begin{center}
            \begin{tikzpicture}[>=stealth',shorten >=1pt,auto,node distance=2cm]
                \node[initial,state,accepting] (q0)      {$q_0$};
                \node[state,accepting] (q1) [right of=q0]     {$q_1$};
                \node[state,accepting] (q2) [right of=q1]     {$q_2$};
                \node[state,accepting] (q3) [right of=q2]     {$q_3$};                               
                \path[->]
                    (q0) edge[loop above] node {$1$} (q0)
                    (q0) edge node {$0$} (q1)
                    (q1) edge[loop above] node {$0$} (q1)
                    (q1) edge node {$1$} (q2)
                    (q2) edge node {$0$} (q3)
                    (q2) edge[bend left] node {$1$} (q0)
                    (q3) edge[bend left] node {$0$} (q1)
                ;
            \end{tikzpicture}
        \end{center}
        
    \end{itemize}
    Sestrojte zásobníkový automat rozpoznávající průnik $L=L_1\cap L_2$ jazyků $L_1$ a $L_2$.

\end{problem}


\end{document}
