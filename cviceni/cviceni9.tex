\documentclass[a4paper,12pt]{amsart}

\usepackage{a4wide}
\usepackage{amsmath}
\usepackage{amssymb}
\usepackage{amsthm}
\usepackage[czech]{babel}
\usepackage{bookmark}
\usepackage{enumerate}
\usepackage[T1]{fontenc}
\usepackage{hyperref}
\usepackage[utf8]{inputenc}
\usepackage{lmodern}
\usepackage{multicol}
\usepackage{tikz}
    \usetikzlibrary{automata, arrows, positioning}


\theoremstyle{definition}
    \newtheorem{problem}{Příklad}



\begin{document}

\thispagestyle{empty}

\section*{NTIN071 A\&G: Cvičení 9 -- Zásobníkové automaty}

\medskip

\subsection*{Cíle výuky:} Po absolvování student umí

\begin{itemize}\setlength{\itemsep}{0pt}
    \item uvést formální definici PDA, přijímání prázdným zásobníkem i koncovým stavem
    \item zkonstruovat zásobníkový automat pro daný jazyk
    \item převádět mezi přijímáním prázdným zásobníkem a koncovým stavem
    \item převést bezkontextovou gramatiku na zásobníkový automat
    \item převést zásobníkový automat na bezkontextovou gramatiku
\end{itemize}

\section*{Příklady na cvičení}

\medskip\begin{problem}[Konstrukce PDA]\label{problem:construct-pda}

    Navrhněte zásobníkový automat přijímající daný jazyk. U (a), (b), (c) přijímejte prázdným zásobníkem, u (d), (e), (f) přijímejte koncovým stavem.

    \medskip

    \begin{enumerate}[(a)]\setlength\itemsep{6pt}
        \item $L=\{ww^R\mid w\in \{0,1\}^*\}$
        \item $L=\{w\in\{(,)\}^*\mid w\text{ je korektní uzávorkování}\}$
        \item $L=\{a^ib^jc^k\mid i=j \text{ nebo } j=k\} $
        \item $L=\{a^{2n}b^{3n}\mid n\geq 0\}$
        \item $L=\{w\in \{0,1\}^*\mid  |w|_0=|w|_1\} $
        \item $L=\{u2v\mid u,v\in \{0,1\}^*\text{ a }|u|\neq |v|\} $
    \end{enumerate}

\end{problem}


\smallskip\begin{problem}[Koncový stav vs. prázdný zásobník]

    Vybrané zásobníkové automaty sestrojené v předchozím příkladu převeďte z přijímání koncovým stavem na přijímání prázdným zásobníkem, a naopak. (Vyzkoušejte si obě konstrukce.)

\end{problem}


\smallskip\begin{problem}[Převod CFG na PDA]

    Pro danou gramatiku $G$ sestrojte PDA $A$ takový, že $L(G)=N(A)$. Dále pro dané slovo $w\in L(G)$ najděte levou derivaci z $G$ a proveďte simulaci výpočtu $A$ (napište přijímající posloupnost konfigurací).

    \begin{enumerate}[(a)]\setlength\itemsep{6pt}
        \item $G=(\{S,T,X\},\{a,b\},\mathcal P,S)$ s následujícími pravidly, $w=aaaabbb$
        \begin{align*}
            \mathcal P=\{S&\rightarrow aTXb, \\
            T&\rightarrow XTS\mid \epsilon,\\ 
            X&\rightarrow a\mid b\}
        \end{align*}
        \item $G=(\{S,T,X\},\{(,),*,+,1\},P,S)$ s následujícími pravidly, $w=1+1*(1+1)$
        \begin{align*}
            P=\{S&\rightarrow S+T\mid T, \\
            T&\rightarrow T*X\mid X,\\ 
            X&\rightarrow 1\mid (S)\}
        \end{align*}
    \end{enumerate}

\end{problem}
    

\begin{problem}[Převod PDA na CFG]

    Zásobníkové automaty z Příkladu 1 (a), (b) převeďte na bezkontextové gramatiky. Pro nějaké rozumně dlouhé slovo $w$ přijímané daným automatem najděte levou derivaci tohoto slova v zkonstruované gramatice.

\end{problem}


\section*{K procvičení a k zamyšlení}


\medskip\begin{problem}[Bonus: Kontextová gramatika]
    
    Uvažme $G=(\{S,A,B,C\},\{a,b,c\},S,P)$, kde:
    \begin{align*}
        P=\{&S\rightarrow aSBC\mid aBC, B\rightarrow BBC,  C\rightarrow CC, CB\rightarrow BC,\\ 
        &aB\rightarrow ab, bB\rightarrow bb, bC\rightarrow bc, cC\rightarrow cc\}
    \end{align*}
    Jaký jazyk generuje? Je gramatika $G$ kontextová? Pokud ne, najděte ekvivalentní kontextovou gramatiku.
    
\end{problem}


\medskip\begin{problem}[Konstrukce PDA]

    Navrhněte zásobníkové automaty pro následující jazyky. (Mohou přijímat koncovým stavem i prázdným zásobníkem, u některých sestrojte obojí, u některých si vyzkoušejte převod mezi těmito dvěma způsoby přijímání.)

    \medskip

    \begin{enumerate}[(a)]\setlength\itemsep{9pt}
        \item $L=\{w\mid w\in\{0,1\}^*,|w|_1\geq 3\}$        
        \item $L=\{w\in \{0,1\}^*\mid w=w^R\}$        
        \item $L=\{a^ib^jc^k\mid i+j=k\}$        
        \item $L=\{w\in\{(,),[,]\}^*\mid w\text{ je korektní uzávorkování}\}$
    \end{enumerate}

\end{problem}


\end{document}




