\documentclass[a4paper,12pt]{amsart}

\usetheme[progressbar=frametitle]{metropolis}
\metroset{block=fill}

\subtitle{NTIN071 Automata and Grammars}
\author{Jakub Bulín (KTIML MFF UK)}

\date{Spring 2025\\ 
    \vspace{1in} 
    \begin{flushleft}
        \it \footnotesize * Adapted from the Czech-lecture slides by Marta Vomlelová with gratitude. The translation, some modifications, and all errors are mine.
    \end{flushleft}
}

%% packages

\usepackage{amsmath}
\usepackage{amssymb}
\usepackage{amsthm}
\usepackage{cancel}
\usepackage{color}
\usepackage{colortbl}
\usepackage{forest}
\usepackage[utf8x]{inputenc}
\usepackage{multicol}
\usepackage{multirow}

%% colors
\definecolor{Gray}{gray}{0.9}

%% TikZ
\usepackage{tikz}
    \usetikzlibrary{
        automata,
        arrows,
        backgrounds,
        decorations.pathmorphing,
        fit,
        positioning,
        shapes,
        shapes.geometric,
        tikzmark
    } 
    \tikzset{>=stealth',shorten >=1pt,auto,node distance=2cm}
    \tikzset{initial text={}}
    \tikzset{elliptic state/.style={draw,ellipse}}

%% amsthm
\theoremstyle{plain}
    \newtheorem*{algorithm}{Algorithm}    
    \newtheorem*{observation}{Observation}
    \newtheorem*{proposition}{Proposition}

\theoremstyle{remark}
    \newtheorem*{exercise}{Exercise}
    \newtheorem*{remark}{Remark}

%% macros
\DeclareMathOperator{\RegE}{RegE}
\DeclareMathOperator{\RL}{RL}

% Just for Lecture 2
\newcommand{\x}{$\times$}
\newcommand{\nx}{\ }


\begin{document}

\thispagestyle{empty}

\section*{NTIN071 A\&G: Cvičení 2 -- Pumping Lemma, Myhill-Nerodeova věta}

\subsection*{Cíle výuky:} Po absolvování student umí

\begin{itemize}\setlength{\itemsep}{0pt}
    \item formulovat a dokázat Pumping Lemma
    \item použít Pumping Lemma k důkazu neregularity daného jazyka
    \item formulovat a dokázat Myhill-Nerodovu větu
    \item použít Myhill-Nerodovu větu k důkazu regularity, ke konstrukci DFA
    \item použít Myhill-Nerodovu větu k důkazu neregularity
\end{itemize}


\section*{Příklady na cvičení}


\medskip\begin{problem}[Pumping Lemma: formulace]

    \begin{enumerate}[(a)]\setlength\itemsep{6pt}
        \item Zformulujte Pumping Lemma pro regulární jazyky (bez nahlížení do poznámek z přednášky).
        \item Jak souvisí $n$ z lemmatu s automatem rozpoznávajícím daný jazyk?
        \item Dokažte je (bez nahlížení do poznámek z přednášky).
        \item Demonstrujte pumpování na jazyce $L=\{w\in\{a,b\}^* \,\mid\,\text{$w$ obsahuje $abba$ jako podslovo}\}$.
    \end{enumerate}

\end{problem}


\medskip\begin{problem}[Pumping Lemma: aplikace]

    Dokažte pomocí Pumping Lemmatu, že následující jazyky nejsou regulární. (Jazyky jsou nad abecedou $\Sigma=\{a,b\}$.)

    \medskip
      
    \begin{enumerate}[(a)]\setlength\itemsep{6pt}        
        \item $L=\{a^ib^j\ \mid\ i\geq j\}$       
        \item $L=\{a^{i^2}\ \mid\ i\geq 0\}$        
        \item $L=\{a^ib^{i+j}a^j\ \mid\ i,j\geq 0\}$
        \item $L=\{ww^R\ \mid \ w\in\Sigma^*\}$, kde $w^R$ je  $w$ napsané pozpátku
    \end{enumerate}
  
\end{problem}


\medskip\begin{problem}[Myhill--Nerodeova věta: formulace]
    
    \begin{enumerate}[(a)]
        \item Zformuluje Myhill--Nerodeovu větu a připomeňte si myšlenku důkazu (bez nahlížení do poznámek z přednášky).
        \item Ukažte, že když zapomeneme libovolnou z podmínek na ekvivalenci $\sim$, výsledné tvrzení neplatí. 
    \end{enumerate}    

\end{problem}


\medskip\begin{problem}[Myhill--Nerodeova věta: aplikace]

    Pomocí Myhill--Nerodeovy věty dokažte nebo vyvraťte, že je jazyk regulární.

    \begin{enumerate}[(a)]\setlength\itemsep{6pt}
        \item $L=\{aa, ab, ba\}$        
        \item $L=\{a^ib^j\ \mid\ i\geq j\}$        
        \item $L=\{a^{i^2}\ \mid\ i\geq 0\}$ 
        \item $L=\{ww^R\ \mid \ w\in\Sigma^*\}$, kde $w^R$ je  $w$ napsané pozpátku
        \item $L=\{a^ib^{i+j}a^j\ \mid\ i,j\geq 0\}$        
    \end{enumerate}

\end{problem}


\section*{K procvičení a k zamyšlení}


\medskip\begin{problem}
    
    Dokažte pomocí Pumping Lemmatu, že následující jazyky nejsou regulární. (Jazyky jsou nad abecedou $\Sigma=\{a,b\}$.)
    
    \medskip
      
    \begin{enumerate}[(a)]\setlength\itemsep{6pt}
        \item $L=\{a^ib^j\ \mid\ i\leq j\}$        
        \item $L=\{a^{2^i}\ \mid\ i\geq 0\}$
        \item $L=\{ww\ \mid \ w\in\Sigma^*\}$
    \end{enumerate}
      
\end{problem}


\medskip\begin{problem}

    Pomocí Myhill--Nerodeovy věty dokažte nebo vyvraťte, že je jazyk regulární.
    \begin{enumerate}[(a)]\setlength\itemsep{6pt}
        \item $L=\{a^ib^j\ \mid\ i\leq j\}$
        \item $L_k=\{a^ib^j\ \mid\ i\leq j\leq k\}$ pro pevně dané $k\in\mathbb N$
        \item $L=\{a^{2^i}\ \mid\ i\geq 0\}$
        \item $L=\{ww\ \mid \ w\in\Sigma^*\}$
    \end{enumerate}

\end{problem}


\medskip\begin{problem}[Pumping Lemma: zobecnění]

    \begin{enumerate}[(a)]\setlength\itemsep{6pt}
        \item Můžeme podmínku $|uv|\leq n$ v Pumping Lemmatu nahradit za $|vw|\leq n$, tedy \emph{iterovat blízko konce}? Dokažte nebo vyvraťte.
        \item Můžeme iterovat blízko předem zvoleného místa ve slově? Jak zformulovat (a dokázat) takové zesílení?
    \end{enumerate}

\end{problem}


\medskip\begin{problem}[Ekvivalence na slovech]

    Uveďte příklad ekvivalence $\sim$ na $\Sigma^*$, která:

    \begin{enumerate}[(a)]\setlength\itemsep{6pt}
        \item je pravá a levá kongruence
        \item je pravá, ale ne levá kongruence
        \item je konečného indexu
    \end{enumerate}

\end{problem}


\end{document}
