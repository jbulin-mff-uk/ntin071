\documentclass[a4paper,12pt]{amsart}

\usepackage{a4wide}
\usepackage{amsmath}
\usepackage{amssymb}
\usepackage{amsthm}
\usepackage[czech]{babel}
\usepackage{bookmark}
\usepackage{enumerate}
\usepackage[T1]{fontenc}
\usepackage{hyperref}
\usepackage[utf8]{inputenc}
\usepackage{lmodern}
\usepackage{multicol}
\usepackage{tikz}
    \usetikzlibrary{automata, arrows, positioning}


\theoremstyle{definition}
    \newtheorem{problem}{Příklad}



\begin{document}

\thispagestyle{empty}

\section*{NTIN071 A\&G: Cvičení 4 -- Uzávěrové vlastnosti regulárních jazyků}

\medskip

\subsection*{Cíle výuky:} Po absolvování student umí

\begin{itemize}\setlength{\itemsep}{0pt}  
    \item formálně popsat konstrukci automatu na základě jiných automatů  
    \item rozhodnout, zda jsou regulární jazyky uzavřené na různé množinové a řetězcové operace, včetně složitějších, a toto tvrzení dokázat nebo vyvrátit  
 \end{itemize}  
 

\section*{Příklady na cvičení}


\medskip\begin{problem}[Uzavřenost na množinové a řetězcové operace] 
    
    Pro danou dvojici DFA $A,B$ sestrojte automat, který rozpoznává daný jazyk. (Sestrojený automat formálně popište.)
    
    \begin{multicols}{3}

        \begin{enumerate}[(a)]\setlength\itemsep{6pt}
            \item $L(A)-L(B)$
            \item $L(A).L(B)$
            \item $L(A)^+$
            \item $L(A)^*$
            \item $L(A)^R$
        \end{enumerate}

        \begin{tabular}{ r | c c }
            A & a & b \\ \hline
            $\to$ 0 & 1 & 2 \\  
            $\ast$ 1 & 3 & 0 \\
            2 & 4 & 5 \\
            3 & 0 & 2 \\
            4 & 2 & 5 \\
            5 & 0 & 3
        \end{tabular}    
        
        \begin{tabular}{ r | c c }
            B & a & b \\ \hline
            $\to$ 0 & 0 & 5 \\  
            $\ast$ 1 & 1 & 3 \\
            2 & 2 & 5 \\
            3 & 3 & 2 \\
            $\ast$ 4 & 6 & 1 \\
            5 & 5 & 1 \\
            $\ast$ 6 & 4 & 2
        \end{tabular}

    \end{multicols}
    
\end{problem}


\begin{problem}[Mazání]

    Mějme nějaký regulární jazyk $L$ nad abecedou $\Sigma=\{a,b\}$. Popište následující jazyky v množinovém zápisu. Rozhodněte, zda jsou (nutně) také regulární, dokažte nebo vyvraťte. Jazyk sestávající ze všech slov vzniklých ze slov jazyka $L$\dots
    
    \medskip
    \begin{enumerate}[(a)]\setlength\itemsep{12pt}
            \item \dots smazáním všech výskytů písmene $a$.
            \item \dots smazáním počátečního písmene a zapsáním tohoto písmene na konec slova.
            \item \dots smazáním nejdelší souvislé posloupnosti $a$ček ze začátku slova.
    \end{enumerate}

\end{problem}


\section*{K procvičení a k zamyšlení}


\medskip\medskip\begin{problem}[Prefixy]
    Jsou regulární jazyky uzavřené na následující operace? Dokažte nebo vyvraťte. (V následujícím je $L$ regulární jazyk nad abecedou $\Sigma$.)
    \begin{enumerate}[(a)]
        \item $\mathrm{init}(L)=\{w\in\Sigma^*\mid \text{existuje }u\in\Sigma^*\text{ takové, že }wu\in L\}$
        \item $\min(L)=\{w\in L\mid \text{neexistují }u\in L,v\in\Sigma^+\text{ takové, že }w=uv\}$
        \item $\max(L)=\{w\in L\mid \text{neexistuje }u\in\Sigma^+\text{ takové, že }wu\in L\}$
    \end{enumerate}
\end{problem}


\medskip\begin{problem}[Posun]
    Pro daný regulární jazyk $L$ nad abecedou $\Sigma$ definujme jazyk $L'$ následovně. Je jazyk $L'$ nutně také regulární?
    $$
    L'=\{uv\mid u,v\in\Sigma^*,vu\in L\}
    $$
\end{problem}


\medskip\begin{problem}[Řez]
    Mějme dva regulární jazyky $L,M$ a definujme jazyk $K$ následovně. Je jazyk $K$ nutně také regulární?
    $$
    K=\{uw\mid u,w\in\Sigma^*, (\exists v\in M)\, uvw\in L\}
    $$    
\end{problem}
    
\medskip\begin{problem}[Záměna přijímajících a nepřijímajících stavů]        
    Zaměníme-li u daného NFA přijímající a nepřijímající stavy, bude jazyk přijímaný výsledným automatem doplňkem jazyka přijímaného původním automatem? Zdůvodněte.        
\end{problem}

\medskip\begin{problem}[Iterace unárních jazyků]  
    
    Ukažte, že pro libovolný jazyk $L$ nad abecedou $\Sigma=\{a\}$ je jazyk $L^*$ regulární.

    % Nápověda: Použijte Bézoutovu větu.

\end{problem}

   

\end{document}
