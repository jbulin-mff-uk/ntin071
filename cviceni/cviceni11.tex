\documentclass[a4paper,12pt]{amsart}

\usetheme[progressbar=frametitle]{metropolis}
\metroset{block=fill}

\subtitle{NTIN071 Automata and Grammars}
\author{Jakub Bulín (KTIML MFF UK)}

\date{Spring 2025\\ 
    \vspace{1in} 
    \begin{flushleft}
        \it \footnotesize * Adapted from the Czech-lecture slides by Marta Vomlelová with gratitude. The translation, some modifications, and all errors are mine.
    \end{flushleft}
}

%% packages

\usepackage{amsmath}
\usepackage{amssymb}
\usepackage{amsthm}
\usepackage{cancel}
\usepackage{color}
\usepackage{colortbl}
\usepackage{forest}
\usepackage[utf8x]{inputenc}
\usepackage{multicol}
\usepackage{multirow}

%% colors
\definecolor{Gray}{gray}{0.9}

%% TikZ
\usepackage{tikz}
    \usetikzlibrary{
        automata,
        arrows,
        backgrounds,
        decorations.pathmorphing,
        fit,
        positioning,
        shapes,
        shapes.geometric,
        tikzmark
    } 
    \tikzset{>=stealth',shorten >=1pt,auto,node distance=2cm}
    \tikzset{initial text={}}
    \tikzset{elliptic state/.style={draw,ellipse}}

%% amsthm
\theoremstyle{plain}
    \newtheorem*{algorithm}{Algorithm}    
    \newtheorem*{observation}{Observation}
    \newtheorem*{proposition}{Proposition}

\theoremstyle{remark}
    \newtheorem*{exercise}{Exercise}
    \newtheorem*{remark}{Remark}

%% macros
\DeclareMathOperator{\RegE}{RegE}
\DeclareMathOperator{\RL}{RL}

% Just for Lecture 2
\newcommand{\x}{$\times$}
\newcommand{\nx}{\ }


\begin{document}

\thispagestyle{empty}

\section*{NTIN071 A\&G: Cvičení 11 -- Turingovy stroje}

\medskip

\subsection*{Cíle výuky:} Po absolvování student umí

\begin{itemize}\setlength{\itemsep}{0pt}
    \item vysvětlit formální definici deterministického i nedeterministického Turingova stroje
    \item popsat graf konfigurací, definovat rozpoznávaný jazyk, počítanou funkci
    \item provést výpočet daného Turingova stroje na daném vstupu
    \item rozpoznat, jaký jazyk rozpoznává daný Turingův stroj
    \item zkonstruovat Turingův stroj rozpoznávající daný jazyk nebo počítající danou funkci
    \item analyzovat různé varianty a modifikace výpočetního modelu Turingova stroje
\end{itemize}

\section*{Příklady na cvičení}

\begin{problem}[Turingův stroj]
    
    Uvažme Turingův stroj popsaný následující tabulkou:
    
    \begin{table}[h]
    \begin{tabular}{r|cccc}
    & B   & a    & b    &  c  \\ \hline
    $\to q_0$ & $(q_1, B, L)$ & $(q_0, a, R)$ & $(q_0, c, R)$ & $(q_0, c, R)$ \\
    $q_1$ & $(q_2, B, R)$ & $(q_1, c, L)$ &  & $(q_1, b, L)$ \\
    $\ast q_2$  &              &              &              &             
    \end{tabular}
    \end{table}
    
    \begin{enumerate}[(a)]
        \item Nakreslete stavový diagram.
        \item Popište výpočet (posloupností konfigurací) na vstupu $w=aabca$.
        \item Jaký jazyk rozpoznává? Jakou funkci počítá?
    \end{enumerate}

\end{problem}


\begin{problem}[Smaž jedničky]

    Navrhněte Turingův stroj nad abecedou $\{0,1\}$, který ze vstupu smaže všechny 1ky a vrátí se na začátek (např. začne-li v konfiguraci $q_0 0011010$, skončí v $q_F 0000$ pro nějaké $q_F\in F$).

\end{problem}


\begin{problem}[Předchůdce]

    Sestrojte Turingův stroj $T$, který pro dané vstupní přirozené číslo $x>0$ v binárním zápisu spočte jeho předchůdce, tj. $x-1$, v binárním zápisu (a vrátí čtecí hlavu na začátek). %Dále:
    
    \smallskip
    \begin{enumerate}[(a)]    
        %\item $T$ formálně popište, není ale nutné uvádět tabulku přechodové funkce, stačí stavový diagram (viz (b)).
        \item Nakreslete stavový diagram $T$.
        \item Napište posloupnost konfigurací, kterými $T$ projde při výpočtu pro vstup $w=10100$.
    \end{enumerate}

    Vytvořte deterministický, jednopáskový, jednostopový stroj (chcete-li např. dvoustopový, musíte jej sami naprogramovat.) Číslo v binárním zápisu nesmí začínat nulou, pokud není rovno 0. Příklady vstupních a výstupních konfigurací: 
        
    \begin{itemize}    
        \item z konfigurace $q_01$ skončíme v $q0$ pro nějaké $q\in F$,
        \item z konfigurace $q_01001$ skončíme v $q1000$ pro nějaké $q'\in F$,
        \item z konfigurace $q_0100$ skončíme v $q11$ pro nějaké $q''\in F$.
    \end{itemize}
    
\end{problem}

\begin{problem}[Jednostraně nekonečná páska]

    Popište jak převést Turingův stroj s (jednou) oboustranně nekonečnou páskou na stroj, jehož páska je nekonečná jen v jednom směru, doprava. (Můžete předpokládat, že v prvním poli je speciální znak $\triangleright$.)

\end{problem}

\medskip\begin{problem}[Nedeterministický test neprvočíselnosti]
    
    Navrhněte nedeterministický Turingův stroj, který přijme jazyk $L=\{1^n \mid\text{$n$ není prvočíslo}\}$.

\end{problem}
    

\section*{K procvičení a k zamyšlení}


\medskip\begin{problem}[Programování Turingových strojů]

    Navrhněte Turingův stroj, který přijme jazyk $L$ Napište posloupnost konfigurací, která ukazuje, že přijmeme slovo $w$. 
    
    \begin{multicols}{2}
    \begin{enumerate}[(a)]
        % \item $L=\{0^n1^n\mid n\geq 0\}$, $w=0011$
        \item $L=\{0^n1^n2^n\mid n\geq 0\}$, $w=001122$
        % \item $L=\{0^i1^j\mid i\leq j\}$, $w=00111$
        \item $L=\{w\in\{0,1\}^*\mid |w|_0=|w|_1\}$,\\ $w=100110$
        \item $L=\{ucu^R\mid u\in\{0,1\}^*\}$, $w=10c01$
        % \item $L=\{uu^R\mid u\in\{0,1\}^*\}$, $w=101101$
        \item $L=\{ucu\mid u \in\{0,1\}^*\}$, $w=10c10$
        \item $L=\{uu\mid u \in\{0,1\}^*\}$, $w=110110$
    \end{enumerate}
    \end{multicols}

\end{problem}

\begin{problem}[Zrcadlení]
    
    Navrhněte Turingův stroj, který ze zadaného vstupního slova vytvoří jeho zrcadlový obraz.

\end{problem}


\begin{problem}[Paměťové bloky]

    Navrhněte Turingův stroj, který prohodí obsah dvou paměťových bloků. Konkrétně, počáteční konfiguraci $q_0u\#v\#w\#x\#y$ (kde $u, v, w, x, y \in \Sigma\setminus\{\#\}$) převede na $fu\#x\#w\#v\#y$ pro nějaké $f\in F$. %Pokuste se zkonstruovat co nejmenší a nejefektivnější stroj.

\end{problem}
    

\begin{problem}[Pohyby hlavy]
    
    Uvažte modifikace Turingova stroje, kde jsou povoleny následující pohyby hlavy. Jaké třídy jazyků rozpoznávají?
    \begin{multicols}{2}
        \begin{enumerate}[(a)]
            \item left (L), right (R)
            \item stay (N), right (R),
            \item stay (N), (L),
            \item left (L), right (R), a stay (N).
        \end{enumerate}    
    \end{multicols}

\end{problem}


\begin{problem}[Jen dvě akce najednou]

    Ukažte, že každý jednopáskový Turingův stroj $M$ lze převést na stroj $M'$, který smí provést jen dvě z následujících tří akcí najednou, tj. instrukce může: 
    \begin{itemize}
        \item změnit stav a pozici hlavy,
        \item změnit stav a přepsat symbol na pásce, nebo
        \item změnit pozici hlavy a přepsat symbol na pásce
    \end{itemize}
    ale žádná instrukce nesmí provést všechny tři akce najednou.

\end{problem}


\begin{problem}[Doprava nebo restartuj]
    Uvažte modifikaci Turingova stroje, kde páska je nekonečná jen v jednom směru (doprava), a hlava může provést dva typy pohybů: right (R) nebo RESTART (návrat na první políčko). Ukažte, jak převést standardní jednopáskový stroj na tento typ stroje.
\end{problem}
        

\begin{problem}[Přepiš jen jednou]

    Uvažte jednopáskový Turingův stroj, který smí přepsat každé políčko na pásce nejvýše jednou. Ukažte, že tento výpočetní model je ekvivalentní standardnímu Turingovu stroji.

\end{problem}
    

\begin{problem}[Nepřepisuj vstup]

    Vysvětlete, proč zakážeme-li Turingovu stroji měnit políčka obsahující vstup, bude ekvivalentní konečnému automatu. (Takové stroje tedy rozpoznávají jen regulární jazyky.) Stačí popsat myšlenku.

\end{problem}
    

\begin{problem}[Uzávěrové vlastnosti]
    
    Ukažte, že jak \emph{rekurzivní} tak \emph{rekurzivně spočetné} jazyky jsou uzavřené na:    
    (a) \emph{sjednocení}, (b) \emph{průnik}, (c) \emph{konkatenaci}, (d) \emph{iteraci}.

    Ukažte, že (e) \emph{rekurzivní jazyky jsou uzavřené na doplněk}, ale (f) \emph{rekurzivně spočetné ne}.
    % Navíc ukažte, že
    % \begin{enumerate}[(a)]
    % \setcounter{enumi}{4}
    % \item rekurzivní jazyky jsou uzavřené na doplněk, ale 
    % \item rekurzivně spočetné ne.
    % \end{enumerate}

\end{problem}


\end{document}
