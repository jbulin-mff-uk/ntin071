\documentclass[a4paper,12pt]{amsart}

\usetheme[progressbar=frametitle]{metropolis}
\metroset{block=fill}

\subtitle{NTIN071 Automata and Grammars}
\author{Jakub Bulín (KTIML MFF UK)}

\date{Spring 2025\\ 
    \vspace{1in} 
    \begin{flushleft}
        \it \footnotesize * Adapted from the Czech-lecture slides by Marta Vomlelová with gratitude. The translation, some modifications, and all errors are mine.
    \end{flushleft}
}

%% packages

\usepackage{amsmath}
\usepackage{amssymb}
\usepackage{amsthm}
\usepackage{cancel}
\usepackage{color}
\usepackage{colortbl}
\usepackage{forest}
\usepackage[utf8x]{inputenc}
\usepackage{multicol}
\usepackage{multirow}

%% colors
\definecolor{Gray}{gray}{0.9}

%% TikZ
\usepackage{tikz}
    \usetikzlibrary{
        automata,
        arrows,
        backgrounds,
        decorations.pathmorphing,
        fit,
        positioning,
        shapes,
        shapes.geometric,
        tikzmark
    } 
    \tikzset{>=stealth',shorten >=1pt,auto,node distance=2cm}
    \tikzset{initial text={}}
    \tikzset{elliptic state/.style={draw,ellipse}}

%% amsthm
\theoremstyle{plain}
    \newtheorem*{algorithm}{Algorithm}    
    \newtheorem*{observation}{Observation}
    \newtheorem*{proposition}{Proposition}

\theoremstyle{remark}
    \newtheorem*{exercise}{Exercise}
    \newtheorem*{remark}{Remark}

%% macros
\DeclareMathOperator{\RegE}{RegE}
\DeclareMathOperator{\RL}{RL}

% Just for Lecture 2
\newcommand{\x}{$\times$}
\newcommand{\nx}{\ }


\begin{document}

\thispagestyle{empty}

\section*{NTIN071 A\&G: Cvičení 1 -- Deterministický konečný automat, rozpoznávaný jazyk, regulární jazyky}

\medskip

\subsection*{Cíle výuky:} Po absolvování student umí

\begin{itemize}\setlength{\itemsep}{0pt}
    \item používat základní terminologii a notaci z teorie formálních jazyků a automatů
    \item vysvětlit formální definici konečného deterministického automatu (DFA) a rozpoznávaného jazyka
    \item popsat jazyk rozpoznávaný daným DFA, pomocí množinového zápisu
    \item sestrojit (a formálně popsat) DFA rozpoznávající daný jazyk
    \item dokázat uzavřenost regulárních jazyků na základní množinové operace
\end{itemize}

\section*{Příklady na cvičení (think-pair-share)}

\medskip

\medskip\begin{problem} [Konstrukce DFA pro daný jazyk]
    
    Sestrojte DFA rozpoznávající daný jazyk.

    \medskip
    
    \begin{enumerate}[(a)]\setlength\itemsep{6pt}        
        \item $L=\{w\in\{a,b\}^* \,\mid\, |w|_a \text{ není dělitelný 3}\}$
        \item $L=\{w\in\{a,b\}^* \,\mid\,\text{2 dělí $|w|_a$ nebo 3 dělí $|w|_b$}\}$ 
        \item $L=\{w\in\{a,b\}^* \,\mid\, \text{2 dělí $|w|_a$ a 3 dělí $|w|_b$}\}$
        \item $L=\{w\in\{0,1\}^* \,\mid\, w\text{ je binární zápis přirozeného čísla dělitelného 3}\}$        
    \end{enumerate}

\end{problem}


\medskip\begin{problem}[Automat zadaný tabulkou]

    Nakreslete stavový diagram a popište rozpoznávaný jazyk v množinovém zápisu.

    \begin{multicols}{2}
        
        \begin{enumerate}[(a)]

            \item \begin{tabular}{ c | c c }
            & 0 & 1 \\   \hline
            $\to$ p & q & p \\  
            $\ast$ q & r & q\\
            $\ast$ r & p & r
            \end{tabular}
    
        \columnbreak
    
            \item \begin{tabular}{ c | c c }
            & 0 & 1 \\   \hline
            $\to$ p & p & q \\  
            $\ast$ q & r & q\\
            $\ast$ r & p & q
            \end{tabular}

        \end{enumerate}

    \end{multicols}

\end{problem}


\medskip\begin{problem}[Popis jazyka a konstrukce automatu pro danou vlasnost]

    Sestrojte DFA přijímající právě všechna slova nad abecedou $\Sigma=\{a,b\}$, která splňují danou vlastnost. Popište daný jazyk pomocí množinového zápisu.

    \medskip
    
    (a) začíná `abba'\hfill (b) končí `abba'\hfill (c) obsahuje `abba' nebo `bab' jako podslovo
\end{problem}


\medskip\begin{problem}[Regulární jazyky a množinové operace]

    Mějme dva regulární jazyky, $L,L'$ nad stejnou abecedou. Ukažte, že platí následující:
  
    \medskip
    
    \begin{enumerate}[(a)]\setlength\itemsep{6pt}
        \item $\Sigma^*\setminus L$ je regulární jazyk
        \item $L\cup L'$ je regulární jazyk
        \item $L\cap L'$ je regulární jazyk       
    \end{enumerate}
      
\end{problem}


\section*{K procvičení a k zamyšlení}


\medskip\begin{problem}
    
    Sestrojte DFA rozpoznávající daný jazyk.

    \medskip
    
    \begin{enumerate}[(a)]\setlength\itemsep{6pt}
        \item $L=\{w\in\{a,b\}^* \,\mid\, |w|_a \text{ je sudý}\}$
        \item $L=\{w\in\{a,b\}^* \,\mid\, |w|_b \text{ je dělitelný 3}\}$
        \item $L=\{w\in\{a,b\}^* \,\mid\, \text{2 nebo 3 dělí }|w|_a\}$
        \item $L=\{w\in\{a,b\}^* \,\mid\, \text{2 a 3 dělí }|w|_a\}$
        \item $L=\{w\in\{0,1\}^* \,\mid\, w\text{ je binární zápis přirozeného čísla dělitelného 5}\}$       
    \end{enumerate}

\end{problem}


\medskip\begin{problem}

    Nakreslete stavový diagram a popište rozpoznávaný jazyk v množinovém zápisu.

    \begin{multicols}{2}
    
        \begin{enumerate}[(a)]    
            \item \begin{tabular}{ c | c c }
            & 0 & 1 \\   \hline
            $\to$ $\ast$ p & q & p \\  
            q & r & q\\
            r & p & r
            \end{tabular}
    
        \columnbreak

            \item \begin{tabular}{ c | c c }
            & 0 & 1 \\   \hline
            $\to$ p & p & q \\  
            q & p & r\\
            $\ast$ r & p & r
            \end{tabular}

        \end{enumerate}

    \end{multicols}

\end{problem}


\medskip\begin{problem}

    Sestrojte DFA rozpoznávající jazyk všech slov nad abecedou $\Sigma=\{a,b\}$ splňujících danou vlastnost:

    \medskip
    
    \begin{enumerate}[(a)]\setlength\itemsep{6pt}
        \item má alespoň dvě písmena a první písmeno je stejné jako poslední
        \item má alespoň dvě písmena a první dvojice písmen je stejná jako poslední dvojice písmen
    \end{enumerate}

\end{problem}


\medskip\begin{problem}

    Co když jsou $L,L'$ regulární jazyky nad různými (ale ne nutně disjunktními) abecedami? Jsou jazyky $\Sigma^*\setminus L$, $L\cup L'$, $L\cap L'$ nutně také regulární?

\end{problem}
  

\medskip\begin{problem}

    Uměli byste ukázat, že jazyk $L^R$ (tj. slova z $L$ napsaná pozpátku) je regulární kdykoliv je $L$ regulární?
      
\end{problem}

\end{document}

